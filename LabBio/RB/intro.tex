\section{Introdução} 
O desenvolvimento de técnicas de biologia molecular e engenharia genética
revolucionou completamente a biologia celular moderna, e tornou possível a
criação de ferramentas extremamente úteis e, hoje em dia, indispensáveis para a
humanidade\cite{teófilo_gallão_2019}, como as ferramentas de edição genética e
de visualização molecular \textit{in vivo}.

As ferramentas mais atuais de edição genética se baseiam na substituição ou
adição de sequências específicas de DNA em plasmídeos
vetores\cite{BriefyHistoryOfGenetics}. Os quais são inseridos em organismos
alvos para que sejam expressados certas característica ou funções de interesse.
Podendo tal expressão ocorrer diretamente no organismo procarionte que vai
possuir o plasmídeo\cite{ruiz_silhavy_2022}, ou utilizá-lo como vetor para
infecção de organismos mais complexos, e assim, fazer com que o segmento
desejado se acople no DNA do organismo alvo. No contexto vegetal, a edição
genética se baseia na introdução de genes exógenos, que podem ser introduzidos
nas células vegetais utilizando micro-organismos modificados que carreguem
plasmídios possuidores dos genes de interesse\cite{embrapa2017manual}. Dessa
forma, uma técnica muito comum para a introdução desses genes é a
agroinfiltração, que pode usar vetores patogênicos para plantas, como a
\textit{Agrobacterium tumefaciens}, para infectá-las com o gene de interesse e
acoplá-lo ao DNA da planta e fazê-la expressar certas
características\cite{embrapa2017manual}.

Em complemento, outra ferramenta crucial para o desenvolvimento da biologia
celular moderna são as técnicas de visualização molecular, dentre elas, uma que
se destaca por sua praticidade e por sua revolução na área (tendo recebido um
Prêmio Nobel de Química em 2008 para os descobridores e desenvolvedores da
proteína fluorescente verde (GFP)\cite{jeremy_jackson_2009}) é a utilização de
proteínas fluorescentes como marcadores de fácil visualização. Proteínas
fluorescentes como a GFP (\textit{Green Fluorescent Protein}), BFP (\textit{Blue
Fluorescent Protein}) e a mCherry podem ser acopladas a proteínas de interesse
possibilitando a realização de análise em tempo real, por microscopia de
fluorescência ou por meio de um espectrofotômetro.  Essa ferramenta de
visualização molecular é muito utilizada para a marcação de proteínas,
organelas e vias metabólicas, possibilitando a análise dinâmica de processos
biológicos em tempo real, afim de fornecer a melhor compreensão do funcionamento
celular e monitoramento de alterações em trabalhos de engenharia
genéticas\cite{misteli_spector_1997}. 

Em organismos modelo, como \textit{Escherichia coli}, essa expressão de
proteínas fluorescentes têm sido vastamente explorada, sendo utilizada em
estudos de regulação gênica, interações entre proteínas e até em otimizações de
métodos de edição genética para utilização em outros
organismos\cite{ruiz_silhavy_2022}. 

No presente estudo, serão utilizadas essas duas ferramentes, acopladas com
outras técnicas de biologia molecular e celular, para implementar os genes de
expressão das proteínas fluorescentes em células procarionte e eucarionte
vegetais. Serão inseridos genes de expressão para GFP, mCherry e BFP, a partir
de plasmídeos geneticamente modificados, em culturas de bactérias \textit{E.
coli}. Em paralelo, também serão realizados testes de agroinfiltração, dos genes
de expressão do GFP com localização nuclear em folhas de \textit{Nicotiana
benthamiana}, \textit{Nicotiana tabacoa} e \textit{Solanum lycopersicum}, por
meio de uma cepa atenuada de \textit{Agrobacterium tumefaciens} contendo um
plasmídeo binário, isto é, um plasmídeo contendo T-DNA (a ser transferido para a
planta) apenas com os genes de interesse e com o sistema CRISPR-Cas9 para
inserção do T-DNA no gene TNT1 que há nas plantas
modelo\cite{hernández-pinzón_2012}.

O objetivo geral deste experimento é aplicar metodologias de biologia molecular
e celular voltadas à expressão e detecção de proteínas fluorescentes em sistemas
procariontes e eucariontes vegetais. Especificamente, pretende-se inserir
plasmídeos contendo genes para GFP, BFP e mCherry em culturas de
\textit{Escherichia coli}  e analisar a expressão qualitativa dessas proteínas
por meio de microscopia de fluorescência. Também, visa-se realizar a
agroinfiltração do gene de expressão da GFP com localização nuclear fusionada à Cas9 em folhas
de espécies vegetais modelo por meio de \textit{Agrobacterium tumefaciens}, e
posteriormente avaliar a expressão GFP no núcleo da células vegetais, de modo a
verificar o sucesso da inserção gênica. 
