\section{Materiais e métodos}
\subsection{Materiais e equipamentos}
Os materiais e equipamentos necessários incluem: solução de ágar; estufa
\textit{Shaker Series} (Innova 44, eppendorf, S.l.); alça de Drigalski de vidro;
banho-maria; células competentes de \textit{Escherichia coli}; DNA plasmidial
GFP, mCherry e mTAGbfp2; gelo; lamínulas e lâminas de microscópio; meio SOC;
microscópio de fluorescência; micropipetas p20, p200 e p1000 (pipetman, GILSON,
França); pontas estéreis; placas de Petri estéreis; tubos de microcentrífuga e
tubos de fundo cônico.

\subsection{Métodos}
\subsubsection {Transformação genética de plantas por meio de Agroinfiltração (\textit{Agrobacterium tumefaciens})}

Para a realização deste experimento são necessários(as): Acetoceringona \qty{150}{\micro M}; Água destilada; Centrifuga; Espécime saudável e madura de \textit{Nicotiana benthamiana}; Espécime saudável e madura de \textit{Nicotiana tabacaum}; Espécime saudável e madura de \textit{Solanum lycopersicum}; Espectinomincina;
Espectrofotômetro; Estreptomicina; Estufa Shaker Series (Innova 44, eppendorf, ); Lâminas de corte; Laminas de vidro e laminulas; Linhagem de Agrobacterium (EHA 105) transformada com um vetor binário (2 RNAs guia com alvo para TNT1); Meio LB; Meio YEB; MES \qty{10}{mM}; $MgCl_2$ \qty{10}{mM}; Micropipeta P 200 (Pipetman, Gilson, França); Microscópio de Fluorescência; Palito de dente autoclavado; Pedaços de isopor; Pinça; Placa de Petri estéreis; Seringa; Tubos Falcon de \qty{15}{mL}; Vetores de Cas9/GFP com localização nuclear; Vidro de relógio; 


\subsection{Métodos}
\subsubsection{Transformação genética de \textit{Escherichia coli} com proteínas
fluorescentes}
A transformação genética de \textit{Escherichia coli} foi realizada três vezes,
cada uma com uma solução diferentes de DNA plasmidial: GFP, mCherry e
mTAGbfp2. Inicialmente, foram adicionados \qty{2}{\micro\liter} de DNA do
plasmídeo ao tubo contendo às células competentes, previamente tratadas. Após
agitação delicada, a mistura foi transferida para um tubo de fundo cônico e
identificado. A mistura foi levada para um banho de felo e mantida por 30
minutos sem agitação. Depois, foi levada para um banho a 42°C por 90 segundos
cronometrados, ainda com cuidado para não agitar. Em seguida, a mistura voltou
ao banho de gelo por 1 a 2 minutos, sem agitação. Foram adicionados
\qty{800}{\micro\liter} de meio SOC à mistura e esta foi levada ao banho maria,
agora a 37°C. Após realizar este preparo com todas as soluções de DNA
plasmidial, as misturas foram levadas ao \textit{shaker} a 37°C sob agitação de
250 rpm. Em seguida, foram separadas três placas de seleção com antibiótico, em
cada uma foram espalhadas \qty{200}{\micro\liter} da mistura. Cada placa foi
identificada e deixada para incubar \textit{overnight} a 37°C. No dia seguinte, foi
verificado o crescimento das bactérias em cada placa, retirado as colônias e
transferido para lâminas de microscópio, cobrindo com uma lamínula. Por fim, as
lâminas foram levadas ao microscópio de fluorescência nas faixas de luz
apropriadas, onde foram retiradas as imagens.
\subsubsection{Transformação genética de plantas por meio de Agroinfiltração (\textit{Agrobacterium tumefaciens})}
As linhagens de \textit{Agrobacterium} (EHA 105) transformadas com um vetor binário (2 RNAs guia com alvo para TNT1) foram inoculadas em placas de Petri contendo um meio YEB suplementado com os antibióticos apropriados e incubadas a \qty{28}{^\circ C} por 48 horas. Então, foram transferidas duas a três colônias isoladas da bacteria, com um palito de dente autoclavado, para tubos Falcon de \qty{15}{mL} com \qty{3}{mL} de meio LB acrescido de estreptomicina e espectinomicina, seguindo incubação sob agitação a \qty{28}{^\circ C} por 12 a 16 horas. Seguidamente, aliquotas de \qty{30}{\micro L} de cada cultura, foram repicadas para tubos Falcon contendo \qty{10}{mL} de LB com antibióticos apropriados, e novamente, foram incubados as mesmas condições anteriores. Ao fim da incubação, centrifugou-se os tubos com as culturas a \qty{4000}{RPM} (\qty{3200}{g}) por \qty{20}{min}, então o pellet foi ressuspenso em \qty{10}{mL} de tampão de infiltração (acetosiringona \qty{150}{\micro M}, $MgCl_2$ \qty{10}{mM}, MES \qty{10}{mM} a um pH de 5,5) e diluiu-se \qty{1}{mL} da suspensão até que a medida da absorbância a 600 nm ($A_600$) em um espectofotometro (especificar modelo)  atingisse $A_600 = 0,5$.
A agroinfiltração da suspensão das bactérias foi realizada na superficie abaxial de folhas de \textit{Nicotiana benthamiana}, \textit{Nicotiana tabacum} e \textit{Solanum lycopersicum} utilizanodo uma seringa estéril, com as plantas mantidas em casa de vegetação por 2 dias. Após esse período, as folhas foram coletadas, seccionada transversalmente com lâminas de barbear e os cortes mais finos foram montados em lâminas com água destilada. Essas lâminas foram analisadas em um microscópio de fluorescência sob o filtro GFP, e as melhores imagens dos resultados forma coletadas e analisadas por meio do programa Fiji.  
