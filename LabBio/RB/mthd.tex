\section{Materias e métodos}
\subsection{Materiais}

\subsection{Métodos}
\section{Materiais e métodos}
\subsection{Materiais e equipamentos}
\subsubsection{Transformação genética de \textit{Escherichia coli} com proteínas
	fluorescentes}
\begin{itemize}[label=\boldmath\(\cdot\)]%TODO: Revisar e adicionar detalhes
    \item Ágar %detalhar
    \item Agitador \textit{shaker} %detalhar
    \item Alça de Drigalski de vidro %detalhar
    \item Banho-maria %detalhar
    \item Células competentes de \textit{Escherichia coli} %detalhar
    \item DNA plasmidial GFP
    \item DNA plasmidial mCherry
    \item DNA plasmidial mTAGbfp2
    \item Estufa %detalhar
    \item Gelo %detalhar
    \item Lamínulas e lâminas de microscópio %detalhar
    \item Meio SOC %detalhar
    \item Microscópio de fluorescência %detalhar
    \item Pipetas e pontas estéreis %detalhar
    \item Placas de Petri estéreis %detalhar
    \item Tubos de microcentrífuga %detalhar
    \item Tudo de fundo cônico %detalhar
\end{itemize}

\subsubsection {Transformação genética de plantas por meio de Agroinfiltração (\textit{Agrobacterium tumefaciens})}

	Para a realização deste experimento são necessários(as): Acetoceringona \qty{150}{\micro M}; Água destilada; Centrifuga; Espécime saudável e madura de \textit{Nicotiana benthamiana}; Espécime saudável e madura de \textit{Nicotiana tabacaum}; Espécime saudável e madura de \textit{Solanum lycopersicum}; Espectinomincina;
	 Espectrofotômetro; Estreptomicina; Estufa Shaker Series (Innova 44, eppendorf, ); Lâminas de corte; Laminas de vidro e laminulas; Linhagem de Agrobacterium (EHA 105) transformada com um vetor binário (2 RNAs guia com alvo para TNT1); Meio LB; Meio YEB; MES \qty{10}{mM}; $MgCl_2$ \qty{10}{mM}; Micropipeta P 200 (Pipetman, Gilson, França); Microscópio de Fluorescência; Palito de dente autoclavado; Pedaços de isopor; Pinça; Placa de Petri estéreis; Seringa; Tubos Falcon de \qty{15}{mL}; Vetores de Cas9/GFP com localização nuclear; Vidro de relógio; 


\subsection{Métodos}
\subsubsection{Transformação genética de \textit{Escherichia coli} com proteínas
fluorescentes}
\begin{enumerate} %TODO: Revisar e adicionar detalhes
    \item Adicione \qty{2}{\micro\liter} de DNA do plasmídeo às células competentes;
    \item Agite delicadamente o tubo para misturar as células ao DNA;
    \item Transferir a mistura DNA + células competentes para um tubo de fundo cônico de 15 ml, previamente identificado;
    \item Transfira a mistura para o gelo e mantenha por 30 min (não agite);
    \item Transfira para o banho a 42°C por exatamente 90 seg (não agite);
    \item Transfira novamente para o gelo por 1-2 min (não agite);
    \item Adicionar 800 microlitros de meio SOC à mistura;
    \item Transfira para banho maria a 37°C;
    \item Transfira para o shaker a 37°C sob agitação de 250 rpm;
    \item Separe as placas de seleção com antibiótico;
    \item Espalhe \qty{200}{\micro\liter} da mistura de células e DNA na placa de seleção e incube overnight a 37°C;
    \item Verifique o crescimento das bactérias;
    \item Retire as colônias e transferir para lâminas e cobrir com uma lamínula;
    \item Realize a aquisição de imagens no microscópio de fluorescência nos canais de apropriados
\end{enumerate}
\subsubsection{Transformação genética de plantas por meio de Agroinfiltração (\textit{Agrobacterium tumefaciens})}
As linhagens de \textit{Agrobacterium} (EHA 105) transformadas com um vetor binário (2 RNAs guia com alvo para TNT1) foram inoculadas em placas de Petri contendo um meio YEB suplementado com os antibióticos apropriados e incubadas a \qty{28}{^\circ C} por 48 horas. Então, foram transferidas duas a três colônias isoladas da bacteria, com um palito de dente autoclavado, para tubos Falcon de \qty{15}{mL} com \qty{3}{mL} de meio LB acrescido de estreptomicina e espectinomicina, seguindo incubação sob agitação a \qty{28}{^\circ C} por 12 a 16 horas. Seguidamente, aliquotas de \qty{30}{\micro L} de cada cultura, foram repicadas para tubos Falcon contendo \qty{10}{mL} de LB com antibióticos apropriados, e novamente, foram incubados as mesmas condições anteriores. Ao fim da incubação, centrifugou-se os tubos com as culturas a \qty{4000}{RPM} (\qty{3200}{g}) por \qty{20}{min}, então o pellet foi ressuspenso em \qty{10}{mL} de tampão de infiltração (acetosiringona \qty{150}{\micro M}, $MgCl_2$ \qty{10}{mM}, MES \qty{10}{mM} a um pH de 5,5) e diluiu-se \qty{1}{mL} da suspensão até que a medida da absorbância a 600 nm ($A_600$) em um espectofotometro (especificar modelo)  atingisse $A_600 = 0,5$.
A agroinfiltração da suspensão das bactérias foi realizada na superficie abaxial de folhas de \textit{Nicotiana benthamiana}, \textit{Nicotiana tabacum} e \textit{Solanum lycopersicum} utilizanodo uma seringa estéril, com as plantas mantidas em casa de vegetação por 2 dias. Após esse período, as folhas foram coletadas, seccionada transversalmente com lâminas de barbear e os cortes mais finos foram montados em lâminas com água destilada. Essas lâminas foram analisadas em um microscópio de fluorescência sob o filtro GFP, e as melhores imagens dos resultados forma coletadas e analisadas por meio do programa Fiji.  
>>>>>>> Stashed changes
