\section{Materiais e métodos}
\subsection{Materiais e equipamentos}
Os materiais e equipamentos necessários incluem: solução de ágar; estufa
\textit{Shaker Series} (Innova 44, eppendorf, S.l.); alça de Drigalski de vidro;
banho-maria; células competentes de \textit{Escherichia coli}; DNA plasmidial
GFP, mCherry e mTAGbfp2; gelo; lamínulas e lâminas de microscópio; meio SOC;
microscópio de fluorescência; micropipetas p20, p200 e p1000 (pipetman, GILSON,
França); pontas estéreis; placas de Petri estéreis; tubos de microcentrífuga e
tubos de fundo cônico.

\subsection{Métodos}
\subsubsection{Transformação genética de \textit{Escherichia coli} com proteínas
fluorescentes}
A transformação genética de \textit{Escherichia coli} foi realizada três vezes,
cada uma com uma solução diferentes de DNA plasmidial: GFP, mCherry e
mTAGbfp2. Inicialmente, foram adicionados \qty{2}{\micro\liter} de DNA do
plasmídeo ao tubo contendo às células competentes, previamente tratadas. Após
agitação delicada, a mistura foi transferida para um tubo de fundo cônico e
identificado. A mistura foi levada para um banho de felo e mantida por 30
minutos sem agitação. Depois, foi levada para um banho a 42°C por 90 segundos
cronometrados, ainda com cuidado para não agitar. Em seguida, a mistura voltou
ao banho de gelo por 1 a 2 minutos, sem agitação. Foram adicionados
\qty{800}{\micro\liter} de meio SOC à mistura e esta foi levada ao banho maria,
agora a 37°C. Após realizar este preparo com todas as soluções de DNA
plasmidial, as misturas foram levadas ao \textit{shaker} a 37°C sob agitação de
250 rpm. Em seguida, foram separadas três placas de seleção com antibiótico, em
cada uma foram espalhadas \qty{200}{\micro\liter} da mistura. Cada placa foi
identificada e deixada para incubar \textit{overnight} a 37°C. No dia seguinte, foi
verificado o crescimento das bactérias em cada placa, retirado as colônias e
transferido para lâminas de microscópio, cobrindo com uma lamínula. Por fim, as
lâminas foram levadas ao microscópio de fluorescência nas faixas de luz
apropriadas, onde foram retiradas as imagens.
