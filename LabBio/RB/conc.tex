\section{Conclusão}
Os resultados obtidos demonstraram sucesso na transformação
genética de \textit{Escherichia coli} com os plasmídeos contendo gene para
expressão de  mCherry e
mTAGbfp2, comprovado pela fluorescência característica das colônias bacterianas.
Por outro lado, a falha na transformação com GFP foi associada à baixa
concentração da solução de DNA plasmidial utilizada no processo.

Já no caso da
transformação de plantas via agroinfiltração, os resultados variaram
dentre as três espécies analisadas. Enquanto \textit{Solanum
lycopersicum} apresentou expressão transgênica pontual, ainda que limitada,
\textit{Nicotiana benthamiana} e \textit{Nicotiana tabacum} não exibiram
fluorescência detectável.  Essa diferença pode estar relacionada a fatores como
o tempo insuficiente de incubação, erros na técnica de corte,  a escolha da
região infiltrada ou mesmo uma maior resistência natural das espécies de
\textit{Nicotiana} à infecção bacteriana.

%Diante desses resultados, fica evidente a necessidade de ajustes metodológicos
%para aumentar a eficiência da transformação, especialmente em \textit{Nicotiana spp.},
%que não responderam adequadamente sob as condições testadas. Estudos futuros
%deverão investigar parâmetros como concentração bacteriana, tempo de incubação e
%técnicas alternativas de infiltração para melhorar a expressão gênica nessas
%espécies.
%Dessa forma, estudos adicionais são necessários para otimizar o protocolo visando
%maior eficiência, particularmente nas espécies de \textit{Nicotiana}, que
%apresentaram baixa resposta sob as condições testadas. 
