\section{Discussões}
% A discussão não deveria ser junto dos resultados assim como as professoras sugeriram?
A análise por High Resolution Melting (HRM) demonstrou ser
uma ferramenta eficiente para a caracterização molecular das amostras
analisadas. Como observado nas curvas de melting (\cref{almeltc}) geradas a
partir das amostras padronizadas, o método permite a distinção clara de
variantes moleculares com base em suas temperaturas de melting (Tm) e no formato
das curvas de dissociação do DNA. A distinção entre os padrões de melting foi
reforçada pelas curvas derivadas (\cref{dmeltc}), que evidenciaram os valores
específicos de Tm para cada amostra, e pelas curvas diferenciais (\cref{diffp}),
que aumentaram a sensibilidade da análise ao destacar pequenas variações de
sequência.

Ao aplicar o mesmo método a um novo conjunto de amostras, contendo controles
positivos, negativos e amostras desconhecidas (\cref{meltc,amplip}), observou-se
uma correspondência clara entre os padrões de melting esperados e os resultados
de amplificação por PCR em tempo real. As curvas de melting das
amostras desconhecidas apresentou perfis compatíveis com os observados nas
amostras controle positivas. Isso sugere que os amplicons gerados destas
amostras são constituidos pela mesma sequência que os amplicons de
\textit{Leishmania sp.} padronizados. 

%TODO: adicionar discussão sobre os SNPs e como isso varia Tm e tal
