\section{Metodologia}
% Estou na dúvida se devo manter os resumos sucintos (como as professoras sugeriram) ou se devo expandi-los mas correr o risco de ou ultrapassar as
% 7 paginas ou de tratar de detalhes secundários e acabar perdendo ponto por isso. Acho que vou esperar para ver a metodologia do Mica para me decidir

\subsection{Extração de DNA}

A extração do DNA genômico foi realizada a partir de amostras biológicas
suspeitas de conter \textit{Leishmania}, utilizando um protocolo adaptado de
Miler, Dykes e Polesky\cite{SODEmiller1988}.  O procedimento consistiu em
etapas sequenciais de lise celular, ligação do DNA à matriz, lavagens para
remoção de impurezas e eluição final em tampão PBS, garantindo a obtenção de
material genético adequado para análises moleculares subsequentes.

\subsection{Espectrofotometria (NanoDrop)}

A quantificação e avaliação da pureza do DNA extraído foram realizadas por
espectrofotometria, utilizando o equipamento NanoDrop One (Thermo Scientific,
EUA). As amostras foram analisadas quanto à concentração (em ng/µL), sendo
consideradas adequadas aquelas com valores superiores a 100 ng/µL. A razão de
absorbância A$_{260}$/A$_{280}$ foi utilizada como indicador de contaminação por
proteínas, com valores ideais entre 1{,}8 e 2{,}0. Já a razão
A$_{260}$/A$_{230}$, superior a 1{,}8, indicou baixa presença de contaminantes
orgânicos e compostos fenólicos.

\subsection{Reação em Cadeia da Polimerase (PCR)}

A amplificação do DNA foi realizada por meio da Reação em Cadeia da Polimerase
(PCR), utilizando primers específicos para a detecção do agente etiológico. As
condições da reação foram otimizadas para obtenção de um fragmento de tamanho
previamente determinado. A confirmação da amplificação foi feita por
eletroforese em gel de agarose a 1,5\%, corado com intercalante de DNA UniSafe
(Uniscience), com posterior visualização sob luz ultravioleta

     
\subsection{qPCR e High Resolution Melting Analysis}

O protocolo foi adaptado de Zampieri et.al.\cite{HRMzampi2016}.  Foram
preparados 28 soluções de \qty{20}{\micro\liter} com \qty{10}{\micro\liter} de
MeltDoctorTM HRM Master Mix (Thermo fischer Scientific), \qty{0,6}{\micro\liter}
de Primer hsp70-F2, \qty{0,6}{\micro\liter} de Primer hsp70C-R,
\qty{6,3}{\micro\liter} de água e \qty{2,5}{\micro\liter} de amostra de DNA
genômico. Destas 28 soluções, oito receberam como amostra controle positivo,
sendo duas de \textit{L. (L.) infantum}, duas de \textit{L. (L.) amazonensis},
duas de \textit{L. (V.) braziliensis} e duas de \textit{L. (V.) shawi}. Duas das
soluções receberam controle negativo, sem DNA genômico. O restante das soluções
receberam as amostras desconhecidas denominadas amostras W, X, Y, Z, sendo
quatro soluções para W, quatro para  X e quatro para Y e seis soluções para Z.
As soluções foram colocadas em uma placa de 96 poços. A placa foi levada ao
equipamento StepOne System (Thermo Fischer Scientific) para realização da reação
de qPCR. Primeiro foi feita incubação à \qty{94}{\celsius} por 5 minutos.
Então, a reação foi submetida à 40 ciclos de desnaturação à \qty{94}{\celsius}
por 30 s seguida de associação e extensão dos amplicons  a \qty{60}{\celsius}
por 30s. 


