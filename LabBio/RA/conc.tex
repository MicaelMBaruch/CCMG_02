\section{Conclusão}

O protocolo adotado demonstrou alta eficácia na detecção do DNA de interesse diagnóstico, 
evidenciando sua aplicabilidade em contextos laboratoriais. A metodologia empregada possibilitou 
a extração e amplificação bem-sucedidas do material genético, garantindo a qualidade necessária 
para análises subsequentes. A amostra W apresentou resultado positivo na eletroforese em gel, 
indicando a presença do agente etiológico investigado. 

Esses achados reforçam o potencial do método como uma ferramenta confiável para o diagnóstico 
molecular da leishmaniose, com possível aplicação em cenários clínicos, epidemiológicos e ambientais.