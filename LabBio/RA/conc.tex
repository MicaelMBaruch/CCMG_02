\section{Conclusão}

A reprodução dos protocolos de \textit{salting out}, eletroforese com RFLP e
qPCR acoplada à HRM foram bem sucedidas. Ambos os protocolos de diagnóstico
adotados demonstraram-se eficazes na detecção sensível e específica do DNA de
interesse. Em particular, as análises de eletroforese em gel com RFLP indicam a
presença apenas de \textit{L. (L.) amazonensis} na amostra W. 

A análise por HRM se mostra uma alternativa robusta e sensível para a
diferenciação de espécies de \textit{Leishmania}. A correlação das curvas de
\textit{melting} das amostras desconhecidas com as espécies conhecidas indica a
viabilidade do método para diagnosticar a espécie em uma amostra com apenas um
experimento. Em decorrência do princípio de analisar \textit{melting
temperature} por trás do método, possibilita que este seja expandido para mais
espécies contanto que haja pequenas variações de sequência (como polimorfismos
de nucleotídeo único, SNPs) em sequência conservada entre os organismos de
interesse que gere alterações nos perfis de \textit{melting} para um mesmo
conjunto de primers. 

