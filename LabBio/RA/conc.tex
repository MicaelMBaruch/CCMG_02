\section{Conclusão}

O protocolo adotado demonstrou alta eficácia na detecção do DNA de interesse diagnóstico, 
evidenciando sua aplicabilidade em contextos laboratoriais. A metodologia empregada possibilitou 
a extração e amplificação bem-sucedidas do material genético, garantindo a qualidade necessária 
para análises subsequentes. A amostra W apresentou resultado positivo na eletroforese em gel, 
indicando a presença do agente etiológico investigado. 

Esses achados reforçam o potencial do método como uma ferramenta confiável para o diagnóstico 
molecular da leishmaniose, com possível aplicação em cenários clínicos, epidemiológicos e ambientais.

No contexto de diagnóstico e tipagem molecular, a análise por HRM se mostra uma
alternativa robusta e sensível para a diferenciação de espécies de
\textit{Leishmania}. O
método é particularmente útil quando pequenas variações de sequência (como
polimorfismos de nucleotídeo único, SNPs) impactam diretamente a Tm dos
amplicons, sendo capazes de gerar perfis de melting 
discriminativos.

Assim, as análises feitas indicam que o protocolo utilizado é capaz de
identificar eficientemente a presença do DNA-alvo e distinguir entre as espécies
de \textit{Leishmania}. 
