\section{Conclusão}

No contexto de diagnóstico e tipagem molecular, conforme discutido por Naze et
al. (2015), a análise por HRM se mostra uma alternativa robusta e sensível para
a diferenciação de cepas ou espécies
O método é particularmente útil quando
pequenas variações de sequência (como polimorfismos de nucleotídeo único, SNPs)
impactam diretamente a Tm dos amplicons, sendo capazes de gerar perfis de
melting reprodutíveis e discriminativos.

Assim, as análises feitas indicam que o protocolo utilizado é capaz de
identificar eficientemente a presença do DNA-alvo e distinguir entre as espécies
de \textit{Leishmania}.  
