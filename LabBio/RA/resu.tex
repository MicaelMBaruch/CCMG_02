\begin{abstract}

    A leishmaniose é uma doença infecciosa causada por protozoários do gênero
    \textit{Leishmania}, transmitidos pela picada de flebotomíneos. Suas
    manifestações clínicas variam de formas cutâneas autolimitadas a formas
    viscerais graves e potencialmente fatais. O diagnóstico precoce e preciso é
    fundamental para o manejo clínico eficaz e o controle epidemiológico da
    doença. Dentre os métodos diagnósticos disponíveis, as técnicas moleculares
    têm se destacado pela elevada sensibilidade e especificidade, especialmente
    aquelas baseadas na extração e análise do DNA do parasita. Neste estudo,
    aplicaram-se métodos moleculares para verificar a presença de
    \textit{Leishmania} em uma amostra-problema (W) e, em caso positivo,
    determinar a espécie envolvida. O DNA extraído apresentou alta pureza
    (A\textsubscript{260}/A\textsubscript{280} = 2{,}07;
    A\textsubscript{260}/A\textsubscript{230} = \num{2.31}) e
    concentração adequada (\qty{140.1}{\nano\gram\per\micro\liter}), sendo
    amplificado com sucesso por PCR. A eletroforese em gel de agarose revelou um
    fragmento compatível com o tamanho esperado, indicando a presença do
    patógeno. Adicionalmente, a análise por qPCR com \textit{High Resolution
    Melting} (HRM) corroborou os resultados obtidos, permitindo a diferenciação
    entre espécies do complexo \textit{Leishmania spp.}. A amostra W apresentou
    perfil de fusão compatível com \textit{Leishmania amazonensis}, confirmando
    a eficácia e robustez da abordagem molecular empregada para o diagnóstico
    específico da leishmaniose. 

\end{abstract}
