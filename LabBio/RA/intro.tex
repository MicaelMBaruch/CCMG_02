\section{Introdução}

Leishmanioses são doenças infecciosas causadas por protozoários do gênero
\textit{Leishmania}, transmitidos por flebotomíneos e endêmicos em 98 países,
afetando mais de 350 milhões de pessoas \cite{hong2020one}. Essas doenças
apresentam manifestações clínicas diversas — de lesões cutâneas autolimitadas à
leishmaniose visceral fatal — que variam conforme a espécie envolvida. No
entanto, a diversidade de espécies causadoras de leishmanioses e as semelhanças
nos sintomas apresentados entre elas dificulta o adequado tratamento
espécies-específico da doença.

A complexidade da epidemiologia da leishmaniose demanda ferramentas diagnósticas
precisas, rápidas e economicamente viáveis.  Neste cenário, métodos moleculares
podem ser a chave na identificação da espécie de \textit{Leishmania} envolvida
na infecção. O presente trabalho teve como objetivo reproduzir de forma adaptada
os métodos moleculares de Graça
et.al.\cite{RFLPgraca2012} e Zampieri et.al.\cite{HRMzampi2016} com objetivo de
diagnosticar dentre as espécies \textit{Leishmania (Leishmania) infantum},
\textit{Leishmania (Leishmania) amazonensis}, \textit{Leishmania (Viannia)
braziliensis} e \textit{Leishmania (Viannia)
shawi} em uma amostra de DNA desconhecida. 

\subsection{Espectrofotometria, Eletroforese e PCR-RFLP}

A espectrofotometria por NanoDrop é uma técnica amplamente utilizada para avaliar a pureza e a 
concentração de ácidos nucleicos extraídos. A quantificação é realizada por meio da absorbância da 
amostra em diferentes comprimentos de onda, especialmente em \SI{260}{\nano\meter} para DNA e RNA, 
e em \SI{280}{\nano\meter} para proteínas. As razões A\textsubscript{260}/A\textsubscript{280} e A\textsubscript{260}/A\textsubscript{230} permitem 
inferir o grau de pureza do material, sendo valores entre 1{,}8 e 2{,}0 indicativos de DNA livre de proteínas 
e valores acima de 2{,}0 livre de outros contaminantes orgânicos~\cite{nanodrop}. Além disso, a pequena quantidade de amostra requerida (1–2~\si{\micro\liter}) 
torna o NanoDrop uma ferramenta rápida e eficiente para controle de qualidade de extrações de DNA antes de análises subsequentes.
% A métrica dada pelas professoras foram essas, porém nos artigos só encontro que A260/A280 ~ 1,8-2,0 e A260/A260 ~ 2,0-2,2

A eletroforese em gel de agarose é um método essencial para a visualização e separação de fragmentos de DNA com base em seu tamanho. 
Após a aplicação das amostras em poços de um gel de agarose e aplicação de corrente elétrica, os fragmentos de DNA migram em direção 
ao polo positivo. A velocidade de migração depende do tamanho dos fragmentos — quanto menor o fragmento, maior sua mobilidade. 
A adição de agentes intercalantes, como brometo de etídio ou SYBR Safe, permite a visualização das bandas de DNA sob luz ultravioleta. 
Esta técnica permite confirmar o sucesso de amplificações por PCR, avaliar a integridade do DNA e estimar o tamanho dos amplicons 
com base em marcadores de peso molecular~\cite{costa1998leishmaniose}.

A PCR seguida de análise por polimorfismo de fragmento de restrição (PCR-RFLP) é uma técnica molecular que combina amplificação de regiões 
específicas do DNA com digestão enzimática, visando identificar variações na sequência por meio do padrão de fragmentos gerados. Inicialmente, 
uma região alvo do genoma é amplificada por PCR. Em seguida, a amostra é incubada com enzimas de restrição que reconhecem sequências específicas 
de nucleotídeos. Alterações pontuais no DNA — como SNPs — podem criar ou abolir locais de corte, resultando em perfis distintos de bandas após separação 
por eletroforese em gel. A PCR-RFLP é especialmente útil para diferenciação de espécies, genotipagem e identificação de variantes alélicas, sendo 
considerada uma técnica de baixo custo, reprodutível e aplicável em diferentes contextos epidemiológicos e diagnósticos~\cite{garcia2005metodos}.

\subsection{qPCR e HRM}
A PCR em tempo real (qPCR)  é um método molecular baseado na amplificação da
quantidade de fitas de DNA por ciclos, enquanto a fluorescência emitida por
corantes intercalantes, é monitorada em tempo real a cada ciclo de amplificação.
Como o aumento da fluorescência é proporcional à quantidade de DNA amplificado,
torna-se possível quantificar o material genético na amostra\cite{Galluzi2018}.
A qPCR associada a análise \textit{High Resolution Melting} (HRM) implica que
após a amplificação, é realizada uma análise de dissociação da dupla fita com
relação à temperatura (\textit{melting curve}). Para tanto, a temperatura é
elevada gradualmente, à um passo definido e a fluorescência é medida
continuamente. À medida que as duplas fitas de DNA se dissociam em fitas
simples, a fluorescência diminui bruscamente, gerando uma curva característica
de cada amplicon. Isto permite a detecção de variações sutis na sequência de DNA
— como polimorfismos de nucleotídeo único (SNPs), inserções, deleções ou
diferenças no conteúdo de GC — por meio de alterações no formato da curva e na
temperatura de melting (Tm)\cite{Wittwer2009} por consequência da natureza
termoquímica destas variações.

No contexto da leishmaniose, a qPCR-HRM tem grande potencial
para o diagnóstico molecular e a discriminação de espécies de
\textit{Leishmania}. Para tanto, a elaboração de primers específicos para uma
sequência conservada entre as espécies, porém com pequenas variações em alguns
nucleotídeos permite obter perfis distintos de Tm e
formas de curva de dissociação para cada espécie.
Estudos anteriores foram bem-sucedidos em diferenciar espécies com base em
diferenças de menos de \qty{1}{\celsius}, na Tm dos amplicons
avaliados\cite{Azam2024,HRMzampi2016,SaadiBenAoun2024}.
Essa abordagem permite diagnósticos rápidos, sem necessidade de sequenciamento,
e pode ser integrada a plataformas de PCR em tempo real já existentes,
representando uma alternativa sensível, específica e custo-efetiva para a
identificação de espécies de \textit{Leishmania}.
 
