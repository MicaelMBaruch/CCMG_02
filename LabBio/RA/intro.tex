\section{Introdução}

Leishmanioses são doenças infecciosas causadas por protozoários do gênero
\textit{Leishmania}, transmitidos por flebotomíneos e endêmicos em 98 países,
afetando mais de 350 milhões de pessoas\cite{hong2020one}. Essas doenças
apresentam manifestações clínicas diversas — de lesões cutâneas autolimitadas à
leishmaniose visceral fatal — que variam conforme a espécie envolvida. No
entanto, a diversidade de espécies causadoras de leishmanioses e as semelhanças
nos sintomas apresentados entre elas dificulta o adequado tratamento
espécie-específico da doença.

A complexidade da epidemiologia da leishmaniose demanda ferramentas diagnósticas
precisas, rápidas e economicamente viáveis.  Neste cenário, métodos moleculares
podem ser a chave na identificação da espécie de \textit{Leishmania} envolvida
na infecção. Duas metodologias de diagnóstico propostas recentemente são a de
Graça et.al.\cite{RFLPgraca2012} e a de Zampieri et.al.\cite{HRMzampi2016}.  A

metodologia de Graça e colegas é baseada em Reação em Cadeia da Polimerase (PCR,
do inglês \textit{Polymerase Chain Reaction}) seguida de análise por
polimorfismo de fragmento de restrição (PCR-RFLP, do inglês \textit{Polymerase
Chain Reaction - Restriction Fragment Length Polymorphism}). Conquanto, a
metodologia de Zampieri e colegas é baseada na PCR em tempo real (qPCR, do
inglês \textit{quantitative Polymerase Chain Reaction}) acoplada
a análise de Alta de Resolução de Dissociação (HRM, do inglês \textit{High
Resolution Melting}) do DNA. 

A PCR-RFLP é uma técnica molecular que combina amplificação de regiões
específicas do DNA com digestão enzimática, visando identificar variações na
sequência por meio do padrão de fragmentos gerados. Inicialmente, uma região
alvo do genoma é amplificada por PCR. Em seguida, a amostra é incubada com
enzimas de restrição que reconhecem sequências específicas de nucleotídeos.
Alterações pontuais no DNA — como polimorfismos de nucleotídeo único (SNPs) —
podem criar ou abolir locais de corte, resultando em perfis distintos de bandas
após separação por eletroforese em gel. A PCR-RFLP é especialmente útil para
diferenciação de espécies, genotipagem e identificação de variantes alélicas,
sendo considerada uma técnica de baixo custo, reprodutível e aplicável em
diferentes contextos epidemiológicos e diagnósticos\cite{garcia2005metodos}.

A eletroforese em gel de agarose é um método essencial para a visualização e
separação de fragmentos de DNA com base em seu tamanho.  Após a aplicação das
amostras em poços de um gel de agarose e aplicação de corrente elétrica, os
fragmentos de DNA migram em direção ao polo positivo. A velocidade de migração
depende do tamanho dos fragmentos — quanto menor o fragmento, maior sua
mobilidade.  A adição de agentes intercalantes, como brometo de etídio ou SYBR
Safe, permite a visualização das bandas de DNA sob luz ultravioleta.  Esta
técnica permite confirmar o sucesso de amplificações por PCR, avaliar a
integridade do DNA e estimar o tamanho dos amplicons com base em marcadores de
peso molecular\cite{costa1998leishmaniose}.

A espectrofotometria por NanoDrop é uma técnica amplamente utilizada para
avaliar a pureza e a concentração de ácidos nucleicos extraídos. A quantificação
é realizada por meio da absorbância da amostra em diferentes comprimentos de
onda, especialmente em \qty{260}{\nano\meter} para DNA e RNA, e em
\qty{280}{\nano\meter} para proteínas. As razões
A\textsubscript{260}/A\textsubscript{280} e
A\textsubscript{260}/A\textsubscript{230} permitem inferir o grau de pureza do
material, sendo valores entre \num{1.8} e \num{2.0} indicativos de DNA livre de
proteínas e valores acima de \num{2.0} livre de outros contaminantes
orgânicos\cite{nanodrop}. Além disso, a pequena quantidade de amostra requerida
(\qtyrange{1}{2}{\micro\liter}) torna o NanoDrop uma ferramenta rápida e
eficiente para controle de qualidade de extrações de DNA antes de análises
subsequentes.
%métrica dada pelas professoras foram essas, porém nos artigos só encontro que
%A260/A280 ~ 1,8-2,0 e A260/A260 ~ 2,0-2,2

A qPCR é um método molecular baseado na amplificação da quantidade de fitas de
DNA por ciclos, enquanto a fluorescência emitida por corantes intercalantes, é
monitorada em tempo real a cada ciclo de amplificação.  Como o aumento da
fluorescência é proporcional à quantidade de DNA amplificado, torna-se possível
quantificar o material genético na amostra\cite{Galluzi2018}.  Para a qPCR
associada a análise HRM, após a amplificação, é realizada uma análise de
dissociação da dupla fita com relação à temperatura (\textit{melting curve}).
Para tanto, a temperatura é elevada gradualmente, à um passo definido e a
fluorescência é medida continuamente. À medida que as duplas fitas de DNA se
dissociam em fitas simples, a fluorescência diminui bruscamente, gerando uma
curva característica de cada amplicon. Isto permite a detecção de variações
sutis na sequência de DNA — como SNPs, inserções, deleções ou diferenças no
conteúdo de GC — por meio de alterações no formato da curva e na temperatura de
melting (Tm)\cite{Wittwer2009} por consequência da natureza termoquímica destas
variações. Dessa forma, a identificação das espécies presentes em amostra de uma
metodologia de diagnóstico por qPCR acoplada à HRM depende diretamente das
diferenças nas sequências dos amplicons gerados pelo conjunto de primers
desenhados. 

O presente trabalho tem como objetivo reproduzir de forma adaptada os métodos
moleculares de Graça et.al.\cite{RFLPgraca2012} e Zampieri
et.al.\cite{HRMzampi2016} com objetivo de diagnosticar dentre as espécies
\textit{Leishmania (Leishmania) infantum}, \textit{Leishmania (Leishmania)
amazonensis}, \textit{Leishmania (Viannia) braziliensis} e \textit{Leishmania
(Viannia) shawi} em uma amostra de DNA desconhecida. 
