\section{Introdução}

Leishmanioses são doenças infecciosas causadas por protozoários do gênero
\textit{Leishmania}, transmitidos por flebotomíneos e endêmicos em 98 países,
afetando mais de 350 milhões de pessoas \cite{hong2020one}. Essas doenças
apresentam manifestações clínicas diversas — de lesões cutâneas autolimitadas à
leishmaniose visceral fatal — que variam conforme a espécie envolvida. No
entanto, a diversidade de espécies causadoras de leishmanioses e as semelhanças
nos sintomas apresentados entre elas dificulta o adequado tratamento
espécies-específico da doença.

A complexidade da epidemiologia da leishmaniose demanda ferramentas diagnósticas
precisas, rápidas e economicamente viáveis.  Neste cenário, métodos moleculares
podem ser a chave na identificação da espécie de \textit{Leishmania} envolvida
na infecção. O presente trabalho teve como objetivo reproduzir de forma adaptada
os métodos moleculares de Graça
et.al.\cite{RFLPgraca2012} e Zampieri et.al.\cite{HRMzampi2016} com objetivo de
diagnosticar dentre as espécies \textit{Leishmania (Leishmania) infantum},
\textit{Leishmania (Leishmania) amazonensis}, \textit{Leishmania (Viannia)
braziliensis} e \textit{Leishmania (Viannia)
shawi} em uma amostra de DNA desconhecida. 

O HRM ...

