\section{Introdução}

\subsection{Espectrofotometria, Eletroforese e PCR-RFLP}

A espectrofotometria por NanoDrop é uma técnica amplamente utilizada para avaliar a pureza e a 
concentração de ácidos nucleicos extraídos. A quantificação é realizada por meio da absorbância da 
amostra em diferentes comprimentos de onda, especialmente em \SI{260}{\nano\meter} para DNA e RNA, 
e em \SI{280}{\nano\meter} para proteínas. A razão A\textsubscript{260}/A\textsubscript{280} permite 
inferir o grau de pureza do material, sendo valores próximos de 1{,}8 indicativos de DNA livre de proteínas 
e outros contaminantes~\cite{Alguém}. Além disso, a pequena quantidade de amostra requerida (1–2~\si{\micro\liter}) 
torna o NanoDrop uma ferramenta rápida e eficiente para controle de qualidade de extrações de DNA antes de análises subsequentes.

A eletroforese em gel de agarose é um método essencial para a visualização e separação de fragmentos de DNA com base em seu tamanho. 
Após a aplicação das amostras em poços de um gel de agarose e aplicação de corrente elétrica, os fragmentos de DNA migram em direção 
ao polo positivo. A velocidade de migração depende do tamanho dos fragmentos — quanto menor o fragmento, maior sua mobilidade. 
A adição de agentes intercalantes, como brometo de etídio ou SYBR Safe, permite a visualização das bandas de DNA sob luz ultravioleta. 
Esta técnica permite confirmar o sucesso de amplificações por PCR, avaliar a integridade do DNA e estimar o tamanho dos amplicons 
com base em marcadores de peso molecular~\cite{Alguém}.

A PCR seguida de análise por polimorfismo de fragmento de restrição (PCR-RFLP) é uma técnica molecular que combina amplificação de regiões 
específicas do DNA com digestão enzimática, visando identificar variações na sequência por meio do padrão de fragmentos gerados. Inicialmente, 
uma região alvo do genoma é amplificada por PCR. Em seguida, a amostra é incubada com enzimas de restrição que reconhecem sequências específicas 
de nucleotídeos. Alterações pontuais no DNA — como SNPs — podem criar ou abolir locais de corte, resultando em perfis distintos de bandas após separação 
por eletroforese em gel. A PCR-RFLP é especialmente útil para diferenciação de espécies, genotipagem e identificação de variantes alélicas, sendo 
considerada uma técnica de baixo custo, reprodutível e aplicável em diferentes contextos epidemiológicos e diagnósticos.
