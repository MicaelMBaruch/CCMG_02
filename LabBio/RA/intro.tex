\section{Introdução}

Leishmanioses são doenças infecciosas causadas por protozoários do gênero
\textit{Leishmania}, transmitidos por flebotomíneos e endêmicos em 98 países,
afetando mais de 350 milhões de pessoas \cite{hong2020one}. Essas doenças
apresentam manifestações clínicas diversas — de lesões cutâneas autolimitadas à
leishmaniose visceral fatal — que variam conforme a espécie envolvida. No
entanto, a diversidade de espécies causadoras de leishmanioses e as semelhanças
nos sintomas apresentados entre elas dificulta o adequado tratamento
espécies-específico da doença.

A complexidade da epidemiologia da leishmaniose demanda ferramentas diagnósticas
precisas, rápidas e economicamente viáveis.  Neste cenário, métodos moleculares
podem ser a chave na identificação da espécie de \textit{Leishmania} envolvida
na infecção. O presente trabalho teve como objetivo reproduzir de forma adaptada
os métodos moleculares de Graça
et.al.\cite{RFLPgraca2012} e Zampieri et.al.\cite{HRMzampi2016} com objetivo de
diagnosticar dentre as espécies \textit{Leishmania (Leishmania) infantum},
\textit{Leishmania (Leishmania) amazonensis}, \textit{Leishmania (Viannia)
braziliensis} e \textit{Leishmania (Viannia)
shawi} em uma amostra de DNA desconhecida. 

A PCR em tempo real (qPCR)  é um método molecular baseado na amplificação da
quantidade de fitas de DNA por ciclos, enquanto a fluorescência emitida por
corantes intercalantes, é monitorada em tempo real a cada ciclo de amplificação.
Como o aumento da fluorescência é proporcional à quantidade de DNA amplificado,
torna-se possível quantificar o material genético na amostra\cite{Galluzi2018}.
A qPCR associada a análise \textit{High Resolution Melting} (HRM) implica que
após a amplificação, é realizada uma análise de dissociação da dupla fita com
relação à temperatura (\textit{melting curve}). Para tanto, a temperatura é
elevada gradualmente, à um passo definido e a fluorescência é medida
continuamente. À medida que as duplas fitas de DNA se dissociam em fitas
simples, a fluorescência diminui bruscamente, gerando uma curva característica
de cada amplicon. Isto permite a detecção de variações sutis na sequência de DNA
— como polimorfismos de nucleotídeo único (SNPs), inserções, deleções ou
diferenças no conteúdo de GC — por meio de alterações no formato da curva e na
temperatura de melting (Tm)\cite{Wittwer2009} por consequência da natureza
termoquímica destas variações.

No contexto da leishmaniose, a qPCR-HRM tem grande potencial
para o diagnóstico molecular e a discriminação de espécies de
\textit{Leishmania}. Para tanto, a elaboração de primers específicos para uma
sequência conservada entre as espécies, porém com pequenas variações em alguns
nucleotídeos permite obter perfis distintos de Tm e
formas de curva de dissociação para cada espécie.
Estudos anteriores foram bem-sucedidos em diferenciar espécies com base em
diferenças de menos de \qty{1}{\celsius}, na Tm dos amplicons
avaliados\cite{Azam2024,HRMzampi2016,SaadiBenAoun2024}.
Essa abordagem permite diagnósticos rápidos, sem necessidade de sequenciamento,
e pode ser integrada a plataformas de PCR em tempo real já existentes,
representando uma alternativa sensível, específica e custo-efetiva para a
identificação de espécies de \textit{Leishmania}.
 

