\section{Resultados}
% Eu sei precisa colocar a imagem ajustar a imagem, mas calma, não sufoca o artista

\subsection{Quantificação e pureza do DNA}
A quantificação do DNA genômico foi realizada por espectrofotometria utilizando equipamento NanoDrop, 
com análise das razões de absorbância A\textsubscript{260}/A\textsubscript{280} e A\textsubscript{260}/A\textsubscript{230}. 
A amostra analisada apresentou concentração de 140{,}1~ng/µL, com razão A\textsubscript{260}/A\textsubscript{280} de 2{,}07 e 
A\textsubscript{260}/A\textsubscript{230} de 2{,}31. Esses valores indicam alta pureza, com ausência significativa de contaminantes 
como proteínas (razão esperada entre 1{,}8 e 2{,}0) ou compostos orgânicos/sais (razão A\textsubscript{260}/A\textsubscript{230} 
ideal > 1{,}8)\cite{Alguém}. A qualidade do DNA obtido foi considerada adequada para as reações de PCR subsequentes.

\subsection{Análise por PCR convencional}
A identificação molecular da amostra-problema foi conduzida por meio de três reações de PCR convencionais, designadas como PCR-A, PCR-B 
e PCR-C, cada uma com especificidade crescente em relação ao táxon de interesse. Os produtos amplificados foram analisados por eletroforese 
em gel de agarose a 1{,}5\%, conforme apresentado nas Figuras~\ref{pcrA} e~\ref{pcrBC}.

\begin{wrapfigure}{L}{.45\textwidth}
 \centering
 \includegraphics[width=.4\textwidth]{fig/PCR_A.jpg}
 \caption{Eletroforese dos produtos da PCR-A (gênero específico). As extremidades do gel contém escadas de peso molecular (ladders). 
 Os primeiros cinco poços da esquerda correspondem às amostras de DNA testadas quanto à presença do gênero \textit{Leishmania}, 
 seguidos por um controle negativo (água). Os poços subsequentes contêm, respectivamente: (1) \textit{L. (L.) infantum}, 
 (2) \textit{L. (L.) amazonensis}, (3) \textit{L. (V.) braziliensis}, (4) \textit{L. (V.) shawi}, (5) amostra-problema (W) e (6) controle negativo.}
 \label{pcrA}
 \end{wrapfigure}

A Figura~\ref{PCR_A} apresenta os resultados da eletroforese em gel de agarose referente à PCR-A, que tem como alvo uma região conservada do DNA genômico do gênero
 \textit{Leishmania}. Esta reação teve como objetivo inicial confirmar se a amostra-problema pertencia ao gênero, além de compará-la com amostras de referência previamente 
 caracterizadas.. Observa-se que tanto a amostra-problema (poço 5) quanto a amostra-controle de \textit{L. (L.) amazonensis} (poço 2) apresentaram bandas com o mesmo tamanho 
 molecular e intensidade semelhante. Esse padrão sugere fortemente que a amostra W pertence à mesma espécie ou, no mínimo, ao mesmo subgênero da amostra-controle. Por outro 
 lado, as demais amostras  exibiram perfis distintos, e os controles negativos (poços 6 e 12) não apresentaram amplificação, confirmando a ausência de contaminações cruzadas.
Esses dados sustentam a hipótese de que a amostra-problema pertence ao gênero \textit{Leishmania}, com forte indício de pertencimento à espécie \textit{L. (L.) amazonensis}, 
conforme indicado pela coincidência do perfil eletroforético com a amostra-controle correspondente.

\begin{wrapfigure}{R}{.45\textwidth}
 \centering
 \includegraphics[width=.4\textwidth]{fig/PCR_BC.jpg}
 \caption{Eletroforese dos produtos da PCR-B (subgênero \textit{Viannia}) e PCR-C (espécies não-\textit{braziliensis} do subgênero \textit{Viannia}). 
 A primeira série de seis poços à esquerda refere-se à PCR-B; em seguida há um poço com escada de peso molecular (ladder), seguido por seis poços referentes 
 à PCR-C. A ordem das amostras em ambas as reações é: (1) \textit{L. (L.) infantum}, (2) \textit{L. (L.) amazonensis}, (3) \textit{L. (V.) braziliensis}, 
 (4) \textit{L. (V.) shawi}, (5) amostra-problema (W) e (6) controle negativo.}
 \label{pcrBC}
 \end{wrapfigure}

As reações subsequentes, PCR-B e PCR-C, foram desenhadas com primers específicos para o subgênero \textit{Leishmania (Viannia)} e para espécies não-\textit{brasiliensis} dentro 
deste subgênero, respectivamente. Em ambas as reações, a amostra W não apresentou bandas detectáveis, ao contrário dos controles positivos específicos para \textit{L. (Viannia) 
braziliensis} e outras espécies do subgênero. A ausência de amplificação nas PCRs B e C reforça que a amostra W não pertence ao subgênero \textit{Viannia}, corroborando a classificação 
como \textit{Leishmania (Leishmania) amazonensis} já inferida pela PCR-A.

Na PCR-B, específica para o subgênero não-\textit{Leishmania (Viannia)}, foi observada uma banda nítida no poço 4, correspondente à amostra \textit{L. (V.) shawi}, como esperado. 
No entanto, uma banda mais tênue e não prevista foi detectada no poço 3, referente a \textit{L. (V.) braziliensis}, o que pode indicar uma amplificação inespecífica ou contaminação 
pontual. As demais amostras, incluindo a amostra-problema (poço 5), não apresentaram bandas visíveis, sugerindo ausência de amplificação com os primers utilizados nesta reação.

Já na PCR-C, voltada à detecção de espécies \textit{braziliensis} dentro do subgênero \textit{Viannia}, foi identificada uma banda clara no poço 3 (L. (V.) braziliensis}), 
como previsto. No entanto, uma amplificação não esperada foi observada no poço 1, correspondente a \textit{L. (L.) infantum}, o que sugere a possibilidade de reação cruzada ou mínima 
contaminação da amostra.
