\section{Conclusão} 
Este relatório explorou a aplicação de princípios físicos ondulatórios em quatro experimentos distintos, enriquecendo a compreensão de fenômenos sonoros. O experimento com o diapasão e a coluna d'água demonstrou a ressonância em tubos fechados, permitindo estimar a velocidade do som no ar em \qty{307,2}{m/s}. A discrepância de aproximadamente 10,4\% em relação ao valor teórico de \qty{343}{m/s} foi atribuída a variações ambientais, imprecisões e possíveis erros sistemáticos.

As figuras de Chladni ofereceram uma visualização das ondas estacionárias em superfícies bidimensionais, onde a areia se acumulava nos pontos nodais. A variação dos padrões com a mudança dos pontos de apoio e excitação ressaltou a influência das condições de contorno e da geometria na formação dos modos de vibração.

O tubo de Rijke ilustrou a geração de som a partir de energia térmica, com o fenômeno dependendo da convecção do ar aquecido. A interrupção do som ao inclinar o tubo evidenciou a importância da orientação vertical para a manutenção do fluxo periódico de ar.

Por fim, o experimento da corda vibrante, similar ao monocórdio de Pitágoras, validou as leis de Mersenne ao demonstrar como a frequência e o tom do som são influenciados pela tensão, comprimento e densidade linear da corda. A capacidade de variar esses parâmetros é crucial para a afinação e a diversidade tonal em instrumentos de corda. Em suma, os experimentos reforçaram conceitos fundamentais da física ondulatória e suas aplicações práticas.

