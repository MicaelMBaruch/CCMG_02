\section{Conclusão}

A realização dos experimentos propostos permitiu uma aplicação prática e aprofundada dos conceitos fundamentais da hidrostática, especialmente do princípio de Arquimedes. Ao longo das atividades, foi possível observar como a força de empuxo atua sobre objetos imersos em fluidos, sendo proporcional ao volume de fluido deslocado, e como essa força interfere no equilíbrio e na flutuação de diferentes corpos.

No experimento do princípio de Arquimedes, observamos de forma clara a relação entre o empuxo e a densidade do material analisado, possibilitando a identificação indireta da composição do corpo submerso. A balança de empuxo, por sua vez, demonstrou a viabilidade de medir massas de objetos por meio da variação do nível de água, evidenciando uma aplicação criativa e eficiente do princípio físico.

A atividade com a gangorra permitiu visualizar o equilíbrio de forças e torques em sistemas parcialmente imersos, destacando o papel do empuxo como uma força de sustentação que influencia diretamente a estabilidade do sistema. Já os bonecos flutuantes ilustraram de forma didática a aplicação do Princípio de Pascal e o comportamento de corpos em função da variação de pressão no interior de um líquido, reforçando conceitos de compressibilidade dos gases e deslocamento de volume.

Assim, os experimentos não apenas consolidaram o conhecimento teórico sobre hidrostática, empuxo e equilíbrio de forças, como também estimularam a capacidade de análise crítica e experimental. A integração entre teoria e prática demonstrou a importância desses conceitos na compreensão de fenômenos cotidianos e no desenvolvimento de tecnologias que envolvem fluidos e flutuação.



