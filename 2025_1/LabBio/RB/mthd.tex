\section{Metodologia}
\subsection{Transformação genética de \textit{Escherichia coli} com proteínas
fluorescentes}
A transformação genética de \textit{Escherichia coli} foi realizada três vezes,
cada uma com uma solução diferentes de DNA plasmidial: GFP, mCherry e
mTAGbfp2. Inicialmente, foram adicionados \qty{2}{\micro\liter} de DNA do
plasmídeo ao tubo contendo às células competentes, previamente tratadas. Após
agitação delicada, a mistura foi transferida para um tubo de fundo cônico e
identificado. A mistura foi levada para um banho de gelo e mantida por 30
minutos sem agitação. Depois, foi levada para um banho a 42°C por 90 segundos
cronometrados, ainda com cuidado para não agitar. Em seguida, a mistura voltou
ao banho de gelo por 1 a 2 minutos, sem agitação. Foram adicionados
\qty{800}{\micro\liter} de meio SOC à mistura e esta foi levada ao banho maria,
agora a 37°C. Após realizar este preparo com todas as soluções de DNA
plasmidial, as misturas foram levadas ao \textit{shaker} a 37°C sob agitação de
250 RPM. Em seguida, foram separadas três placas de seleção com antibiótico, em
cada uma foram espalhadas \qty{200}{\micro\liter} da mistura. Cada placa foi
identificada e deixada para incubar \textit{overnight} a 37°C. No dia seguinte, foi
verificado o crescimento das bactérias em cada placa, retirado as colônias e
transferido para lâminas de microscópio, cobrindo com uma lamínula. Por fim, as
lâminas foram levadas ao microscópio de fluorescência nas faixas de luz
apropriadas, onde foram retiradas as imagens.

\subsection{Transformação genética de plantas por meio de Agroinfiltração (\textit{Agrobacterium tumefaciens})}
As linhagens de \textit{Agrobacterium} (EHA 105) transformadas com um vetor
binário (2 RNAs guia com alvo para TNT1) foram inoculadas em placas de Petri
contendo um meio YEB suplementado com os antibióticos apropriados e incubadas a
\qty{28}{\celsius} por 48 horas. Então, foram transferidas duas a três colônias
isoladas da bacteria, com um palito de dente autoclavado, para tubos Falcon de
\qty{15}{mL} com \qty{3}{mL} de meio LB acrescido de estreptomicina e
espectinomicina, seguindo incubação sob agitação a \qty{28}{\celsius} por 12 a
16 horas. Em seguida, alíquotas de \qty{30}{\micro L} de cada cultura, foram
repicadas para tubos Falcon contendo \qty{10}{mL} de LB com antibióticos
apropriados, e novamente, foram incubados nas mesmas condições anteriores. Ao fim
da incubação, centrifugou-se os tubos com as culturas a \qty{4000}{RPM}
(\qty{3200}{g}) por \qty{20}{min}, então o pellet foi ressuspenso em
\qty{10}{mL} de tampão de infiltração (acetoseringona \qty{150}{\micro M},
\ce{MgCl2} \qty{10}{mM}, MES \qty{10}{mM} a um pH de 5,5) e diluiu-se \qty{1}{mL}
da suspensão até que a medida da absorbância a 600 nm ($A_{600}$) em um
espectrofotômetro UV-Vis (UVmini-1240, SHIMADZU) atingisse
$A_{600} = 0,5$.
A agroinfiltração da suspensão das bactérias foi realizada na superfície abaxial
de folhas de \textit{Nicotiana benthamiana}, \textit{Nicotiana tabacum} e
\textit{Solanum lycopersicum} utilizando uma seringa estéril, com as plantas
mantidas em casa de vegetação por 2 dias. Após esse período, as folhas foram
coletadas, seccionada transversalmente com lâminas de barbear e os cortes mais
finos foram montados em lâminas com água destilada. Essas lâminas foram
analisadas em um microscópio de fluorescência sob o filtro GFP, e as melhores
imagens dos resultados foram coletadas e analisadas por meio do programa Fiji.  
