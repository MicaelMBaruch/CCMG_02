\begin{abstract}
   O desenvolvimento de técnicas de biologia molecular e engenharia genética tem
   revolucionado a biologia celular moderna, com ferramentas como a edição
   genética e a visualização molecular \textit{in vivo} por
   meio de proteínas fluorescentes. Este relatório teve como objetivo aplicar
   metodologias de biologia molecular para expressar proteínas fluorescentes em
   sistemas procariontes (\textit{Escherichia coli}) e vegetais (\textit{Nicotiana
   benthamiana}, \textit{Nicotiana tabacum} e \textit{Solanum lycopersicum}), visando avaliar a
   eficiência da transformação genética e da agroinfiltração.
   Para a transformação de \textit{E. coli}, foram utilizados plasmídeos contendo genes
   para GFP, mCherry e BFP (mTAGbfp2), seguindo um protocolo padrão de transformação
   bacteriana, incluindo etapas de choque térmico e seleção em placas com
   antibióticos. A expressão das proteínas foi analisada por microscopia de
   fluorescência. No caso das plantas, a agroinfiltração foi realizada com
   \textit{Agrobacterium tumefaciens} portando um plasmídeo binário, contendo GFP com localização nuclear
   fusionado a Cas9, seguida de incubação e análise
   microscópica das folhas infiltradas.
   Os resultados demonstraram sucesso na transformação de \textit{E. coli} com os
   plasmídeos de mCherry e mTAGbfp2, evidenciado pela fluorescência vermelha e
   azul, respectivamente, enquanto a transformação com GFP falhou devido à baixa
   concentração do plasmídeo. Na agroinfiltração, apenas \textit{Solanum lycopersicum}
   apresentou fluorescência pontual, enquanto \textit{Nicotiana benthamiana} e
   \textit{Nicotiana
   tabacum} não exibiram expressão detectável, possivelmente devido a fatores
   como tempo insuficiente de incubação ou resistência à infecção.
   Conclui-se que a transformação de \textit{E. coli} com mCherry e mTAGbfp2 foi eficaz,
   enquanto a agroinfiltração apresentou resultados variáveis entre as espécies
   vegetais, destacando a importância de otimizar condições experimentais para
   a expressão transgênica em plantas. 
\end{abstract}
