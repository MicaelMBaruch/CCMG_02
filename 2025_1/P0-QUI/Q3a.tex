\begin{xcs}
    A \qty{273}{K}, o argônio tem os seguintes coeficientes do virial: 
    B = \qty{-21,7}{cm^3.mol^{-1}} e C = \qty{1200}{cm^6 mol^{-2}}.
    Admitindo que a lei dos gases perfeitos seja
    suficientemente exata para estimar o segundo e terceiro termos da expansão
    (ou seja, use a lei dos gases perfeitos em caso de necessidade): 
    \begin{enumerate}[label=\alph*.]
        \item calcule o fator de compressibilidade do argônio a 100 atm e 273 K.
            Sugestão: Obtenha uma expressão para Z em função de P com B, C e T
            constantes. 
    \end{enumerate}
\end{xcs}
\begin{rsl}
    %TODO: só explicar a substituição algébrica?
    Vejamos a definição de Z
    \begin{align*}
        Z = \frac{V_m}{V_m^o} 
    \end{align*}
    onde \( V_m \) é o volume molar real do gás e \( V_m^o \) é o volume molar
    ideal do gás. Temos um modelo para gás ideal, de forma que
    \begin{align*}
        Z = \frac{PV_m}{RT} 
    \end{align*}
    Podemos reescrever P na forma da equação do virial
    \begin{align*}
        Z &= \frac{V_m}{RT} . \frac{RT}{V_m} \left(1 + \frac{B}{V_m} +
        \frac{C}{V_m^2}  \right)\\
        &= \left( 1 + \frac{B}{V_m} + \frac{C}{V_m^2}  \right)
    \end{align*}
    Agora, fazemos uma aproximação de um comportamento ideal do gás, de forma
    que
    \begin{align*}
        Z = \left( 1 + \frac{BP}{RT} + \frac{CP^2}{R^2T^2}  \right)
    \end{align*}
    Por fim, determinamos Z a partir dessa equação:
    \begin{align*}
        Z = \left(1 + \frac{-21,7 \cdot 100}{273 \cdot 82,057} + 
        \frac{1200 \cdot 100^2}{273^2 \cdot 82,057^2}\right) 
        = 0,927
    \end{align*}
    Concluímos que \( Z = 0,927 \) para as condições dadas.
\end{rsl}
