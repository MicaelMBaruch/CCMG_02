\section{Introdução}
Fenômenos  térmicos, sua manipulação e entendimento constituem uma parte fundamental do conhecimento humano, em particular dentro da área da termodinâmica. Seu estudo permite compreensão de diversos fenômenos, como mudanças de estado físico, expansão de materiais sob aquecimento e causas de variação de temperatura. Tendo isso em vista, este relatório discute a realização de cinco experimentos que englobam medição de temperatura, evaporação e dilatação de sólidos, líquidos e gases, além do uso utensílios que dependem desses fenômenos para seu funcionamento.

A primeira experimentação envolveu um Termoscópio de Galileu, instrumento inventado pelo cientista florentino que permite uma medição qualitativa da temperatura, fora de escala, sendo um precursor dos termômetros. O fenômeno ocorre graças ao aquecimento do ar em um bulbo que causa deslocamento de uma coluna de água, graças à propriedade geral de que o aumento de temperatura de um gás induz incremento em sua pressão e volume, como descrito pela Lei dos Gases Perfeitos, sendo ela uma boa aproximação para a maioria dos gases \cite{Nussenzveig_2014}:
\begin{align*}
PV = nRT    
\end{align*}
em que \(P\) é a pressão, \(V\) o volume, \(n\) o número de moles, \(R\) a Constante Universal dos Gases e \(T\) a temperatura. Com isso, no experimento é esperado observar uma expansão do gás ao aquecê-lo.

Segue-se então ao Termômetro de Galileu, que utiliza globos de vidro de diferentes densidades imersos em um tubo com um líquido, que flutuam ou afundam em função da variação de temperatura do fluído. O instrumento foi produzido a partir da descoberta de que a densidade de um líquido varia com a temperatura, uma vez que a densidade pode ser calculada por:
\begin{align*}
\rho = \frac{m}{V}    
\end{align*}
onde \(\rho\) corresponde à densidade, \(m\) à massa e \(V\) ao volume. Sendo que no experimento a massa é mantida constante, enquanto o volume segue a fórmula geral para líquidos \cite{Nussenzveig_2014}:
\begin{align*}
\frac{\Delta V}{V} = \beta \Delta T    
\end{align*}
em que \(\Delta V\) é a variação de volume percebida, \(V\) é o volume inicial, \(\beta\) é o coeficiente de dilatação volumétrica e \(\Delta T\) é a variação de temperatura. Por fim, temos que a densidade está relacionada ao empuxo, pelo Princípio de Arquimedes:
\begin{align*}
    E = \rho g V
\end{align*}
no qual \(E\) refere-se ao empuxo, \(\rho\) à densidade, \(g\) à aceleração gravitacional e \(V\) ao volume de fluído deslocado. Com isso, obtém-se um utensílio que possibilita medição de algumas faixas de temperatura dentro de um intervalo a partir da observação da flutuação ou afundamento dos globos.

Em seguida, analisou-se um psicrômetro. Nele utilizam-se de dois termômetros, um deles umedecido, e obtêm-se medições de temperaturas diferentes, por conta da evaporação da água. Isso permite calcular a umidade relativa do ar, uma vez que quanto mais seco o ar, maior a taxa de evaporação \cite{evaporacao}. Essa é uma forma relativamente simples de poder obter-se valores de umidade. Por fim, é esperado que o valor obtido no instrumento seja próximo à umidade real.

Quanto à próxima experimentação, essa envolveu um Anel de Gravesande, descrito pelo matemático holandês Willem Jacob 's Gravesande em 1747 sob o título “Of the Dilatation arising from Heat” \cite{Gravesande_Desaguliers_Adams_1747}. Muito utilizado no ensino de dilatação térmica, o instrumento tem um anel e uma esfera metálica que passa por ele, mas que ao ser aquecida não deve ser capaz de atravessá-lo.

Continuando o tema de dilatação metálica tem-se o último experimento: o par bimetálico. São utilizadas duas lâminas de ligas metálicas com coeficientes de dilatação diferentes, esperando-se observar uma deformação quando aquecidas. A dilatação linear pode ser descrita por \cite{Nussenzveig_2014}:
\begin{align*}
    \Delta l = \alpha l_0 \Delta T 
\end{align*}
Sendo \(\Delta l\) a dilatação medida, \(\alpha\) o coeficiente de dilatação linear, \(l_0\) o comprimento original e \(\Delta T\) a variação de temperatura. Além do conhecimento desse fenômeno ser importante para a criação de estruturas evitando-se que se deformem por variações térmicas, também pode ser aproveitado em sistemas como disjuntores.
\newpage
