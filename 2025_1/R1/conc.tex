\section{Conclusão}
Para concluir, após os dados coletados e analisados, foi possível  demonstrar
como os princípios físicos de hidrostática citados na introdução possibilitaram
a explicação dos fenômenos de cada experimento.

Primeiramente, para o experimento dos discos de Magdeburgo, concluiu-se que
devido a diferença de pressão encontrada nos discos a força necessária para a
separação dos dois discos se tornou muito grande, o que fez com que não fosse
possível com uma força humana, realizar tal separação.

Segundamente, para o experimento do frasco com água e membrana, foi possível
concluir-se que devido a diferença de pressões, e a tensão superficial da água
fez com que a água invertida não vazasse do frasco, esse equilíbrio foi alterado
com base na inclinação do copo devido a diferença de pressão da água nos pontos
da membrana.   

Por fim, foi possível concluir que o manômetro segue completamente a lei de
Stevin. Uma vez que, foi perceptível a variação de altura influenciada pela
variação de pressão ao imergir a extremidade móvel na água.

\newpage
\section{Referências}
NUSSENZVEIG, H. Moysés. Curso de física básica: fluidos, oscilações e ondas,
calor. São Paulo: E. Blücher, 2014.

%TODO: Colocar refs aqui, seguindo ABNT -TF8
