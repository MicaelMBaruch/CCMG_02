\section{Introdução}

Esta introdução fundamenta-se no conteúdo apresentado no livro-texto referência da disciplina: Curso de Física Básica – Vol. 2, 4ª Edição, de H. Moysés Nussenzveig.
O estudo da mecânica dos fluidos é essencial para a compreensão de uma ampla gama de fenômenos naturais e tecnológicos, desde o comportamento da atmosfera terrestre até o funcionamento de dispositivos utilizados em medição e controle de pressão. Fluidos são substâncias que podem escoar e assumir a forma do recipiente que os contém, sendo divididos em líquidos e gases. O comportamento dessas substâncias é descrito por diversas leis e princípios fundamentais, como o princípio de Pascal, os efeitos da tensão superficial, os conceitos de densidade e pressão atmosférica.
O presente relatório tem como objetivo discutir quatro experimentos que abordam aspectos fundamentais da mecânica dos fluidos: densidade, pressão, pressão atmosférica e equilíbrio de pressões. Através desses experimentos, buscou-se entender como os fluidos interagem com seu meio e como as forças atuantes nesses sistemas influenciam diferentes fenômenos físicos.
\subsection{Considerações Históricas}
O estudo da mecânica dos fluidos tem história rica e fundamental na evolução da física. O trabalho de Blaise Pascal no século XVII foi essencial para a compreensão da transmissão da pressão em fluidos, enquanto Otto Von Guericke demonstrou de forma impressionante a força da pressão atmosférica com os Hemisférios de Magdeburg. Essas descobertas foram cruciais para o desenvolvimento de tecnologias modernas, como sistemas hidráulicos e pneumáticos.


\subsection{Experimentos}
Os experimentos que serão realizados explorarão aplicações práticas dos conceitos discutidos nas demonstrações


O primeiro experimento envolverá um densímetro em U, um dispositivo utilizado para medir a densidade de fluidos. Para a melhor compressão, é essencial lembrar que a pressão (\(p\)) em um fluido em repouso é determinada pela equação \(p =\) \(\rho\) \(g\) \(h\), onde \(\rho\) representa a densidade do fluido, \(g\) a aceleração gravitacional e \(h\) a altura da coluna líquida. No densímetro em U, dois fluidos imiscíveis ajustam suas alturas de modo que as pressões nos pontos de equilíbrio se igualem. Com base nesse princípio, o objetivo do experimento é determinar a densidade de um líquido desconhecido a partir da altura relativa das colunas líquidas. Espera-se observar que quanto menor a densidade do líquido desconhecido em relação ao fluido de referência, maior será sua altura na coluna. Esse conceito tem aplicações práticas importantes, como na calibração de misturas na indústria química e na análise de fluidos biológicos, como urina e plasma sanguíneo.


No segundo experimento,  será utilizado um manômetro de tubo aberto, um instrumento projetado para medir pressões relativas. A relação fundamental que rege seu funcionamento é data pela equação de Steven \(\Delta P = \rho g \Delta h\), indicando que a diferença de pressão entre dois pontos é proporcional à diferença de altura do fluido no manômetro. Assim, o objetivo deste experimento é demonstrar a relação entre pressão e altura da coluna líquida, além de medir pressões relativas. Durante a observação, espera-se que o nível do líquido manométrico varie proporcionalmente à diferença de pressão aplicada entre os dois lados do tubo. Essa propriedade torna os manômetros de tubo aberto amplamente utilizados em diversas áreas, como no monitoramento de pressões em tubulações industriais, em sistemas de combustão e até mesmo em dispositivos médicos, como esfigmomanômetros.


Seguindo para o terceiro experimento, serão utilizados os Hemisférios de Magdeburg, um experimento clássico realizado por Otto von Guericke no século XVII para demonstrar a força da pressão atmosférica e o princípio do vácuo parcial. A pressão atmosférica atua em todas as direções e pode ser calculada pela equação \(p =\frac{F}{A}\). Quando o ar dentro dos hemisférios é removido, a força exercida pela atmosfera no exterior impede que eles sejam separados. O objetivo desta demonstração é evidenciar a magnitude da pressão atmosférica, algo que ficará claro ao observar que, mesmo aplicando uma força significativa, os hemisférios não se desprendem. Esse princípio é amplamente explorado em tecnologias modernas, como ventosas, seringas e dispositivos de sucção utilizados na engenharia e na medicina.


Por fim, o último experimento explora o equilíbrio de pressões, ilustrando como a interação entre pressão atmosférica e tensão superficial pode manter um sistema em equilíbrio. A diferença de pressão entre o interior e o exterior do copo cria uma força resultante que impede a saída da água, mesmo quando a cobertura é removida. Assim, o objetivo é demonstrar esse equilíbrio entre a pressão atmosférica e a força de coesão molecular. Durante a observação, espera-se que a água permaneça dentro do copo devido à ação combinada da pressão externa e das forças intermoleculares. Esse fenômeno tem aplicações diretas em diversas áreas, como no armazenamento de líquidos, na engenharia civil e até mesmo em processos biológicos, como a ascensão da seiva nas plantas.
