\section{Conclusão} 
A partir dos experimentos realizados, foi possível demonstrar a aplicabilidade de conceitos da mecânica estatística e da termodinâmica. O primeiro ensaio ilustrou a diferença entre a distribuição normal, observada no tabuleiro de Galton, e a de Maxwell-Boltzmann, que rege o estouro das pipocas. O segundo experimento validou a Lei dos Gases Ideais ao permitir o cálculo da constante R com razoável precisão, enquanto o terceiro demonstrou visualmente a teoria cinética dos gases.

Assim, concluímos que o conjunto de práticas cumpriu sua finalidade de conectar fenômenos macroscópicos observáveis com seus fundamentos microscópicos. Foi evidenciado como modelos estatísticos são capazes de descrever o comportamento de sistemas com um vasto número de partículas, apesar das limitações e incertezas dos métodos de medição empregados.

