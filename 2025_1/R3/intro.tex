\section{Introdução}
A mecânica dos fluidos constitui um dos pilares fundamentais da física clássica, com aplicações que abrangem desde sistemas industriais até fenômenos naturais. Seu estudo experimental remonta ao século XIX com os trabalhos pioneiros de George Gabriel Stokes (1819-1903)\cite{araujo}, que estabeleceu as bases matemáticas para compreender o movimento de corpos em meios viscosos. Neste relatório, investigaremos três fenômenos inter-relacionados: (1) a determinação da viscosidade pelo método de Stokes, (2) a análise da sustentação aerodinâmica em asas, e (3) o comportamento comparativo de fluidos com diferentes viscosidades.
O viscosímetro de Stokes representa uma aplicação direta da teoria desenvolvida por George Stokes em 1851. Seu princípio fundamental baseia-se na medição da velocidade terminal $v_t$ de esferas em queda livre dentro de um fluido, relacionando-a com a viscosidade $\eta$ através da equação:


\begin{align*}
        \eta = \frac{2 g r^2 (\rho_{\text{esfera}} - \rho_{\text{fluido}})}{9 v_t}
\end{align*}

onde $g$ representa a aceleração gravitacional, $r$ o raio da esfera, $\rho_{\text{esfera}}$ e $\rho_{\text{fluido}}$ as densidades do material da esfera e do fluido, respectivamente \cite{Munson2004-tl}. Esta relação permite quantificar propriedades reológicas de diversos líquidos, sendo amplamente aplicada em indústrias química e de materiais.

No estudo da aerodinâmica, investigaremos o fenômeno de sustentação em asas, fundamentado nos trabalhos de Daniel Bernoulli (1700-1782) \cite{moyses}. O princípio de Bernoulli estabelece que:

\begin{align*}
        P + \frac{1}{2} \rho v^2 + \rho gh = \text{constante}
\end{align*}

onde $P$ é a pressão estática, $\rho$ a densidade do fluido, $v$ a velocidade de escoamento, $g$ a aceleração gravitacional e $h$ a altura \cite{Munson2004-tl}. Em perfis aerodinâmicos, a diferença de velocidade entre os fluxos nas superfícies superior e inferior gera uma força de sustentação $L$ dada por:

\begin{align*}
       L = \frac{1}{2} \rho v^2 C_L A
\end{align*}

sendo $C_L$ o coeficiente de sustentação e $A$ a área da asa \cite{moyses}.

Finalmente, analisaremos comparativamente o movimento de esferas em fluidos de diferentes viscosidades. O tempo de queda $t$ em um tubo de altura $h$ pode ser relacionado com a viscosidade pela expressão \cite{Munson2004-tl}:

\begin{align*}
       t = \frac{9\eta h}{2gr^2(\rho_{\text{esfera}} - \rho_{\text{fluido}})}
\end{align*}

Este estudo experimental tem como objetivos, determinar a viscosidade de fluidos padrão pelo método de Stokes, analisar qualitativamente parâmetros de sustentação aerodinâmica e estabelecer relações entre viscosidade e tempo de descida.

A relevância deste trabalho estende-se a aplicações em engenharia aeronáutica, projetos hidrodinâmicos e desenvolvimento de novos materiais \cite{araujo}. Através desta investigação experimental, buscamos consolidar a compreensão dos fenômenos fluidodinâmicos, demonstrando como princípios físicos fundamentais se manifestam em sistemas reais.
