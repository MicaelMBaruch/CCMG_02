\section{Conclusão}

A realização dos experimentos propostos permitiu uma compreensão prática e integrada de diversos conceitos fundamentais da mecânica dos fluidos. Por meio do viscosímetro de Stokes, foi possível observar o movimento de esferas em um fluido e, a partir de medições do tempo e do diâmetro, refletir sobre a relação entre viscosidade e velocidade terminal. O uso de instrumentos como micrômetro, paquímetro e densímetro reforçou a importância da precisão na coleta de dados experimentais.

Os modelos de planagem, especialmente o que incorporava o tubo de Pitot, proporcionaram uma visualização clara do fenômeno de sustentação em asas de avião. As variações de pressão medidas em diferentes regiões da asa explicam, de maneira qualitativa e quantitativa, o surgimento da força de sustentação a partir do escoamento do ar. Isso se alinha com os fundamentos da aerodinâmica, destacando o papel das diferenças de pressão geradas pelo perfil da asa.

Finalmente, o experimento com os tubos fechados e esferas evidenciou de forma simples, porém eficaz, como a viscosidade influencia o movimento de um corpo em um fluido. A análise dos vídeos forneceu uma base quantitativa que complementa as observações qualitativas, permitindo comparar fluidos distintos com base em seu efeito sobre o movimento da esfera.

Portanto, os experimentos realizados não apenas ilustraram conceitos teóricos, como também reforçaram a importância da experimentação para a consolidação do conhecimento em física dos fluidos.