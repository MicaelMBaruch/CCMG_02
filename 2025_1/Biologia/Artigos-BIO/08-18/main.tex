\documentclass{article}
% Author: Micael Baruch adaptado do original de Luan Leal
% Last update: 2025-5-26

% ----------------------------

% ----------------------------   IMPORTS   ----------------------------
\usepackage{amssymb, amsthm, amsmath, geometry, siunitx, caption, float, graphicx, bm}
\usepackage[version=4]{mhchem}
\usepackage{enumitem}
\usepackage[utf8]{inputenc}
\usepackage[onehalfspacing]{setspaceenhanced}
\usepackage[brazil]{babel} % Adaptação ao pt-br
\usepackage{hyperref} % Usado para inserir links
\usepackage[capitalize, brazilian, noabbrev]{cleveref} % Referência adaptada ao pt-br
\usepackage{subcaption}
\usepackage{makecell}
\usepackage[num,overcite]{abntex2cite}

% ----------------------------   LAYOUT   ----------------------------
\citebrackets[]
\geometry{a4paper, lmargin=3cm, tmargin=3cm, rmargin=2cm, bmargin=2cm}
\onehalfspacing
%\setlength{\parindent}{45pt}
\sisetup{output-decimal-marker = {,}}

% ----------------------------  THEOREMS  ----------------------------
% -Ambiente de definição
\theoremstyle{definition}
\newtheorem{dfn}{Definição}[section]

% -Ambiente de observação
\theoremstyle{remark}
\newtheorem{obs}{Observação}

% -Ambiente de lema
\theoremstyle{definition}
\newtheorem{lema}{Lema}

% -Ambiente de exemplo
\theoremstyle{definition}
\newtheorem{xp}{Exemplo}[section]

% -Ambiente de proposição
\newtheorem{prop}{Proposição}

% -Ambiente de teorema e demonstração
\theoremstyle{plain}
\newtheorem{thm}{Teorema}
\theoremstyle{remark}
\newtheorem*{dms}{Demonstração}

% -Ambiente de exercício e resolução
\theoremstyle{definition}
\newtheorem{xcs}{Exercício}
\theoremstyle{remark}
\newtheorem*{rsl}{Resolução}

% ----------------------------  COMMANDS  ----------------------------
%\newcommand{\RR}{\mathbb{R}} % \mathbb{R} = \RR
%\newcommand{\ZZ}{\mathbb{Z}} % \mathbb{Z} = \ZZ

\author{Luan Leal, Micael Baruch}
\usepackage{csquotes}
  \renewcommand{\mkbegdispquote}[2]{\openautoquote}
\title{Análise do artigo:\\
    \texttt{Organelle, Bead, and Microtubule Translocations Promoted by Soluble Factors from the\\ Squid Giant Axon}
    }

\begin{document}
\maketitle

Durante a leitura do artigo, devemos responder as seguintes perguntas:
\begin{enumerate}
    \item Qual o tema principal do artigo?
    \item O que se sabia sobre esse tema antes da realização do estudo em questão? 
    \item Qual o objetivo do estudo, ou seja, qual a pergunta de pesquisa? 
    \item Qual foi a estratégia utilizada, incluindo abordagem experimental e modelos  biológicos? 
    \item Quais os métodos escolhidos e por que esses métodos são adequados ao estudo? 
    \item Descreva os resultados e como cada um contribui para responder a pergunta proposta.
    \item Qual a conclusão do estudo e como isso avança a área?  
\end{enumerate}

\section{Anotações}
\begin{enumerate}
    \item Qual o tema principal do artigo?\\
        Sistema reconstituído para examinar movimento de organelas ao longo de microtúbulos purificados
        
    \item O que se sabia sobre esse tema antes da realização do estudo em questão? \\
        Que microtúbulos estão associados ao movimento de organelas. Que microtúbulos são formados por por \(\alpha\)-tubulina. Uma diversidade de moléculas é transportada por microtúbulos. Sobre o ``motor'' de transporte em si, moléculas distintas são transportadas a uma mesma taxa, indicando um mesmo motor. Motor parece conectar-se a coisas por carga (carregou contas inertes)

    \item Qual o objetivo do estudo, ou seja, qual a pergunta de pesquisa? \\
        Determinar o mecanismo de transporte por microtúbulos.
    \item Qual foi a estratégia utilizada, incluindo abordagem experimental e modelos  biológicos? \\
        Pegaram tubulina de um sobrenadante de alta velocidade (de centrifugação) de um lóbulo óptico de lula e polimerizaram em microtúbulos. Verificaram por eletrofores em gel que deu certo. Removeram proteínas associadas aso microtúbulos com \qty{1}{M} \ce{NaCl}
    \item Quais os métodos escolhidos e por que esses métodos são adequados ao estudo? 
    \item Descreva os resultados e como cada um contribui para responder a pergunta proposta.\\
        Primeiro, foi observado que sem o conteúdo de sobrenadante, não havia movimento promovido pelos microtúbulos sintetizados. 

        Segundo, na presença do sobrenadante, o movimento observado teve velocidade média e comportamento similar ao relatado em microtúbuluos observados de extração. Além disso, o tratamento com um inibidor de movimento conhecido, Vanadato, também inibiu igualmente os microtúbulos extraídos quanto os sintetizados.

        Terceiro, sob tratamento térmico só das organelas ou só do sobrenadante, o movimento era interrompido. Sugerindo que ambos contém proteínas importantes para o movimento.

        Quarto, na ausência de organelas, mas com sobrenadante S2, ainda ocorreu movimento organizado dos próprios microtúbulos.

        Quinto, ao tratar vidro com poly-D-lysina os microtúbulos não apresentaram movimento, mesmo com sobrenadante S2. O movimento das organelas seguiu inalterado.
    \item Qual a conclusão do estudo e como isso avança a área? \\
        Foi proposto um mecanismo para o transporte de moléculas pelos microtúbulos no qual uma única proteína orientada corretamente é responsável pelo transporte de moléculas e microtúbulos em microtúbulos e no vidro. Descobriu-se que esta proteína fica majoritariamente solubilizada, mas pode se atar com a superfície das moléculas a serem transportadas.
\end{enumerate}


Algumas citações diretas do artigo que achamos relevantes para a discussão:
\begin{displayquote}
    Very little is known about the molecular “motor” that powers organelle transport. Because organelles of various types and sizes move at the same rate along isolated microtubules in dissociated axoplasm, all organelles may use the same type of motor (Vale et al., 1985). 
\end{displayquote}
% Deixei de exemplo caso alguém queira usar para colocar coisas mais importantes, depois vejo de dar uma atenção maior nesse artigo e coloco tudo que eu achar relevante por aqui.

\end{document}
