\textbf{(4)} Você sabe que a combustão incompleta de combustíveis fósseis
pode gerar monóxido de carbono e dióxido de carbono, gases importantes no efeito
estufa tido como responsável pelo aquecimento global. Em função disso, o estudo
da reação
\begin{align*}
    \ce{CO(g) + H2O(g) -> CO2(g) + H2(g)}
\end{align*}
é de fundamental importância sob o ponto de vista prático. A partir dos dados fornecidos,
calcule:\\

(a) verifique se a reação é favorável a 298 K.\\

Para verificar se a reação é favorável ou não, precisa-se calcular a variação da energia livre de Gibbs padrão \( \Delta_rG^\circ_{298}\):

\begin{align*}
	\Delta_rG^\circ = \Delta_rH^\circ - T \Delta_rS^\circ
\end{align*}

Dessa forma, primeiro será calculado a variação de entalpia da reação \( \Delta_rH^\circ_{298}\) com os dados fornecidos:

\begin{align*}
    \Delta_rH^\circ &= \sum\Delta_fH^\circ_{\text{produtos}} -
    \sum\Delta_fH^\circ_{\text{reagentes}}\\ \\
    \Delta_rH^\circ &= [\Delta_fH^\circ_{\ce{CO2}} + \Delta_fH^\circ_{\ce{H2}}]
    - [ \Delta_fH^\circ_{\ce{CO}} + \Delta_fH^\circ_{\ce{H2O}}]\\ \\
    \Delta_rH^\circ &= -393,54\unit{kJ \per \mole} - (-352,35\unit{kJ \per
    \mole}) = \qty{-41,19}{kJ \per \mole}
\end{align*}

Então, tomando dos dados \(\Delta_r S^\circ = \qty{-42,4}{J \per \mole . K}\) e
convertendo para \(\qty{-0,0424}{kJ \per mol . K}\), é possível aplicar os valores no cálculo da \(\Delta_r G^\circ_{298}\):
\begin{align*}
	\Delta_r G^\circ &= \qty{-41,19}{kJ\per mol} - (298K \cdot (-0,0424\unit{kJ
    \per \mole . \kelvin})) \\ \\
	\Delta_r G^\circ &= \qty{-28,5548}{kJ\per mol}
\end{align*}

Assim, como \(\Delta_r G^\circ_{298} < 0\), a reação \textbf{é favorável}. \\ 
 
(b) determine a temperatura na qual a reação se torna favorável no sentido 
oposto. Considere que apenas o \(C_{P,m}\), das espécies envolvidas pode ser
considerado constante nesse intervalo de temperatura.\\

Para descobrir a que temperatura a reação inversa se torna favorável, basta descobrir a partir de que temperatura \(\Delta_r G^\circ > 0\):

\begin{align*}
	\Delta_r G^\circ = &\Delta_r H^\circ - T \Delta_r S^\circ \\ \\
	0 < &\Delta_r H^\circ - T \Delta_r S^\circ \\ \\
	T < &\frac{\Delta_r H^\circ}{\Delta_r S^\circ}
\end{align*}

Porém, os dados fornecidos de \(\Delta_r H^\circ\)  e  \(\Delta_r S^\circ\) são apenas relativos a temperatura de \qty{298}{K}, e são necessários os relativos à nova temperatura. Dessa forma, deverá ser usado correções para obter \(\Delta_r H^\circ_{T}\) e \(\Delta_r S^\circ_{T}\)  a partir de \(\Delta_r H^\circ_{298}\) e \(\Delta_r S^\circ_{298} \), e para isso será necessário calcular-se \(\Delta C_{P,m}\):

 \begin{align*}
 	\Delta C_{P,m} &= \sum C_{P,m, produtos} - \sum C_{P,m, reagentes}\\ \\
 	\Delta C_{P,m} &= [C_{P,m} (\qty{298}{K}) (\ce{CO(g)}) + C_{P,m}
    (\qty{298}{K}) (\ce{H2O(g)})] - [ C_{P,m} (\qty{298}{K}) (\ce{CO2(g)}) +
    C_{P,m} (\qty{298}{K}) (\ce{H2(g)})]\\ \\
 	\Delta C_{P,m} &= \qty{0,00321}{kJ \per mol . K}
 \end{align*}
 
Com o valor obtido, deve-se aplicar para realizar a correção dos termos \(\Delta_rH^\circ_{298}\) e \(\Delta_rS^\circ_{298} \):

\begin{align*}
	\Delta_rS^\circ_{T} &= \Delta_rS^\circ_{298} +\Delta C_{P,m} \cdot \ln (\dfrac{T}{298}) \\
	\Delta_rS^\circ_{T} &= \qty{-0,0424}{kJ \per mol . K}
    +\qty{0,00321}{kJ \per mol . K} \cdot \ln (\dfrac{T}{298}) \\ \\
	\Delta_rH^\circ_{T} &= \Delta_rH^\circ_{298} +\Delta C_{P,m} \cdot (T - 298) \\
	\Delta_rH^\circ_{T} &= \qty{-41,19}{kJ \per mol }
    +\qty{0,00321}{kJ \per mol . K} \cdot (T - 298) \\
\end{align*}

Dessa forma, substitui-se esses valores na primeira inequação:

\begin{align*}
    &T < \frac{\Delta_rH^\circ}{\Delta_rS^\circ}\\ 
    &T < \frac{\qty{-41,19}{kJ \per mol } +\qty{0,00321}{kJ \per mol . K} \cdot
    (T - 298)}{\qty{-0,0424}{kJ \per mol . K} +\qty{0,00321}{kJ \per mol K}
    \cdot \ln (\dfrac{T}{298})} \\  
    &T \cdot \qty{-0,0424}{kJ \per mol . K} + T \cdot
        \qty{0,00321}{kJ \per mol . K} \cdot  \ln (\dfrac{T}{298}) <
    \qty{-42,15}{kJ \per mol } + T \cdot \qty{0,00321}{kJ \per mol . K} \\
    &T \cdot \qty{0,00321}{kJ \per mol . K} \cdot \ln (\dfrac{T}{298}) <
       \qty{-42,15}{kJ \per mol } + T \cdot \qty{0,04561}{kJ \per mol . K}\\
    &\qty{0,00321}{kJ \per mol . K} \cdot \ln (\dfrac{T}{298}) <
    \frac{\qty{-42,15}{kJ \per mol }}{T}  + \frac{T \cdot \qty{0,04561}{kJ \per mol . K}}{T} \\
    &\qty{0,00321}{kJ \per mol . K} \cdot \ln (\dfrac{T}{298}) <
    \frac{\qty{-42,15}{kJ \per mol }}{T}  + \qty{0,04561}{kJ \per mol . K} \\
\end{align*}

 Agora, podemos aproximar o valor para \( T \) por aproximações
 consecutivas. Até que a inequação se torne verdadeira. Para
 isso, inciamos calculando o valor de T analiticamente caso \( \Delta C_p = 0 \)
 que resulta em \(T \approx 971 K\). Então, utilizando este valor de T vamos
 calcular se após correção com \( \Delta C_p \) a inequação permanece
 verdadeira. Vamos fazer esta comparação iterativamente, os valores obtidos
 podem ser visualizados na \cref{programafoda}. 

 \begin{table}[H]
     \centering
 \begin{tabular}{c c c}
     \hline
     T(\unit{K}) & \( \Delta_r H^\circ (\unit{kJ \per mol.K}) \) & \(
     \Delta_r S^\circ (\unit{kJ \per mol.K}) \)\\
     \hline
     971 & 0,00379 & -0,0958 \\
     990 & 0,00395 & 0,00303 \\
     1010 & 0,00392 &  0,00388 \\
     1020 & 0,00395 & 0,00428 \\
     1015 & 0,00393 & 0,00408 \\ 
     \hline
 \end{tabular}
 \caption{Valores iterativos de \( T \) para que \( \Delta_r H^\circ < \Delta_r
 S^\circ \)}  
 \label{programafoda}
\end{table}
 
 Como mostrado, a temperatura em que a reação inversa se torna favorável é
 $\approx \qty{1015}{K}$ \\
 
\noindent Dados:\\
\( \Delta_f H^0_{298(\ce{CO(g)})} = \qty{-110.52}{kJ \per mol};\) 
\( \Delta_f H^0_{298(\ce{H2O(g)})} = \qty{-241.83}{kJ \per mol};\) 
\( \Delta_f H^0_{298(\ce{CO2(g)})} = \qty{-392.54}{kJ \per mol}\) \\
Para a reação direta: \( \Delta _r S^0_{298} = \qty{-42.4}{J \per mol . K} \);\\
\(C_{P,m} (\qty{298}{K}) (\ce{CO(g)}) = \qty{29.14}{J \per mol . K}; \)
\(C_{P,m} (\qty{298}{K}) (\ce{H2O(g)}) = \qty{33.58}{J \per mol . K}; \)\\
\(C_{P,m} (\qty{298}{K}) (\ce{CO2(g)}) = \qty{37.11}{J \per mol . K}; \)
\(C_{P,m} (\qty{298}{K}) (\ce{H2(g)}) = \qty{28.82}{J \per mol . K} \)
