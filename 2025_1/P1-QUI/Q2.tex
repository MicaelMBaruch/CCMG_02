\textbf{(2)} Calcule a variação de entalpia relacionada à compressão isotérmica
de 1 mol de \ce{CO2} de 0 atm a 10 atm a 300 K. Para tal, considere que o
\ce{CO2} não pode ser considerado com um gás ideal nessas condições e utilize os
dados abaixo:\\

Calor específico do \ce{CO2} em função da temperatura:
\begin{align*}
    C_{p, \ce{CO2}} (\unit{J \per K . mol}) = \num{26.00} + \num{43.5} \left(\frac{10^{-3}
    T}{\unit{K}}\right) && \qty{273}{K} < T < \qty{1500}{K}
\end{align*}

Valores dos coeficientes de Joule-Thomson do \ce{CO2} em função da pressão:
\begin{align*}
    \mu_{JT, \ce{CO2}} (\unit{K} / \unit{atm}) = \num{0.94} - \num{0.0027} \left(
    \frac{p}{\unit{atm}} \right) - \num{1.51} \left( \frac{10^{-5}
    p^2}{\unit{atm}^2} \right) && \qty{0}{atm} < p < \qty{100}{atm}
\end{align*}

Justifique sua abordagem e mostre as passagens utilizadas.\\

    \textbf{Resposta:} A variação de entalpia \( H \) é dada por:
    \begin{align*}
        dH = -\mu C_p dp + C_p dT
    \end{align*}
    Como a compressão é isotérmica, temos \( dT = 0 \). Portanto, podemos
    simplificar a equação:
    \begin{align*}
        dH = -\mu C_p dp
    \end{align*}
    Queremos calcular \( \Delta H \), dado pela integral:
    \begin{align*}
        \Delta H = \int_{\qty{0}{atm}}^{\qty{10}{atm}} -\mu C_p dp
    \end{align*}
    Temos a função de \( C_p \), calculamos seu valor a \qty{300}{K}:
    \begin{align*}
        C_{p, \ce{CO2}} &= \num{26.00} + \num{43.5} \left( \frac{\qty{300}{K}
            \cdot 10^{-3}}{K}
            \right) (\unit{J \per K . mol})\\
                        &= \num{39.05} \left( \unit{J \per K . mol} \right) 
    \end{align*}
    Como esse valor é constante em relação a \( p \), reescrevemos a integral:
    \begin{align*}
        \Delta H = C_p \int_{\qty{0}{atm}}^{\qty{10}{atm}} -\mu dp
    \end{align*}
    Agora, pela expressão de \( \mu \), temos:
    \begin{align*}
        \Delta H &= - \qty{39.05}{J \per K . mol}
                 \int_{\qty{0}{atm}}^{\qty{10}{atm}} \num{0.94} -
                 \num{0.0027} \left( \frac{p}{\unit{atm}} \right) - \num{1.51} \left(
                 \frac{p^2 \cdot 10^{-5}}{\unit{atm^2}} \right) dp\\
                 &= - \qty{39.05}{J \per K . mol} \left[ \left( \num{0.94}p -
                     \num{0.00135}p^2 \frac{1}{\unit{atm}} - \num{5.033e-6}p^3
                 \frac{1}{\unit{atm^2}} \right) \unit{K \per atm}
         \right]_{\qty{0}{atm}}^{\qty{10}{atm}}\\
                 &= - \qty{39.05}{J \per K . mol} \cdot \qty{9.259967}{K} \\
                 &= - \qty{361.601711}{J \per mol}\\
                 &\approx - \qty{361.6}{J \per mol}
    \end{align*}
    
