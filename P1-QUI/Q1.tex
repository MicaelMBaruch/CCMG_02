\textbf{(1)} Os valores de pressão interna (\(\pi_T\)) de amostras de \ce{C2H4(g)} e
\ce{H2O(g)} a 500 K e pressões próximas a ambiente foram determinados.\\

(a) Explique o significado físico da pressão interna.\\

   \textbf{Resposta:} A pressão interna \( \pi_T \) representa a variação da
   energia interna com a variação do volume com a temperatura constante. Isto é,
   como a energia interna varia quando o gás expande ou contrai adiabaticamente.
   % Adiabaticamente ou isotermicamente?
   Essa variação de energia interna pode ser justificada por interações
   atrativas e repulsivas à nível molecular. Desta forma, temos \( \pi_T = 0 \) para
   um gás ideal. 

(b) Para cada um desses gases, nessas condições, você espera que o valor da
pressão interna \(\pi_T\)  seja igual, menor ou maior que zero? Discuta as
condições de contorno consideradas e explique o raciocínio.\\

    \textbf{Resposta:} À 500 K, em pressão próxima à pressão atmosférica, tanto
    \ce{H2O(g)} quanto \ce{C2H4(g)} já ebuliram. Desta forma, as interações
    % me parece errado assumir que os gases já ebuliram para justificar essa
    % questão
    intermoleculares predominantes em cada um são forças repulsivas.  Portanto,
    ao expandir adiabaticamente, é de se esperar que a energia interna diminua,
    portanto \( \pi_T < 0 \).  
