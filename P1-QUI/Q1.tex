\textbf{(1)} Os valores de pressão interna (\(\pi_T\)) de amostras de \ce{C2H4(g)} e
\ce{H2O (g)} a 500 K e pressões próximas a ambiente foram determinados.\\

(a) Explique o significado físico da pressão interna.\\

\textbf{Resposta:}   A pressão interna \( \pi_T \) representa a variação da energia interna com a
   variação do volume à temperatura constante. Isto é, como a energia
   interna varia quando o gás expande ou contrai isotermicamente. Essa variação
   de energia interna pode ser justificada por interações atrativas e repulsivas
   à nível molecular. Desta forma, \( \pi_T = 0 \) para um gás ideal.\\ 

(b) Para cada um desses gases, nessas condições, você espera que o valor da
pressão interna \(\pi_T\)  seja igual, menor ou maior que zero? Discuta as
condições de contorno consideradas e explique o raciocínio.\\

\textbf{Resposta:} A relação entre \( \pi_T \) e zero depende da natureza das
forças entre as partículas. A expansão de volume aumenta a energia interna caso
as forças dominantes sejam atrativas e diminui caso sejam repulsivas. Como \( dU
= \pi_T dV + C_p dT \), à temperatura constante, aumento na energia interna ocorre
quando \( \pi_T > 0 \). Em particular, podemos determinar se as forças
dominantes são atrativas utilizando a temperatura de Boyle. Como ambas as
temperaturas estão abaixo da temperatura de Boyle, o fator de compressiblidade à
pressão próxima à atmosférica é menor que 1, portanto, o volume do gás real é
menor que o volume esperado para um gás ideal nas mesmas condições. Isto ocorre
devido às forças atrativas dominantes entre as partículas. Desta forma, o
aumento no volume implica aumento na energia interna, implicando que \( \pi_T >
0 \).

Similarmente, caso os gases nesta situação sejam modelados pela equação de Van
der Waals, a derivação termodinâmica indica que \( \pi_T = a \frac{n^2}{V^2} \)
que é sempre um valor positivo.

