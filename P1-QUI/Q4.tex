\textbf{(4)} Você sabe que a combustão incompleta de combustíveis fósseis
pode gerar monóxido de carbono e dióxido de carbono, gases importantes no efeito
estufa tido como responsável pelo aquecimento global. Em função disso, o estudo
da reação
\begin{align*}
    \ce{CO(g) + H2O(g) -> CO2(g) + H2(g)}
\end{align*}
é de fundamental importância sob o ponto de vista prático. A partir dos dados fornecidos,
calcule:\\

(a) verifique se a reação é favorável a 298 K.\\

Para verificar se a reação é favorável ou não, precisa-se calcular a variação da energia livre de Gibbs padrão \( \Delta_rG^\circ_{298}\):

\begin{align*}
	\ce{\Delta_rG^\circ = \Delta_rH^\circ - T \Delta_rS^\circ}
\end{align*}

Dessa forma, primeiro será calculado a variação de entalpia da reação \( \Delta_rH^\circ_{298}\) com os dados fornecidos:

\begin{align*}
	\ce{\Delta_rH^\circ &= \sum\Delta_fH^\circ_{produtos} - \sum\Delta_fH^\circ_{reagentes}}\\ \\
	\ce{\Delta_rH^\circ &= [Delta_fH^\circ_{CO_2} + Delta_fH^\circ_{H_2}] - [ Delta_fH^\circ_{CO} + Delta_fH^\circ_{H_2O}]}\\ \\
	\ce{\Delta_rH^\circ &= -393,54 - (-352,35) \rightarrow -41,19kJmol^{-1}}
\end{align*}

Agora, convertendo \(\Delta_rS^\circ = -42,4Jmol^{-1}K^{-1} \rightarrow -0,0424kJ mol^{-1}K^{-1}\), é possível aplicar os valores no cálculo da \(\Delta_rG^\circ_{298}\):
\begin{align*}
	\ce{\Delta_rG^\circ &= -41,19kJmol^{-1} - (298K * -0,0424)} \\ \\
	\ce{\Delta_rG^\circ &= -28,5548kJmol^{-1}}
\end{align*}

Assim, como \(\Delta_rG^\circ_{298} < 0\), a reação é favorável.
 
(b) determine a temperatura na qual a reação se torna favorável no sentido 
oposto. Considere que apenas o \(C_{P,m}\), das espécies envolvidas pode ser
considerado constante nesse intervalo de temperatura.\\

Para descobrir a que temperatura a reação inversa se torna favorável, é só descobrir a partir de que temperatura \(\Delta_rG^\circ > 0\):

\begin{align*}
	\ce{ &\Delta_rG^\circ = \Delta_rH^\circ - T \Delta_rS^\circ} \\
	0 < &-41,19kJmol^{-1} - (T * - 0,0424kJmol^{-1}K^{-1})\\
	T > 971,46 K
\end{align*}


\noindent Dados:\\
\( \Delta_f H^0_{298(\ce{CO(g)})} = \qty{-110.52}{kJ \per mol};\) 
\( \Delta_f H^0_{298(\ce{H2O(g)})} = \qty{-241.83}{kJ \per mol};\) 
\( \Delta_f H^0_{298(\ce{CO2(g)})} = \qty{-392.54}{kJ \per mol}\) \\
Para a reação direta: \( \Delta _r S^0_{298} = \qty{-42.4}{J \per mol . K} \);\\
\(C_{P,m} (\qty{298}{K}) (\ce{CO(g)}) = \qty{29.14}{J \per mol . K}; \)
\(C_{P,m} (\qty{298}{K}) (\ce{H2O(g)}) = \qty{33.58}{J \per mol . K}; \)\\
\(C_{P,m} (\qty{298}{K}) (\ce{CO2(g)}) = \qty{37.11}{J \per mol . K}; \)
\(C_{P,m} (\qty{298}{K}) (\ce{H2(g)}) = \qty{28.82}{J \per mol . K} \)
