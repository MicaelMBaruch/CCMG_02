\textbf{(4)} Você sabe que a combustão incompleta de combustíveis fósseis
pode gerar monóxido de carbono e dióxido de carbono, gases importantes no efeito
estufa tido como responsável pelo aquecimento global. Em função disso, o estudo
da reação
\begin{align*}
    \ce{CO(g) + H2O(g) -> CO2(g) + H2(g)}
\end{align*}
é de fundamental importância sob o ponto de vista prático. A partir dos dados fornecidos,
calcule:\\

(a) verifique se a reação é favorável a 298 K.\\

(b) determine a temperatura na qual a reação se torna favorável no sentido
oposto. Considere que apenas o \(C_{P,m}\), das espécies envolvidas pode ser
considerado constante nesse intervalo de temperatura.\\

Dados: Falta coisa aqui
