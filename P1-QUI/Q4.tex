\textbf{(4)} Você sabe que a combustão incompleta de combustíveis fósseis
pode gerar monóxido de carbono e dióxido de carbono, gases importantes no efeito
estufa tido como responsável pelo aquecimento global. Em função disso, o estudo
da reação
\begin{align*}
    \ce{CO(g) + H2O(g) -> CO2(g) + H2(g)}
\end{align*}
é de fundamental importância sob o ponto de vista prático. A partir dos dados fornecidos,
calcule:\\

(a) verifique se a reação é favorável a 298 K.\\

(b) determine a temperatura na qual a reação se torna favorável no sentido
oposto. Considere que apenas o \(C_{P,m}\), das espécies envolvidas pode ser
considerado constante nesse intervalo de temperatura.\\

\noindent Dados:\\
\( \Delta_f H^0_{298(\ce{CO(g)})} = \qty{-110.52}{kJ \per mol};\) 
\( \Delta_f H^0_{298(\ce{H2O(g)})} = \qty{-241.83}{kJ \per mol};\) 
\( \Delta_f H^0_{298(\ce{CO2(g)})} = \qty{-392.54}{kJ \per mol}\) \\
Para a reação direta: \( \Delta _r S^0_{298} = \qty{-42.4}{J \per mol . K} \);\\
\(C_{P,m} (\qty{298}{K}) (\ce{CO(g)} = \qty{29.14}{J \per mol . K}; \)
\(C_{P,m} (\qty{298}{K}) (\ce{H2O(g)} = \qty{33.58}{J \per mol . K}; \)\\
\(C_{P,m} (\qty{298}{K}) (\ce{CO2(g)} = \qty{37.11}{J \per mol . K}; \)
\(C_{P,m} (\qty{298}{K}) (\ce{H2(g)} = \qty{28.82}{J \per mol . K} \)
