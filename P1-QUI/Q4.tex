\textbf{(4)} Você sabe que a combustão incompleta de combustíveis fósseis
pode gerar monóxido de carbono e dióxido de carbono, gases importantes no efeito
estufa tido como responsável pelo aquecimento global. Em função disso, o estudo
da reação
\begin{align*}
    \ce{CO(g) + H2O(g) -> CO2(g) + H2(g)}
\end{align*}
é de fundamental importância sob o ponto de vista prático. A partir dos dados fornecidos,
calcule:\\

(a) verifique se a reação é favorável a 298 K.\\

Para verificar se a reação é favorável ou não, precisa-se calcular a variação da energia livre de Gibbs padrão \( \Delta_rG^\circ_{298}\):

\begin{align*}
	\Delta_rG^\circ = \Delta_rH^\circ - T \Delta_rS^\circ
\end{align*}

Dessa forma, primeiro será calculado a variação de entalpia da reação \( \Delta_rH^\circ_{298}\) com os dados fornecidos:

\begin{align*}
    \Delta_rH^\circ &= \sum\Delta_fH^\circ_{\text{produtos}} -
    \sum\Delta_fH^\circ_{\text{reagentes}}\\ \\
    \Delta_rH^\circ &= [\Delta_fH^\circ_{\ce{CO2}} + \Delta_fH^\circ_{\ce{H2}}]
    - [ \Delta_fH^\circ_{\ce{CO}} + \Delta_fH^\circ_{\ce{H2O}}]\\ \\
	\Delta_rH^\circ &= -393,54 - (-352,35) = \qty{-41,19}{kJ . mol^{-1}}
\end{align*}

Agora, convertendo \(\Delta_rS^\circ = \qty{-42,4}{J . mol^{-1} . K^{-1}} =
\qty{-0,0424}{kJ . mol^{-1} . K^{-1}}\), é possível aplicar os valores no cálculo da \(\Delta_rG^\circ_{298}\):
\begin{align*}
	\Delta_rG^\circ &= \qty{-41,19}{kJ.mol^{-1}} - (298K \cdot -0,0424) \\ \\
	\Delta_rG^\circ &= \qty{-28,5548}{kJ.mol^{-1}}
\end{align*}

Assim, como \(\Delta_rG^\circ_{298} < 0\), a reação \textbf{é favorável}. \\ 
 
(b) determine a temperatura na qual a reação se torna favorável no sentido 
oposto. Considere que apenas o \(C_{P,m}\), das espécies envolvidas pode ser
considerado constante nesse intervalo de temperatura.\\

Para descobrir a que temperatura a reação inversa se torna favorável, é só descobrir a partir de que temperatura \(\Delta_rG^\circ > 0\):

\begin{align*}
	\Delta_rG^\circ = &\Delta_rH^\circ - T \Delta_rS^\circ \\ \\
	0 < &\Delta_rH^\circ - T \Delta_rS^\circ \\ \\
	T < &\frac{\Delta_rH^\circ}{\Delta_rS^\circ}
\end{align*}

Porém, os dados fornecidos de \(\Delta_rH^\circ\)  e  \(\Delta_rS^\circ\) são apenas relativos a temperatura de \qty{298}{K}, e são necessários os relativos à nova temperatura. Dessa forma, deverá ser usado correções para obter \(\Delta_rH^\circ_{T}\) e \(\Delta_rS^\circ_{T}\)  a partir de \(\Delta_rH^\circ_{298}\) e \(\Delta_rS^\circ_{298} \), e para isso será necessário calcular-se \(\Delta C_{P,m}\):

 \begin{align*}
 	\Delta C_{P,m} &= \sum C_{P,m, produtos} - \sum C_{P,m, reagentes}\\ \\
 	\Delta C_{P,m} &= [C_{P,m} (\qty{298}{K}) (\ce{CO(g)}) + C_{P,m}
    (\qty{298}{K}) (\ce{H2O(g)})] - [ C_{P,m} (\qty{298}{K}) (\ce{CO2(g)}) +
    C_{P,m} (\qty{298}{K}) (\ce{H2(g)})]\\ \\
 	\Delta C_{P,m} &= \qty{0,00321}{kJmol^{-1}K^{-1}}
 \end{align*}
 
Com o valor obtido, deve-se aplicar para realizar a correção dos termo \(\Delta_rH^\circ_{298}\) e \(\Delta_rS^\circ_{298} \):

\begin{align*}
	\Delta_rS^\circ_{T} &= \Delta_rS^\circ_{298} +\Delta C_{P,m} \cdot \ln (\dfrac{T}{298}) \\
	\Delta_rS^\circ_{T} &= \qty{-0,0424}{kJmol^{-1}K^{-1}}
    +\qty{0,00321}{kJmol^{-1}K^{-1}} \cdot \ln (\dfrac{T}{298}) \\ \\
	\Delta_rH^\circ_{T} &= \Delta_rH^\circ_{298} +\Delta C_{P,m} \cdot (T - 298) \\
	\Delta_rH^\circ_{T} &= \qty{-41,19}{kJmol^{-1}}
    +\qty{0,00321}{kJmol^{-1}K^{-1}} \cdot (T - 298) \\
\end{align*}

Dessa forma, substitui-se esses valores na primeira inequação:

\begin{align*}
    &T = \frac{\Delta_rH^\circ}{\Delta_rS^\circ}\\ 
    &T = \frac{\qty{-41,19}{kJmol^{-1}} +\qty{0,00321}{kJmol^{-1}K^{-1}} \cdot
    (T - 298)}{\qty{-0,0424}{kJmol^{-1}K^{-1}} +\qty{0,00321}{kJmol^{-1}K^{-1}}
    \cdot \ln (\dfrac{T}{298})} \\  
    &T \cdot \qty{-0,0424}{kJmol^{-1}K^{-1}} + T \cdot
        \qty{0,00321}{kJmol^{-1}K^{-1}} \cdot  \ln (\dfrac{T}{298}) =
    \qty{-42,15}{kJmol^{-1}} + T \cdot \qty{0,00321}{kJmol^{-1}K^{-1}} \\
    &T \cdot \qty{0,00321}{kJmol^{-1}K^{-1}} \cdot \ln (\dfrac{T}{298}) =
       \qty{-42,15}{kJmol^{-1}} + T \cdot \qty{0,04561}{kJmol^{-1}K^{-1}}\\
    &\qty{0,00321}{kJmol^{-1}K^{-1}} \cdot \ln (\dfrac{T}{298}) =
    \frac{\qty{-42,15}{kJmol^{-1}}}{T}  + \frac{T \cdot \qty{0,04561}{kJmol^{-1}K^{-1}}}{T} \\
    &\qty{0,00321}{kJmol^{-1}K^{-1}} \cdot \ln (\dfrac{T}{298}) = \frac{\qty{-42,15}{kJmol^{-1}}}{T}  + \qty{0,04561}{kJmol^{-1}K^{-1}} \\
\end{align*}

 Agora, para realizar essa conta de forma mais simplificada pode-se usar um chute consciente para verificar a partir de que ponto a inequação se torna verdadeira. Um chute consciente seria utilizar qual seria o valor de T se não houvesse a correção dos termos, sendo \(T \approx 971 K\):
 
 \begin{align*}
 	&\text{Para T} = \qty{971}{K} \rightarrow 0,00379 > -0,0958 \\
 	&\text{Para T} = \qty{990}{K} \rightarrow 0,00395 > 0,00303 \\
 	&\text{Para T} = \qty{1010}{K} \rightarrow 0,00392 > 0,00388 \\
 	&\text{Para T} = \qty{1020}{K} \rightarrow 0,00395 < 0,00428 \\
 	&\text{Para T} = \qty{1015}{K} \rightarrow 0,00393 < 0,00408 
 \end{align*}
 
 Como mostrado, a temperatura em que a reção inversa se torna favorável é $\approx \qty{1015}{K}$ 
  
\noindent Dados:\\
\( \Delta_f H^0_{298(\ce{CO(g)})} = \qty{-110.52}{kJ \per mol};\) 
\( \Delta_f H^0_{298(\ce{H2O(g)})} = \qty{-241.83}{kJ \per mol};\) 
\( \Delta_f H^0_{298(\ce{CO2(g)})} = \qty{-392.54}{kJ \per mol}\) \\
Para a reação direta: \( \Delta _r S^0_{298} = \qty{-42.4}{J \per mol . K} \);\\
\(C_{P,m} (\qty{298}{K}) (\ce{CO(g)}) = \qty{29.14}{J \per mol . K}; \)
\(C_{P,m} (\qty{298}{K}) (\ce{H2O(g)}) = \qty{33.58}{J \per mol . K}; \)\\
\(C_{P,m} (\qty{298}{K}) (\ce{CO2(g)}) = \qty{37.11}{J \per mol . K}; \)
\(C_{P,m} (\qty{298}{K}) (\ce{H2(g)}) = \qty{28.82}{J \per mol . K} \)
