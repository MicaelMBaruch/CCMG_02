\textbf{(2)} Calcule a variação de entalpia relacionada à compressão isotérmica
de 1 mol de \ce{CO2} de 0 atm a 10 atm a 300 K. Para tal, considere que o
\ce{CO2} não pode ser considerado com um gás ideal nessas condições e utilize os
dados abaixo:\\

Calor específico do \ce{CO2} em função da temperatura:
\begin{align*}
    C_{P, \ce{CO2}} (J / K mol) = \num{26.00} + \num{43.5} \left(\frac{10^{-3}
    T}{\unit{K}}\right) && \qty{273}{K} < T < \qty{1500}{K}
\end{align*}

Valores dos coeficientes de Joule-Thomson do \ce{CO2} em função da pressão:
\begin{align*}
    \mu_{JT, \ce{CO2}} (\unit{K} / \unit{atm}) \num{0.94} - \num{0.0027} \left(
    \frac{P}{\unit{atm}} \right) - \num{1.51} \left( \frac{10^{-5}
    P^2}{\unit{atm}^2} \right) && \qty{0}{atm} < P < \qty{100}{atm}
\end{align*}

Justifique sua abordagem e mostre as passagens utilizadas.\\
