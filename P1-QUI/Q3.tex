\textbf{(3)} Considere o sistema representado abaixo, onde um compartimento de
volume V, dividido em dois volumes de \num{0,25} V e um de \num{0,50} V. Um dos volumes
menores foi preenchido com \num{0,75} mols de \ce{N2} e o outro dos volumes menores com
\num{0,25} mols de \ce{O2}, ambos a 300 K, como ilustrado abaixo. Em um certo momento, a
barreira que divide o volume é removida.  Considere ainda que os gases são
ideais.\\

\begin{figure}[H]
    \centering
    \includegraphics[width=.8\linewidth]{Q3.png}
\end{figure}

(a) Calcule a variação de entropia do \ce{N2} e do \ce{O2} entre os estados
final e inicial sabendo que a temperatura permanece constante. Indique os
cálculos a partir da derivada total apropriada.\\

    \textbf{Resposta:} Para calcular a variação de entropia de cada gás, utilizamos a expressão para a variação de entropia de um gás ideal em processo isotérmico:

    \begin{align*}
    \Delta S = nR \ln\left(\frac{V_f}{V_i}\right)
    \end{align*}
    
    Para o \ce{N2}:
    \begin{itemize}
        \item $n_{\ce{N2}} = \num{0,75}$ mol
        \item $V_i = \num{0,25}V$
        \item $V_f = V$
    \end{itemize}
    
    \begin{align*}
        \Delta S_{\ce{N2}} = \num{0,75}\unit{mol} \cdot R \cdot
        \ln\left(\frac{V}{\num{0,25}V}\right) = (\num{0,75}\unit{mol})R\ln(4)
    \end{align*}

    Para o \ce{O2}:
    \begin{itemize}
        \item $n_{\ce{O2}} = \num{0,25}$ mol
        \item $V_i = \num{0,25}V$
        \item $V_f = V$
    \end{itemize}
    
    \begin{align*}
        \Delta S_{\ce{O2}} = \num{0,25}\unit{mol} \cdot R \cdot
        \ln\left(\frac{V}{\num{0,25}\unit{mol}V}\right) = (\num{0,25}\unit{mol})R\ln(4)
    \end{align*}
    
    Portanto, as variações de entropia são:
    \begin{align*}
        \boxed{\Delta S_{\ce{N2}} = 8,64\unit{J\per K}} \quad \text{e} \quad
        \boxed{\Delta S_{\ce{O2}} = 2,88\unit{J\per K}}
    \end{align*}

(b) Calcule a variação de entropia total do sistema entre o estado inicial e
final (após a remoção das divisórias).\\

    \textbf{Resposta:} A variação de entropia total do sistema é a soma das contribuições de cada gás:

    \begin{align*}
    \Delta S_{\text{total}} = \Delta S_{\ce{N2}} + \Delta S_{\ce{O2}} =
    \num{0,75}R\ln(4) + \num{0,25}R\ln(4) 
    \end{align*}
    
    Resultando em:
    \begin{align*}
        \boxed{\Delta S_{\text{total}} = 11,32\unit{J \per K}}
    \end{align*}

(c) Explique porque a mistura de \ce{O2} e \ce{N2}, nas condições descritas,
pode ser considerada como um processo adiabático (\(dq= 0\)).\\

    \textbf{Resposta:} A mistura de \ce{O2} e \ce{N2} nas condições descritas pode 
    ser considerada um processo adiabático (\(dq = 0\)) porque não há transferência 
    de calor entre o sistema e a vizinhança. Isso ocorre porque a temperatura 
    permanece constante (conforme item (a)) e a expansão dos gases é irreversível, 
    sem realização de trabalho útil. Como os gases são ideais e a energia interna 
    depende apenas da temperatura, a variação de energia interna (\(\Delta U\)) é zero. 
    Pela Primeira Lei da Termodinâmica,

    \begin{align*}
    \Delta U = q + w,
    \end{align*}

    como \(\Delta U = 0\) e \(w = 0\) (não há trabalho contra uma pressão externa, 
    pois a expansão é livre no vácuo), conclui-se que \(q = 0\). Portanto, o processo 
    é adiabático.\\ 

(d) Algumas vezes os alunos respondem o item a dizendo que a variação de
entropia é zero pois a entropia é definida como \(dS= dq_{\text{rev}}/T\) e,
sendo o processo adiabático (item c), \(dS\) também deveria ser zero. Explique
qual é o problema com esse raciocínio.\\

    \textbf{Resposta:} O raciocínio de que a variação de entropia é zero porque 
    \(dS = \frac{dq_{\text{rev}}}{T}\) e o processo é adiabático (\(dq = 0\)) é 
    incorreto porque ignora a natureza irreversível do processo. A definição 
    \(dS = \frac{dq_{\text{rev}}}{T}\) só se aplica a processos reversíveis. 
    Neste caso, a mistura dos gases é um processo irreversível, e a entropia deve 
    ser calculada considerando a expansão isotérmica irreversível de cada gás no 
    volume final. A entropia do sistema aumenta devido ao aumento da desordem 
    molecular, mesmo sem transferência de calor. Portanto, a variação de entropia 
    não é zero, e o erro está em aplicar uma fórmula válida apenas para processos 
    reversíveis a um processo irreversível.\\

