\section{Resultados e discussões}

\subsection{Pipocas e distribuição normal}

\subsection{Pressão na garrafa}

\subsection{Equipamento de teoria cinética dos gases}

Ao ligar o equipamento, observamos que conforme a potência do motor foi aumentada, as bolinhas começaram a se agitar mais intensamente e a distância entre os dois êmbolos aumentou. Ao fixar a altura do pistão superior, segurando-o pela haste, foi possível sentir um aumento na resistência do êmbolo a permanecer no local conforme a potência do motor foi aumentada. 

Como a altura do pistão de baixo era limitada (havia uma altura mínima e uma altura máxima), então sabemos com certeza que a expansão na distância entre os êmbolos tem de ser devido ao aumento na agitação das bolinhas. Isso é precisamente o modelo utilizado na teoria cinética dos gases. Se pensarmos em cada bolinha como uma molécula de gás, o aumento na potência do motor, aumenta a energia cinética média das bolinhas o que é equivalente a pensar no aumento da temperatura de um gás. Como a energia cinética média é maior, a frequência e intensidade com que as bolinhas atingem o êmbolo superior é maior, ou seja, a maior quantidade de movimento sendo transferida para o êmbolo maior. Isso faz com que ele suba ou apresente maior resistência a ficar no mesmo lugar (caso seja segurado). Ou seja, o aumento na temperatura, gera um aumento na pressão caso o volume seja mantido constante ou um aumento no volume caso a pressão seja mantida constante. 
