\section{Discussão}
\subsection{Densímetro}

A partir dos resultados, a lei de Stevin nos permite calcular a densidade do
líquido misterioso:
\begin{align*}
    \rho_l \cdot h_l &= \rho_w \cdot h_w\\
    \rho_l &= \frac{\rho_w \cdot h_w}{h_l}\\
    \rho_l &= \frac{966,5 \cdot 0,123}{0,153} = \qty{767,0}{kg/m^3} 
\end{align*}
Logo, a densidade do líquido misterioso é de \qty{767,0}{kg/m^3}. Consultando
uma tabela de referência, não foi possui atribuir essa densidade a nenhum
líquido. Em seguida, vejamos o cálculo de erro.

\subsubsection{Cálculo de erro}

Considerando os erros experimentais \(\sigma h_w = \pm0,0005 m\) e \(\sigma h_l = \pm 0,0005 m\), temos pela propagação de erro:
\begin{align*}
    \sigma_l &= \sqrt{\left(\frac{\partial\rho_l}{\partial h_w}\right)^2\sigma_{h_w}^2 + \left(\frac{\partial\rho_l}{\partial h_l}\right)\sigma_{h_l}^2}\\
    &= \sqrt{\left(\frac{\rho_w}{h_l}\right)^2 \sigma_{h_w}^2 + \left(\frac{-\rho_2 \cdot h_w}{h_l^2}\right)^2\sigma_l^2}\\
    &= \qty{4,05}{kg/m^3}
\end{align*}

Por fim, mesmo considerando o erro, não conseguimos encontrar um líquido cuja densidade seja compatível com a ora calculada.

\subsection{Manômetro}

Em relação aos resultados obtidos, é possível diferenciar os dados em dois
casos: (a) equilíbrio atmosférico, onde \( h_{\text{atm}} \) é igual a \(
h_{\text{Conect}} \); e (b) submersão da extremidade livre sem vazamento de ar,
onde \( h_{\text{atm}} > h_{\text{Conect}}\). 

No caso (a), temos apenas uma aplicação direta da Lei de Stevin, onde pontos de
mesma altura no líquido possuem a mesma pressão. Enquanto, no caso (b), temos um
deslocamento da coluna de água do recipiente, gerando um aumento da pressão nos
pontos inferiores e, consequentemente, gerando uma diferença de pressão no
manômetro, que pode ser observada pela mudança de alturas em relação ao
equilíbrio. De fato, quanto mais submersa é a extremidade livre, maior o
deslocamento da coluna de água do recipiente que, por sua vez, gera maior
diferença de alturas no manômetro.

\subsection{Discos de Magdeburg}

O resultado observado pode ser explicado pela diferença entre as pressões
interna e externa aos discos. Em especial, consideramos um vácuo parcial onde
adotamos a pressão interna como zero. Vejamos:
\begin{align*}
    \Delta p &= p_{\text{ext}} - p_{\text{int}}\\
    &= p_{\text{ext}}\\
    &= p_{\text{atm}}
\end{align*}
Podemos considerar, então, a existência de uma ``força atmosférica'' distribuída
por toda a superfície dos discos dada por:
\begin{align*}
    F_{\text{atm}} &= p_{\text{atm}} \cdot A_{\text{disco}}
\end{align*}
Em particular, podemos olhar para um dos discos num modelo de corpo livre.
Assim temos:
\begin{align*}
    \sum F = F_{\text{atm}} - T - n = 0
\end{align*}
onde \( T \) é a força de tração exercida e \( n \) é a força normal exercida
entre um disco e outro. Como \( n \) depende da força exercida entre os
discos e a força \( T \) reduz esse contato, temos que conforme \( T \) aumenta
\( n \) diminui. Logo, \( T \) deve ser superior ao valor máximo de \( n
\)\footnote{que ocorre quando \( T=0 \), assim assume o valor de
\(F_{\text{atm}}\)} para que haja movimento do disco, ou seja, 
\( T > F_{\text{atm}} \) para que haja movimento.

Agora, podemos avaliar a força de tração necessária:
\begin{align*}
    F_{\text{atm}} &= p_{\text{atm}} \cdot A_{\text{disco}}\\
    &= (9,235 \cdot 10^4 \unit{Pa}) \cdot (9,62 \cdot 10^{-2} \unit{m})\\
    &= 8,884 \cdot 10^3 \unit{N}
\end{align*}
Logo, a força de tração necessária é de \( 8,884 \cdot 10^3 \unit{N} \). 
Aqui, usamos a pressão atmosférica medida pelo Instituto de Astronomia,
Geofísica e Ciências Atmosféricas da Universidade de São Paulo\footnote{Retirado
de \url{http://www.estacao.iag.usp.br/index.php}}.

\subsubsection{Cálculo de erro}
A área foi calculada a partir do diâmetro do hemisfério, de \qty{35}{cm},
cujo erro associado é \(\sigma_R = \pm 0,5 \unit{cm} =
\pm 0,5 \times 10^{-2} \unit{m} \).
Enquanto o erro associado à pressão atmosférica é \(\sigma_P = \pm 5
\unit{Pa}\).

Por propagação de incerteza (tome \(R\) como raio e \(P\) como pressão):

\begin{align*}
&\sqrt{\left( \frac{\partial }{\partial R} R^{2}\pi P \cdot \sigma_R \right)^{2} + \left( \frac{\partial }{\partial P} R^{2}\pi P \cdot\sigma_P \right)^{2}}\\
&= \sqrt{\left( 2 \cdot 0,175 \pi 92350 \cdot 0,005 \right)^{2} + \left(0,175^{2}\pi \cdot 5 \right)^{2}} \\
&\cong  \pm 5,07 \cdot 10^2 \unit{N}
\end{align*}

Dessa forma, podemos afirmar com mais precisão que seriam necessários
(\(8884,07 \pm 507 \)) N de força para separar as esferas.

\subsection{Frasco com membrana}

Podemos atribuir a permanência do líquido no frasco a dois fenômenos: (1) a
tensão superficial da água em contato com a membrana; (2) a diferença de
pressões interna e externa ao frasco. Inclusive, a inclinação realizada, que
quebra o estado de equilíbrio, se relaciona com o último. 

Podemos afirmar, pela Lei de Stevin, que a pressão da água na membrana, quando o
frasco se encontra alinhado com a vertical é a mesma em todos os pontos.
Contudo, ao ser inclinada, cada altura da membrana possui uma pressão distinta,
em especial, o ponto mais alto da membrana possui a menor pressão. Assim, a
inclinação de desequilíbrio é aquela tal que a pressão da água na membrana se
torne menor que a pressão atmosférica, permitindo a entrada de ar. A partir
desse instante, há uma ruptura do equilíbrio e o líquido escorre.
