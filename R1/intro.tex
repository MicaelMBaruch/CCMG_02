\section{Introdução}

    O estudo de pressão aplicado a fluídos remonta de séculos atrás e apresenta diversas finalidades até os dias atuais, sendo sua compreensão essencial para o entendimento da dinâmica de gases e líquidos, além do estudo da termodinâmica. Para melhor compreensão dos estudantes acerca da diferença de pressão causada por variação de “profundidade” num fluído, além daquela resultante do uso de líquidos de diferentes densidades, foram realizados dois experimentos: manômetro e densímetro em U.
    
	O manômetro é um equipamento utilizado para medir a pressão de um fluido por meio da comparação com uma outra pressão, normalmente de referência como uma atmosfera, usualmente com cada uma em uma extremidade de um tubo. Tal equipamento é amplamente utilizado por exemplo em cilindros de gás pressurizado. A versão utilizada no experimento envolve uma ponta direcionada para cima, sob pressão atmosférica, enquanto a segunda foi inserida numa garrafa de água, para observar-se a variação de pressão conforme se movesse a ponta na água. A experiência permitirá constatar o funcionamento do manômetro, a variação de pressão em um fluido e a comparação entre pressão atmosférica e aquela sob a água.
    
	Já o densímetro em U é um equipamento que possibilita comparar a densidade de dois líquidos, colocados em extremidades opostas do recipiente, dado que não se misturem. Além disso, permite calcular a densidade de um a partir do segundo, sendo ideal para quando sabe-se a densidade de um líquido e deseja-se a do outro, como é o caso do experimento realizado, entre água e um segundo fluido.
    
	Para as conclusões que serão tiradas da experimentação, será preciso conhecimento acerca da pressão em fluidos. Pressão refere-se à força aplicada sobre dada área, sendo medida em N/m² (Pa). Além disso, pelos casos abordados serem líquidos, eles serão tratados como incompressíveis. Em particular será usada a Lei de Stevin, que afirma “a pressão em um ponto do fluido em equilíbrio estático depende da profundidade do ponto e não da dimensão horizontal do recipiente”. Sendo que a pressão a uma dada profundidade pode ser calculada por \(P = \rho gh + Po \) (sendo Po a pressão na superfície do líquido, usualmente a atmosférica). Ademais, como são usados vasos comunicantes, é essencial o conhecimento de que dois pontos em mesma altura em diferentes vasos, desde que conectados pelo fluído, têm a mesma pressão, logo \(Patm + \rho1*g*h1 = Patm + \rho2*g*h2\) e portanto \(\rho1*h1 = \rho 2*h2\).
    
	Pela análise dos dados coletados espera-se poder, pelo experimento do manômetro, descrever o funcionamento do equipamento, além de obter-se informações de diferentes pressões para diferentes profundidades, chegando-se a explicar como elas variam. Já no experimento do densímetro deve-se obter a densidade do segundo líquido pela comparação entre as alturas e utilizando-se a densidade da água.

