\section{Resultados e discussões}


\subsection{Anel de Gravesande}
Ao realizar o experimento do Anel de Gravesande, observou-se que, à temperatura ambiente, a esfera metálica passava justamente através do anel. Após o aquecimento da esfera utilizando um pequeno balão cheio de álcool e com um pavio, notou-se que a esfera não conseguia mais atravessar o anel. Ao deixar a esfera esfriar por alguns instantes, ela voltava a passar pelo anel.

O funcionamento do Anel de Gravesande é uma demonstração clássica do fenômeno da dilatação térmica volumétrica dos sólidos. Em que quando um material é aquecido, a energia térmica fornecida aumenta a energia inética média de seus átomos e moléculas. Isso faz com que eles vibrem com maior amplitude, resultado em um aumento médio nas distancias interatômicas e, consequentemente, na expansão do volume do material.

Tal dilatação volumétrica (\(\Delta V\)) pode ser aproximada pela seguinte relação:
\begin{align*}
	\Delta V = V_0 * \gamma * \Delta T
\end{align*}
sendo, \(V_0\) o volume inicial do objeto, \(\gamma\) é o coeficiente de dilatação volumétrica do material e \(\Delta T\) é a variação de temperatura.

No caso do experimento realizado, a esfera era pequena, em uma ordem de, no máximo, uma dezena de centímetros, e o aquecimento da esfera, por ter sido realizado por pouco tempo e com uma pequena fonte de calor, a alteração de volume na esfera, foi de menos que um centímetro ao quadrado o que ja foi o suficiente para impedir a passagem da esfera pelo anel.

Por fim, essa dinâmica física da dilatação térmica  volumétrica, é vastamente utilizada em diversas áreas da sociedade atual. Alguns exemplos mais práticos, é a utilização dessa dinâmica para a área industrial, para melhor encaixe de peças, e para a de resistência de alguns materiais, outro exemplo bem palpável é a utilização de termômetros de mercúrio, em que seu funcionamento se baseia na dilatação térmica do mercúrio em uma ampola graduada.  

\subsection{Par Bimetálico}
No experimento do par bimetálico, foi observado que, ao aquecer a lâmina bimetálica torcida, observou-se que a lâmina se esticava rapidamente, fazendo com que fosse puxado o ponteiro que estava na extremidade livre, assim fazendo-o movimentar e encostar na outra fase. Ao deixar a lâmina bimetálica esfriar por alguns instantes, observou-se que ela voltou à torção original, com o ponteiro para baixo, na fase inicial.

O funcionamento físico desse experimento do par bimetálico baseia-se na diferença do coeficiente de dilatação linear de dois metais distintos, que compõem a lâmina. Dessa forma, quando a lâmina é aquecida, ambos os metais tentam se expandir. No entanto, o metal com maior coeficiente de dilatação (\(\alpha_1\)) tenderá  a se expandir mais do que o metal com menor coeficiente de dilatação(\(\alpha_2\)) para a mesma variação de temperatura(\(\Delta T\)).

Como os dois metais estão unidos, eles não podem se expandir livremente de forma independente. Para acomodar essa diferença de expansão, a lâmina se curva para o lado que compõe o metal que possui o menor coeficiente de dilatação linear. Dessa forma, como a lamina estava enrolada, e ao aquece-la, observou-se uma diminuição na torção, é possível dizer que a composição da lamina bimetálica tinha o metal com maior coeficiente de dilatação para fora, mostrado com a cor azul na Figura \cref{ParBimetalico}

\begin{figure}[H]
	\centering
	\includegraphics[width=0.25\linewidth]{fig/ParBimetálico.png}
	\caption{Lâmina bimetálica fixa composta por dois metais, representados pelas cores azul e vermelho, e ponteiro representado com verde. Fonte: Autoria própria.}
	\label{ParBimetalico}
\end{figure}

A capacidade do par bimetálico de se curvar de forma previsível com as mudanças de temperatura o torna muito útil em diversas aplicações práticas. Seu uso mais conhecido é em termostatos, onde funciona para controlar a temperatura em sistemas de ar condicionado, aquecedores e refrigeradores. Além disso, essa deformação causada pelo calor é o princípio de funcionamento de termômetros analógicos e de disjuntores térmicos, que são essenciais para proteger circuitos elétricos contra o superaquecimento resultante de correntes excessivas.