\section{Conclusão} 
No estudo do Termômetro de Galileu, observou-se que os bulbos permaneceram submersos, contrariando a expectativa de flutuação diferencial conforme a temperatura. Tal resultado sugere uma possível descalibração do instrumento ou danos aos componentes, impedindo a demonstração prática do princípio da variação da densidade do líquido com a temperatura e seu efeito no empuxo.

O psicrômetro possibilitou a determinação da umidade relativa do ar por meio da diferença de temperatura entre um termômetro seco e um úmido, resultado da evaporação da água no bulbo deste último. A medição obtida foi contextualizada com dados meteorológicos regionais, destacando a influência de fatores locais como a vegetação.

O experimento com o Anel de Gravesande evidenciou o fenômeno da dilatação térmica em sólidos: a esfera metálica, que atravessava o anel à temperatura ambiente, não o fez após ser aquecida, voltando a passar após o resfriamento. Esta prática demonstrou como o aumento da temperatura leva à expansão volumétrica dos materiais.

Finalmente, o par bimetálico ilustrou a dilatação linear diferencial entre dois metais distintos. Ao ser aquecida, a lâmina bimetálica curvou-se devido à diferença nos coeficientes de dilatação dos seus componentes, um princípio fundamental em dispositivos como termostatos e disjuntores.
