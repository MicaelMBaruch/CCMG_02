\section{Metodologia}

\subsection{Termoscópio de Galileu}
O experimento foi realizado com um frasco de vidro conectado a um tubo capilar inserido em um recipiente contendo líquido corante. O frasco foi parcialmente submerso em um recipiente com água aquecida, permitindo a observação do deslocamento da coluna líquida no tubo. A montagem está ilustrada na \cref{fig:termoscopio}.

\begin{figure}[H]
    \centering
    \includegraphics[width=0.35\linewidth]{fig/termoscopio.png}
    \caption{Montagem experimental do termoscópio de Galileu realizada em laboratório. A imagem mostra o frasco de vidro parcialmente submerso em um recipiente com água, conectado a um tubo capilar contendo líquido corante, utilizado para demonstrar a dilatação térmica do ar. Fonte: fotografia tirada durante a aula prática.}
    \label{fig:termoscopio}
\end{figure}

\subsection{Termômetro de Galileu}
O aparato experimental consistiu em um cilindro de vidro preenchido com líquido transparente, contendo esferas calibradas de densidade próxima. As esferas foram observadas em repouso, e sua posição vertical dentro do tubo foi registrada. A \cref{fig:termometro} apresenta a montagem.

\begin{figure}[H]
    \centering
    \includegraphics[width=0.25\linewidth]{fig/termometro.png}
    \caption{Termômetro de Galileu observado durante a aula. O dispositivo contém cápsulas esféricas calibradas, submersas em um líquido transparente dentro de um cilindro de vidro. As diferentes posições das cápsulas ilustram a resposta do sistema à variação de temperatura ambiente. Fonte: registro fotográfico feito no laboratório.}
    \label{fig:termometro}
\end{figure}

\subsection{Higrômetro (Psicrômetro)}
Foram utilizados dois termômetros idênticos, sendo um mantido com o bulbo seco e o outro com o bulbo envolto em gaze úmida. Ambos foram posicionados lado a lado em suporte comum. Após exposição ao ar ambiente por tempo determinado, foram anotadas as temperaturas dos dois termômetros. A montagem pode ser visualizada na \cref{fig:psicrometro}.

\begin{figure}[H]
    \centering
    \includegraphics[width=0.30\linewidth]{fig/psicrometro.png}
    \caption{Psicrômetro montado sobre suporte metálico, com os dois termômetros fixados lado a lado. O bulbo de um dos termômetros está envolvido por gaze umedecida. O experimento foi utilizado para aferição da umidade relativa do ar. Fonte: fotografia registrada durante a atividade experimental.}
    \label{fig:psicrometro}
\end{figure}

\subsection{Anel de Gravesande}
Foi utilizada uma esfera metálica acoplada a uma haste, e um anel metálico com diâmetro interno ligeiramente maior que o da esfera em temperatura ambiente. A esfera foi inicialmente posicionada sobre o anel. Em seguida, foi aquecida com uma fonte térmica até não mais atravessar o anel. Após resfriamento natural, o teste de passagem foi repetido. A montagem é mostrada na \cref{fig:gravesande}.

\begin{figure}[H]
    \centering
    \includegraphics[width=0.30\linewidth]{fig/gravesande.png}
    \caption{Montagem do experimento do Anel de Gravesande. A imagem mostra a esfera metálica acoplada a uma haste sendo testada quanto à passagem pelo anel metálico, antes e após o aquecimento. Fonte: fotografia capturada no momento da prática.}
    \label{fig:gravesande}
\end{figure}

\subsection{Par Bimetálico}
A montagem experimental consistiu em uma lâmina bimetálica fixa em uma extremidade e livre na outra. A lâmina foi aquecida com uma chama direta e observada durante todo o processo. Após resfriamento espontâneo, foi registrada a posição final da lâmina. A \cref{fig:bimetalico} apresenta o aparato utilizado.

\begin{figure}[H]
    \centering
    \includegraphics[width=0.35\linewidth]{fig/bimetalica.png}
    \caption{Lâmina bimetálica fixada em suporte metálico. A curvatura visível na imagem foi induzida por aquecimento com fonte térmica externa. O experimento foi realizado para observar a dilatação diferencial entre os dois metais constituintes. Fonte: imagem feita durante o experimento em aula.}
    \label{fig:bimetalico}
\end{figure}
