\begin{enumerate}
    \item Qual o tema principal do artigo?\\
        Sistema reconstituído para examinar movimento de organelas ao longo de microtúbulos purificados
        
    \item O que se sabia sobre esse tema antes da realização do estudo em questão? \\
        Que microtúbulos estão associados ao movimento de organelas. Que microtúbulos são formados por por \(\alpha\)-tubulina. Uma diversidade de moléculas é transportada por microtúbulos. Sobre o ``motor'' de transporte em si, moléculas distintas são transportadas a uma mesma taxa, indicando um mesmo motor. Motor parece conectar-se a coisas por carga (carregou contas inertes)

    \item Qual o objetivo do estudo, ou seja, qual a pergunta de pesquisa? \\
        Determinar o mecanismo de transporte por microtúbulos.
    \item Qual foi a estratégia utilizada, incluindo abordagem experimental e modelos  biológicos? \\
        Pegaram tubulina de um sobrenadante de alta velocidade (de centrifugação) de um lóbulo óptico de lula e polimerizaram em microtúbulos. Verificaram por eletrofores em gel que deu certo. Removeram proteínas associadas aso microtúbulos com \qty{1}{M} \ce{NaCl}
    \item Quais os métodos escolhidos e por que esses métodos são adequados ao estudo? 
    \item Descreva os resultados e como cada um contribui para responder a pergunta proposta.\\
        Primeiro, foi observado que sem o conteúdo de sobrenadante, não havia movimento promovido pelos microtúbulos sintetizados. 

        Segundo, na presença do sobrenadante, o movimento observado teve velocidade média e comportamento similar ao relatado em microtúbuluos observados de extração. Além disso, o tratamento com um inibidor de movimento conhecido, Vanadato, também inibiu igualmente os microtúbulos extraídos quanto os sintetizados.

        Terceiro, sob tratamento térmico só das organelas ou só do sobrenadante, o movimento era interrompido. Sugerindo que ambos contém proteínas importantes para o movimento.

        Quarto, na ausência de organelas, mas com sobrenadante S2, ainda ocorreu movimento organizado dos próprios microtúbulos.

        Quinto, ao tratar vidro com poly-D-lysina os microtúbulos não apresentaram movimento, mesmo com sobrenadante S2. O movimento das organelas seguiu inalterado.
    \item Qual a conclusão do estudo e como isso avança a área? \\
        Foi proposto um mecanismo para o transporte de moléculas pelos microtúbulos no qual uma única proteína orientada corretamente é responsável pelo transporte de moléculas e microtúbulos em microtúbulos e no vidro. Descobriu-se que esta proteína fica majoritariamente solubilizada, mas pode se atar com a superfície das moléculas a serem transportadas.
\end{enumerate}
