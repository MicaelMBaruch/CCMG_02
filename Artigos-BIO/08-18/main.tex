\documentclass{article}
% Author: Luan Leal
% Last update: 2024-12-03

% ----------------------------

% ----------------------------   IMPORTS   ----------------------------
\usepackage{amssymb, amsthm, amsmath, geometry, siunitx, caption, float, graphicx}
\usepackage{enumitem}
\usepackage[utf8]{inputenc}
\usepackage[onehalfspacing]{setspaceenhanced}
\usepackage[brazil]{babel} % Adaptação ao pt-br
\usepackage{hyperref} % Usado para inserir links
\usepackage[capitalize, brazilian, noabbrev]{cleveref} % Referência adaptada ao pt-br
\usepackage{subcaption}
\usepackage{makecell}
\usepackage[num,overcite]{abntex2cite}

% ----------------------------   LAYOUT   ----------------------------
\citebrackets[]
\geometry{a4paper, lmargin=3cm, tmargin=3cm, rmargin=2cm, bmargin=2cm}
\onehalfspacing
%\setlength{\parindent}{45pt}
\sisetup{output-decimal-marker = {,}}

% ----------------------------  THEOREMS  ----------------------------
% -Ambiente de definição
\theoremstyle{definition}
\newtheorem{dfn}{Definição}[section]

% -Ambiente de observação
\theoremstyle{remark}
\newtheorem{obs}{Observação}

% -Ambiente de lema
\theoremstyle{definition}
\newtheorem{lema}{Lema}

% -Ambiente de exemplo
\theoremstyle{definition}
\newtheorem{xp}{Exemplo}[section]

% -Ambiente de proposição
\newtheorem{prop}{Proposição}

% -Ambiente de teorema e demonstração
\theoremstyle{plain}
\newtheorem{thm}{Teorema}
\theoremstyle{remark}
\newtheorem*{dms}{Demonstração}

% -Ambiente de exercício e resolução
\theoremstyle{definition}
\newtheorem{xcs}{Exercício}
\theoremstyle{remark}
\newtheorem*{rsl}{Resolução}

% ----------------------------  COMMANDS  ----------------------------
%\newcommand{\RR}{\mathbb{R}} % \mathbb{R} = \RR
%\newcommand{\ZZ}{\mathbb{Z}} % \mathbb{Z} = \ZZ

\usepackage{csquotes}
  \renewcommand{\mkbegdispquote}[2]{\openautoquote}
  \renewcommand{\mkenddispquote}[2]{\closeautoquote#1#2}
\author{Luan Leal}
\title{Análise do artigo:\\
    \texttt{Organelle, Bead, and Microtubule Translocations Promoted by Soluble Factors from the\\ Squid Giant Axon}
    }

\begin{document}
\maketitle

Durante a leitura do artigo, devemos responder as seguintes perguntas:
\begin{enumerate}
    \item Qual o tema principal do artigo?
    \item O que se sabia sobre esse tema antes da realização do estudo em questão? 
    \item Qual o objetivo do estudo, ou seja, qual a pergunta de pesquisa? 
    \item Qual foi a estratégia utilizada, incluindo abordagem experimental e modelos  biológicos? 
    \item Quais os métodos escolhidos e por que esses métodos são adequados ao estudo? 
    \item Descreva os resultados e como cada um contribui para responder a pergunta proposta.
    \item Qual a conclusão do estudo e como isso avança a área?  
\end{enumerate}

Algumas citações diretas do artigo que achamos relevantes para a discussão:
\begin{displayquote}
    Very little is known about the molecular “motor” that powers organelle transport. Because organelles of various types and sizes move at the same rate along isolated microtubules in dissociated axoplasm, all organelles may use the same type of motor (Vale et al., 1985). 
\end{displayquote}
% Deixei de exemplo caso alguém queira usar para colocar coisas mais importantes, depois vejo de dar uma atenção maior nesse artigo e coloco tudo que eu achar relevante por aqui.

\end{document}
