\begin{xcs}
    A equação \( P = \frac{RT}{V_m - b} \) é algumas vezes utilizada para
    descrever o comportamento de gases reais. 
    \begin{enumerate}[label=\alph*.]
        \item É possível liquefazer gases que seguem essa equação? Justifique
            seu raciocínio. Sugestão: considere a similaridade da equação com a
            equação de van der Waals. 
    \end{enumerate}
\end{xcs}
\begin{rsl}
    Primeiro, avaliamos a definição de líquido: um fluído constituído por
    partículas globalmente desorganizadas que ocupam um volume máximo e pode
    ser reduzido se aplicada suficiente pressão. O fato de existir um volume
    máximo implica que existe alguma força atrativa entre as partículas.
    Portanto, podemos avaliar a equação e determinar se algum dos termos
    representa tal força. $R$, $T$ e $V_m$ são termos conhecidos e 
    representam: uma constante, a temperatura e o volume, respectivamente, resta, portanto,
    avaliar o significado físico de $b$ e se ele é capaz de modelar as forças atrativas necessárias para liquefazer um dado gás. Note que, a pressão de um gás é diretamente proporcional ao número de colisões nas paredes do recipiente e ao momento linear das partículas participantes dessas colisões; ambos os fenômenos têm seu valores reduzidos pela força de atração, ou seja, se $b$ representa forças atrativas, ele deve ser inversamente proporcional a pressão. Para tanto, avaliamos o comportamento do gás conforme variamos $b$, dada uma mesma configuração $V_m$, $T$ e
    $R$, obtemos que a pressão do gás é diretamente proporcional a $b$ quando $b$>0 e $\neq$ de $V_m$, porém é inversamente proporcional se $b$<0. Fisicamente, podemos interpretar que $b$ pode representar o volume ocupado pelas moléculas, servindo como coeficiente de ajuste do volume ocupado pelo gás . Similarmente,podemos interpretar que $b$ pode descreve as forças repulsivas entre as moléculas.
    Vejamos se \( b \) pode representar forças atrativas, isto é, vejamos se \(
    b\) varia de forma que diminua o número de colisões contra as paredes do
    recipiente e o momento linear das partículas participantes dessas colisões.
     O efeito que essas interações vão
    ter no comportamento do gás depende da densidade de partículas do gás, do tamanho
    delas e da natureza das interações entre elas. Dessa forma, um termo
    que descreva corretamente essas interações deve depender do volume pela
    quantidade de partículas. Ou seja, se \( b \) descrevesse essas interações
    atrativas (de algum modo, talvez assumindo valores negativos),
    então ele deveria se relacionar com \(
    V_m^{-2}\) (como \( a \) na equação de Van der Waals) já que tanto frequência quanto momento dependem de vm neste caso e não seria uma
    simples subtração de \( V_m \).
    Portanto, como não há modelagem de forças atrativas em nosso modelo, um gás 
    que se comporte de acordo com essa equação não poderia se liquefazer.
    % TODO: Melhorar a explicação (se sobrar tempo).
\end{rsl}
