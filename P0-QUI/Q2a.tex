\begin{xcs}
    A equação \( P = \frac{RT}{V_m - b} \) é algumas vezes utilizada para
    descrever o comportamento de gases reais. 
    \begin{enumerate}[label=\alph*.]
        \item É possível liquefazer gases que seguem essa equação? Justifique
            seu raciocínio. Sugestão: considere a similaridade da equação com a
            equação de van der Waals. 
    \end{enumerate}
\end{xcs}
\begin{rsl}
    Primeiro, avaliamos a definição de líquido: um fluído constituído de partículas
    globalmente desorganizadas que ocupam um volume máximo, que pode ser reduzido se
    aplicada suficiente pressão. O fato de existir um volume máximo implica que existe
    alguma força atrativa entre as partículas. Portanto, podemos avaliar a equação e 
    determinar se algum dos termos representa tal força.
    Como $R$, $T$ e $V_m$ são termos conhecidos que representam uma constante, a temperatura
    e o volume, respectivamente, resta avaliar o significado físico de $b$. Para tanto, avaliamos que quanto maior
    o valor de $b$, maior a pressão para uma mesma configuração $V_m$, $T$ e $R$. Similarmente, se 
    o volume diminui dado um valor fixo para $b$, então a pressão aumenta. Dessa forma, quanto maior
    $b$, maior a pressão. Fisicamente, podemos interpretar que $b$ representa o volume ocupado pelas
    moléculas, dado que elas não possam ocupar o mesmo espaço. Similarmente, podemos interpretar que 
    $b$ descreve forças repulsivas entre as moléculas. 
    Por fim, não há forças atrativas em nosso modelo, portanto, um gás que siga literalmente essa 
    equação não poderia se liquefazer.      

    % Terminar (justificar caso negativo e tal.)
\end{rsl}
