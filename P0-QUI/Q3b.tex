\begin{xcs}
    A \qty{273}{K}, o argônio tem os seguintes coeficientes do virial: 
    B = \qty{-21,7}{cm^3 mol^{-1}} e C = \qty{1200}{cm^6 mol^{-2}}.
    Admitindo que a lei dos gases perfeitos seja
    suficientemente exata para estimar o segundo e terceiro termos da expansão
    (ou seja, use a lei dos gases perfeitos em caso de necessidade): 
    \begin{enumerate}[label=\alph*.]
        \item[b.] Explique como o fator de compressibilidade Z varia com a
            temperatura. Faça um esboço indicando o comportamento de Z acima e
            abaixo da temperatura de Boyle (T\(_B\)), identificando também essa
            temperatura no esboço. 
    \end{enumerate}
\end{xcs}
\begin{rsl}
    %TODO: Falta tudo...
    
\end{rsl}
