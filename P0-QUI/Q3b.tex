\begin{xcs}
    A \qty{273}{K}, o argônio tem os seguintes coeficientes do virial: 
    B = \qty{-21,7}{cm^3 mol^{-1}} e C = \qty{1200}{cm^6 mol^{-2}}.
    Admitindo que a lei dos gases perfeitos seja
    suficientemente exata para estimar o segundo e terceiro termos da expansão
    (ou seja, use a lei dos gases perfeitos em caso de necessidade): 
    \begin{enumerate}[label=\alph*.]
        \item[b.] Explique como o fator de compressibilidade Z varia com a
            temperatura. Faça um esboço indicando o comportamento de Z acima e
            abaixo da temperatura de Boyle (T\(_B\)), identificando também essa
            temperatura no esboço. 
    \end{enumerate}
\end{xcs}
\begin{rsl}
    %TODO: Corrigir esse gibbrish, adicionar o esquema
    Conforme a temperatura aumenta, a  energia cinética das partículas de gás aumentam, em determinado ponto, com isso, as forças atrativas entre as partículas torna-se menos relevante para descrever seu comportamento. Portanto, há uma temperatura na qual o comportamento de um gás real se aproxima do comportamento de um gás ideal, essa é a temperatura de Boyle. Essa mesma conjectura pode ser obtida avaliando que os coeficientes do virial variam com a temperatura, portanto sendo possível que a temperatura de Boyle exista. Nesta temperatura as contribuições das forças atrativas e repulsivas se anulam, de forma que o fator de compressibilidade (Z) torna-se 1. A partir da temperatura de Boyle, as forças atrativas tornam-se cada vez menos relevantes, de modo que Z pode ser apenas maior ou igual a 1 de forma estritamente crescente. Esse raciocínio está resumido no esboço %TODO: cross ref com o esboço
    no qual o eixo X é a pressão, o eixo Y é Z e temos três curvas isotérmicas. Comparando pontos isobáricos entre cada curva, podemos observar exatamente a relação enunciada. 
\end{rsl}
