\begin{xcs}
    A \qty{273}{K}, o argônio tem os seguintes coeficientes do virial: 
    B = \qty{-21,7}{cm^3 mol^{-1}} e C = \qty{1200}{cm^6 mol^{-2}}.
    Admitindo que a lei dos gases perfeitos seja
    suficientemente exata para estimar o segundo e terceiro termos da expansão
    (ou seja, use a lei dos gases perfeitos em caso de necessidade): 
    \begin{enumerate}[label=\alph*.]
        \item[b.] Explique como o fator de compressibilidade Z varia com a
            temperatura. Faça um esboço indicando o comportamento de Z acima e
            abaixo da temperatura de Boyle (T\(_B\)), identificando também essa
            temperatura no esboço. 
    \end{enumerate}
\end{xcs}
\begin{rsl}
    %TODO: Corrigir esse gibberish, adicionar o esquema
    Quando a temperatura aumenta, a energia cinética das partículas de gás
    aumentam, em determinado ponto, as forças atrativas entre as partículas
    podem ser desprezadas. Como o fator de compressibilidade é determinado com
    base na variação de volume, é vazio discutir o caso de volume fixo, invés
    disso, é necessário estudar o que ocorre se o volume do sistema é
    extremamente responsivo às variações, ou seja, a pressão é praticamente
    constante. Neste cenário, quando a energia cinética das partículas é grande
    suficiente, a força aplicada por elas em contato com as paredes do
    recipiente fazem o volume se expandir. Dessa forma, em uma determinada
    temperatura, as repulsões entre as partículas também podem ser
    desconsideradas, porque torna-se extremamente raro que uma partícula se
    aproxime de outra o suficiente para sentir qualquer repulsão. No entanto, há
    uma temperatura na qual ainda há atração relevante entre as partículas que
    faz o gás ocupar menos volume do que um gás ideal, ao mesmo tempo em que a
    frequência com que se chocam faz com que o gás se expanda mais que o ideal.
    Nesta temperatura as contribuições das forças atrativas e repulsivas se
    anulam e é chamada Temperatura de Boyle. Nesta temperatura, o fator de
    compressibilidade torna-se 1. A partir da temperatura de Boyle, as forças
    atrativas tornam-se cada vez menos relevantes, de modo que Z pode ser apenas
    maior ou igual a 1 de forma estritamente crescente.
\end{rsl}
