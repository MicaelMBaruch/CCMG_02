\documentclass{article}
% Author: Micael Baruch adaptado do original de Luan Leal
% Last update: 2025-5-26

% ----------------------------

% ----------------------------   IMPORTS   ----------------------------
\usepackage{amssymb, amsthm, amsmath, geometry, siunitx, caption, float, graphicx, bm}
\usepackage[version=4]{mhchem}
\usepackage{enumitem}
\usepackage[utf8]{inputenc}
\usepackage[onehalfspacing]{setspaceenhanced}
\usepackage[brazil]{babel} % Adaptação ao pt-br
\usepackage{hyperref} % Usado para inserir links
\usepackage[capitalize, brazilian, noabbrev]{cleveref} % Referência adaptada ao pt-br
\usepackage{subcaption}
\usepackage{makecell}
\usepackage[num,overcite]{abntex2cite}

% ----------------------------   LAYOUT   ----------------------------
\citebrackets[]
\geometry{a4paper, lmargin=3cm, tmargin=3cm, rmargin=2cm, bmargin=2cm}
\onehalfspacing
%\setlength{\parindent}{45pt}
\sisetup{output-decimal-marker = {,}}

% ----------------------------  THEOREMS  ----------------------------
% -Ambiente de definição
\theoremstyle{definition}
\newtheorem{dfn}{Definição}[section]

% -Ambiente de observação
\theoremstyle{remark}
\newtheorem{obs}{Observação}

% -Ambiente de lema
\theoremstyle{definition}
\newtheorem{lema}{Lema}

% -Ambiente de exemplo
\theoremstyle{definition}
\newtheorem{xp}{Exemplo}[section]

% -Ambiente de proposição
\newtheorem{prop}{Proposição}

% -Ambiente de teorema e demonstração
\theoremstyle{plain}
\newtheorem{thm}{Teorema}
\theoremstyle{remark}
\newtheorem*{dms}{Demonstração}

% -Ambiente de exercício e resolução
\theoremstyle{definition}
\newtheorem{xcs}{Exercício}
\theoremstyle{remark}
\newtheorem*{rsl}{Resolução}

% ----------------------------  COMMANDS  ----------------------------
%\newcommand{\RR}{\mathbb{R}} % \mathbb{R} = \RR
%\newcommand{\ZZ}{\mathbb{Z}} % \mathbb{Z} = \ZZ

\author{Matheus Justino; Matheus Queiroz; Micael Baruch; Luan Leal}
\title{0-ésima Prova de Química II}

\begin{document}
\maketitle

    \textbf{Utilize caso achar necessário \(R = \qty{8,3145}{J\ K^{-1} mol^{-1}}, \
    R = \qty{0,082057}{L\ atm\ K^{-1} mol^{-1}}, \ 
    R = \qty{82,05745}{cm^3\ atm\ K^{-1} mol^{-1}}, \
    R = \qty{1,897}{cal\ K^{-1} mol^{-1}}\).
    Justifique sua resposta e mostre as etapas de cálculo. Não esqueça das
    unidades!}
    
    \textbf{(1)} Os valores de pressão interna (\(\pi_T\)) de amostras de \ce{C2H4(g)} e
\ce{H2O (g)} a 500 K e pressões próximas a ambiente foram determinados.\\

(a) Explique o significado físico da pressão interna.\\

\textbf{Resposta:}   A pressão interna \( \pi_T \) representa a variação da energia interna com a
   variação do volume com a temperatura constante. Isto é, como a energia
   interna varia quando o gás expande ou contrai isotermicamente. Essa variação
   de energia interna pode ser justificada por interações atrativas e repulsivas
   à nível molecular. Desta forma, \( \pi_T = 0 \) para um gás ideal. 

(b) Para cada um desses gases, nessas condições, você espera que o valor da
pressão interna \(\pi_T\)  seja igual, menor ou maior que zero? Discuta as
condições de contorno consideradas e explique o raciocínio.\\

\textbf{Resposta:} A relação entre \( \pi_T \) e zero depende da natureza das
forças entre as partículas. Em particular, podemos determinar se as forças
dominantes são atrativas utilizando a temperatura de Boyle. Como ambas as
temperaturas estão abaixo da temperatura de Boyle, o fator de compressiblidade à
pressão próxima à atmosférica é menor que 1, portanto, o volume do gás real é
menor que o volume esperado para um gás ideal nas mesmas condições. Isto ocorre
devido às forças atrativas dominantes entre as partículas. Desta forma, o
aumento no volume implica aumento na energia interna, implicando que \( \pi_T >
0 \).

    \begin{xcs}
    A equação \( P = \frac{RT}{V_m - b} \) é algumas vezes utilizada para
    descrever o comportamento de gases reais. 
    \begin{enumerate}[label=\alph*.]
        \item É possível liquefazer gases que seguem essa equação? Justifique
            seu raciocínio. Sugestão: considere a similaridade da equação com a
            equação de van der Waals. 
    \end{enumerate}
\end{xcs}
\begin{rsl}
    Primeiro, avaliamos a definição de líquido: um fluído constituído de
    partículas globalmente desorganizadas que ocupam um volume máximo, que pode
    ser reduzido se aplicada suficiente pressão. O fato de existir um volume
    máximo implica que existe alguma força atrativa entre as partículas.
    Portanto, podemos avaliar a equação e determinar se algum dos termos
    representa tal força.  Como $R$, $T$ e $V_m$ são termos conhecidos que
    representam uma constante, a temperatura e o volume, respectivamente, resta
    avaliar o significado físico de $b$. Para tanto, avaliamos que quanto maior
    o valor de $b$, maior a pressão para uma mesma configuração $V_m$, $T$ e
    $R$. Similarmente, se o volume diminui dado um valor fixo para $b$, então a
    pressão aumenta. Dessa forma, quanto maior $b$, maior a pressão.
    Fisicamente, podemos interpretar que $b$ representa o volume ocupado pelas
    moléculas, dado que elas não possam ocupar o mesmo espaço. Similarmente,
    podemos interpretar que $b$ descreve forças repulsivas entre as moléculas.
    Vejamos se \( b \) pode representar forças atrativas, isto é, vejamos se \(
    b\) varia de forma que diminua o número de colisões contra as paredes do
    recipiente e o momento linear das partículas participantes dessas colisões.
    Se \( b \) descrevesse essa variação, então ele deveria se relacionar com \(
    V_m^{-2}\) (como \( a \) na equação de Van der Waals)  e não seria uma
    simples subtração de \( V_m \).
    Por fim, não há forças atrativas em nosso modelo, portanto, um gás que siga
    literalmente essa equação não poderia se liquefazer.      

    % TODO: Melhorar a explicação (se sobrar tempo).
\end{rsl}

    \begin{xcs}
    A equação \( P = \frac{RT}{V_m - b} \) é algumas vezes utilizada para
    descrever o comportamento de gases reais. 
    \begin{enumerate}[label=\alph*.]
        \item[b.] Discuta as condições que uma equação deve satisfazer para ser
            empregada como uma equação de estado de um gás real e verifique se a
            equação acima satisfaz tais condições (ou seja, demonstre
            matematicamente). 
    \end{enumerate}
\end{xcs}
\begin{rsl}
    %TODO: Falta tudo que antecede a prova matemática.
    Agora, vamos provar que \( V_m \to b^+ \) implica em \( P(V_m) \to +\infty \).
    Para isso, basta verificar que 
    \begin{align*}
        \lim_{V_m \to b^+} P(V_m) = \lim_{V_m \to b^+} \frac{RT}{V_m-b} = +\infty
    \end{align*}
    isto é, \( \forall M > 0 \), \( \exists \delta > 0 \) tal que:
    \begin{align*}
        0 < V_m - b < \delta \Rightarrow M < \frac{RT}{V_m-b}
    \end{align*}
    Note que para \( \delta = \frac{RT}{M} \), temos
    \begin{align*}
        V_m - b &< \frac{RT}{M}\\
        M &< \frac{RT}{V_m - b} 
    \end{align*}
    que é exatamente o que queríamos. Logo, concluimos que \( V_m \to b^+
    \Rightarrow P(V_m) \to +\infty \).
\end{rsl}

    \begin{xcs}
    A \qty{273}{K}, o argônio tem os seguintes coeficientes do virial: 
    B = \qty{-21,7}{cm^3.mol^{-1}} e C = \qty{1200}{cm^6 mol^{-2}}.
    Admitindo que a lei dos gases perfeitos seja
    suficientemente exata para estimar o segundo e terceiro termos da expansão
    (ou seja, use a lei dos gases perfeitos em caso de necessidade): 
    \begin{enumerate}[label=\alph*.]
        \item calcule o fator de compressibilidade do argônio a 100 atm e 273 K.
            Sugestão: Obtenha uma expressão para Z em função de P com B, C e T
            constantes. 
    \end{enumerate}
\end{xcs}
\begin{rsl}
    %TODO: só explicar a substituição algébrica?
    Vejamos a definição de Z
    \begin{align*}
        Z = \frac{V_m}{V_m^o} 
    \end{align*}
    onde \( V_m \) é o volume molar real do gás e \( V_m^o \) é o volume molar
    ideal do gás. Temos um modelo para gás ideal, de forma que
    \begin{align*}
        Z = \frac{PV_m}{RT} 
    \end{align*}
    Podemos reescrever P na forma da equação do virial
    \begin{align*}
        Z &= \frac{V_m}{RT} . \frac{RT}{V_m} \left(1 + \frac{B}{V_m} +
        \frac{C}{V_m^2}  \right)\\
        &= \left( 1 + \frac{B}{V_m} + \frac{C}{V_m^2}  \right)
    \end{align*}
    Agora, fazemos uma aproximação de um comportamento ideal do gás, de forma
    que
    \begin{align*}
        Z = \left( 1 + \frac{BP}{RT} + \frac{CP^2}{R^2T^2}  \right)
    \end{align*}
    Por fim, determinamos Z a partir dessa equação:
    \begin{align*}
        Z = \left(1 + \frac{-21,7 \cdot 100}{273 \cdot 82,057} + 
        \frac{1200 \cdot 100^2}{273^2 \cdot 82,057^2}\right) 
        = 0,927
    \end{align*}
    Concluímos que \( Z = 0,927 \) para as condições dadas.
\end{rsl}

    \begin{xcs}
    A \qty{273}{K}, o argônio tem os seguintes coeficientes do virial: 
    B = \qty{-21,7}{cm^3 mol^{-1}} e C = \qty{1200}{cm^6 mol^{-2}}.
    Admitindo que a lei dos gases perfeitos seja
    suficientemente exata para estimar o segundo e terceiro termos da expansão
    (ou seja, use a lei dos gases perfeitos em caso de necessidade): 
    \begin{enumerate}[label=\alph*.]
        \item[b.] Explique como o fator de compressibilidade Z varia com a
            temperatura. Faça um esboço indicando o comportamento de Z acima e
            abaixo da temperatura de Boyle (T\(_B\)), identificando também essa
            temperatura no esboço. 
    \end{enumerate}
\end{xcs}
\begin{rsl}
    %TODO: Corrigir esse gibberish, adicionar o esquema
    Quando a temperatura aumenta, a energia cinética das partículas de gás
    aumentam, em determinado ponto, as forças atrativas entre as partículas
    podem ser desprezadas. Como o fator de compressibilidade é determinado com
    base na variação de volume, é vazio discutir o caso de volume fixo, invés
    disso, é necessário estudar o que ocorre se o volume do sistema é
    extremamente responsivo às variações, ou seja, a pressão é praticamente
    constante. Neste cenário, quando a energia cinética das partículas é grande
    suficiente, a força aplicada por elas em contato com as paredes do
    recipiente fazem o volume se expandir. Dessa forma, em uma determinada
    temperatura, as repulsões entre as partículas também podem ser
    desconsideradas, porque torna-se extremamente raro que uma partícula se
    aproxime de outra o suficiente para sentir qualquer repulsão. No entanto, há
    uma temperatura na qual ainda há atração relevante entre as partículas que
    faz o gás ocupar menos volume do que um gás ideal, ao mesmo tempo em que a
    frequência com que se chocam faz com que o gás se expanda mais que o ideal.
    Nesta temperatura as contribuições das forças atrativas e repulsivas se
    anulam e é chamada Temperatura de Boyle. Nesta temperatura, o fator de
    compressibilidade torna-se 1. A partir da temperatura de Boyle, as forças
    atrativas tornam-se cada vez menos relevantes, de modo que Z pode ser apenas
    maior ou igual a 1 de forma estritamente crescente.
\end{rsl}


\end{document}
