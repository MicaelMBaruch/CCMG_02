\begin{xcs}
    A \qty{273}{K}, o argônio tem os seguintes coeficientes do virial: 
    B = \qty{-21,7}{cm^3 mol^{-1}} e C = \qty{1200}{cm^6 mol^{-2}}.
    Admitindo que a lei dos gases perfeitos seja
    suficientemente exata para estimar o segundo e terceiro termos da expansão
    (ou seja, use a lei dos gases perfeitos em caso de necessidade): 
    \begin{enumerate}[label=\alph*.]
        \item calcule o fator de compressibilidade do argônio a 100 atm e 273 K.
            Sugestão: Obtenha uma expressão para Z em função de P com B, C e T
            constantes. 
    \end{enumerate}
\end{xcs}
\begin{rsl}
    %TODO: Parece que falta tudo? Que coisa...
    \begin{align*}
        Z = (1 + \frac{-21,7 \cdot 100}{273 \cdot 82,057} + 
        \frac{1200 \cdot 100^2}{273^2 \cdot 82,057^2}) 
        = 0,927
    \end{align*}
\end{rsl}
