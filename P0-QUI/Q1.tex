\begin{xcs}
    Em um experimento realizado com 1,0000 mol de N\(_2\) gasoso a 0,00°C, os
    seguintes volumes foram observados em função da pressão: 
    \begin{center}
    \begin{tabular}{c | c c c}
    \hline
        P/atm & 1,0000 & 3,0000 & 5,0000\\
        V/cm\(^3\) mol\(^{-1}\) & 22405 & 7461,4 & 4473,1\\
    \hline
    \end{tabular}
    \end{center}
\end{xcs}
\begin{rsl}
    Para realizar o cálculo do valor da constante universal dos gases \( R \),
    poderemos utilizar a fórmula da lei dos gases ideais.  
    \[ PV_m = RT \rightarrow R = \frac{PV_m}{T} \] 
    Dessa forma, sabemos que serão necessários
    os dados de Pressão, Volume molar e Temperatura, para calcularmos o valor da
    constante universal dos gases.  Temos que a temperatura será constante em
    0,00°C, ou seja 273,15 K, e temos os dados de três casos de Pressão e
    Volumes diferentes.  Calculando a constante para cada caso que temos:
    Pressão a 1,0000 atm: 
    \[ R = \frac{PV_m}{T} \rightarrow R = \frac{1,0000 \, \text{atm} \cdot 22405
    \, \text{cm}^3 \text{mol}^{-1}}{273,15 \, \text{K}} \rightarrow R = 82,02452
    \, \text{cm}^3 \text{atm} \, \text{mol}^{-1} \, \text{K}^{-1} \] 
    Pressão a 3,0000 atm: 
    \[ R = \frac{PV_m}{T} \rightarrow R =
    \frac{3,0000 \, \text{atm} \cdot 7461,4 \, \text{cm}^3
    \text{mol}^{-1}}{273,15 \, \text{K}} \rightarrow R = 81,02452 \, \text{cm}^3
    \text{atm} \, \text{mol}^{-1} \, \text{K}^{-1} \] 
    Pressão a 5,0000 atm: 
    \[ R = \frac{PV_m}{T} \rightarrow R = \frac{5,0000 \, \text{atm} \cdot
    4473,1 \, \text{cm}^3 \text{mol}^{-1}}{273,15 \, \text{K}} \rightarrow R =
    81,87991 \, \text{cm}^3 \text{atm} \, \text{mol}^{-1} \, \text{K}^{-1} \]
\end{rsl}

