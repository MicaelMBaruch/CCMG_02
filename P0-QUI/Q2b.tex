\begin{xcs}
    A equação \( P = \frac{RT}{V_m - b} \) é algumas vezes utilizada para
    descrever o comportamento de gases reais. 
    \begin{enumerate}[label=\alph*.]
        \item[b.] Discuta as condições que uma equação deve satisfazer para ser
            empregada como uma equação de estado de um gás real e verifique se a
            equação acima satisfaz tais condições (ou seja, demonstre
            matematicamente). 
    \end{enumerate}
\end{xcs}
\begin{rsl}
    %TODO: Falta tudo que antecede a prova matemática.
    Agora, vamos provar que \( V_m \to b^+ \) implica em \( P(V_m) \to +\infty \).
    Para isso, basta verificar que 
    \begin{align*}
        \lim_{V_m \to b^+} P(V_m) = \lim_{V_m \to b^+} \frac{RT}{V_m-b} = +\infty
    \end{align*}
    isto é, \( \forall M > 0 \), \( \exists \delta > 0 \) tal que:
    \begin{align*}
        0 < V_m - b < \delta \Rightarrow M < \frac{RT}{V_m-b}
    \end{align*}
    Note que para \( \delta = \frac{RT}{M} \), temos
    \begin{align*}
        V_m - b &< \frac{RT}{M}\\
        M &< \frac{RT}{V_m - b} 
    \end{align*}
    que é exatamente o que queríamos. Logo, concluimos que \( V_m \to b^+
    \Rightarrow P(V_m) \to +\infty \).
\end{rsl}
