\begin{xcs}
    A equação \( P = \frac{RT}{V_m - b} \) é algumas vezes utilizada para
    descrever o comportamento de gases reais. 
    \begin{enumerate}[label=\alph*.]
        \item[b.] Discuta as condições que uma equação deve satisfazer para ser
            empregada como uma equação de estado de um gás real e verifique se a
            equação acima satisfaz tais condições (ou seja, demonstre
            matematicamente). 
    \end{enumerate}
\end{xcs}
\begin{rsl}
    %TODO: Sei lá, sinto que dei um bilhão de voltas e não provei muita coisa
    %TODO: No entanto, ... pela quantidade de partículas.
    Para que uma equação possa ser empregada como equação de estado de um gás real,
    deve satisfazer as seguinte condições:
    \begin{enumerate}
        \item Considerar o volume das partículas;
        \item Considerar as forças de atração e repulsão entre as partículas;
    \end{enumerate}
    Vamos avaliar se a equação dada condiz com esses critérios. Para o primeiro
    critério, vamos
    provar que \( V_m \to b^+ \) implica em \( P(V_m) \to +\infty \), isto é, \(
    b\) representa o volume das partículas (ou as forças repulsivas).
    Para isso, basta verificar que 
    \begin{align*}
        \lim_{V_m \to b^+} P(V_m) = \lim_{V_m \to b^+} \frac{RT}{V_m-b} = +\infty
    \end{align*}
    isto é, \( \forall M > 0 \), \( \exists \delta > 0 \) tal que:
    \begin{align*}
        0 < V_m - b < \delta \Rightarrow M < \frac{RT}{V_m-b}
    \end{align*}
    Note que para \( \delta = \frac{RT}{M} \), temos
    \begin{align*}
        V_m - b &< \frac{RT}{M}\\
        M &< \frac{RT}{V_m - b} 
    \end{align*}
    que é exatamente o que queríamos. Logo, concluimos que \( V_m \to b^+
    \Rightarrow P(V_m) \to +\infty \). Portanto, \( b \) representa interações
    repulsivas entre as partículas, isto é, para o modelo de esferas rígidas
    temos que \( b \) representa o volume das partículas individuais. Note que é
    plausível dizer que uma equação considere simultaneamente as forças de
    repulsão e o volume das partículas, já que as forças repulsivas dependem
    usualmente de uma distância tão pequena para serem sentidas que é
    equivalente a dizer que as partículas estão se tocando. 

    Para discutirmos as interações atrativas desse modelo, podemos retomar ao
    exercício anterior. Já sabemos que seria vazio acusar que o valor de $b$
    poderia ser negativo e representar forças atrativas que de alguma forma
    diminuem a pressão causada por um gás em condições isovolumétricas. Como
    nenhum outro termo da equação modela interações atrativas \footnote{Assim
    como \( \cfrac{a}{V_m^2} \) modela na
    equação de Van der Waals}, provamos que a
    equação a cima não pode descrever um gás real, pois não atende ao requisito
    de considerar as forças repulsivas \textbf{e} atrativas do gás
    consigo mesmo.   
\end{rsl}
