% Author: Micael Baruch adaptado do original de Luan Leal
% Last update: 2025-5-26

% ----------------------------

% ----------------------------   IMPORTS   ----------------------------
\usepackage{amssymb, amsthm, amsmath, geometry, siunitx, caption, float, graphicx, bm}
\usepackage[version=4]{mhchem}
\usepackage{enumitem}
\usepackage[utf8]{inputenc}
\usepackage[onehalfspacing]{setspaceenhanced}
\usepackage[brazil]{babel} % Adaptação ao pt-br
\usepackage{hyperref} % Usado para inserir links
\usepackage{booktabs} % Para melhores tabelas
\usepackage[capitalize, brazilian, noabbrev]{cleveref} % Referência adaptada ao pt-br
\usepackage{subcaption}
\usepackage{makecell}
\usepackage[num,overcite]{abntex2cite}

% ----------------------------   LAYOUT   ----------------------------
\citebrackets[]
\geometry{a4paper, lmargin=3cm, tmargin=3cm, rmargin=2cm, bmargin=2cm}
\onehalfspacing
%\setlength{\parindent}{45pt}
\sisetup{output-decimal-marker = {,}}

% ----------------------------  THEOREMS  ----------------------------
% -Ambiente de definição
\theoremstyle{definition}
\newtheorem{dfn}{Definição}[section]

% -Ambiente de observação
\theoremstyle{remark}
\newtheorem{obs}{Observação}

% -Ambiente de lema
\theoremstyle{definition}
\newtheorem{lema}{Lema}

% -Ambiente de exemplo
\theoremstyle{definition}
\newtheorem{xp}{Exemplo}[section]

% -Ambiente de proposição
\newtheorem{prop}{Proposição}

% -Ambiente de teorema e demonstração
\theoremstyle{plain}
\newtheorem{thm}{Teorema}
\theoremstyle{remark}
\newtheorem*{dms}{Demonstração}

% -Ambiente de exercício e resolução
\theoremstyle{definition}
\newtheorem{xcs}{Exercício}
\theoremstyle{remark}
\newtheorem*{rsl}{Resolução}

% ----------------------------  COMMANDS  ----------------------------
%\newcommand{\RR}{\mathbb{R}} % \mathbb{R} = \RR
%\newcommand{\ZZ}{\mathbb{Z}} % \mathbb{Z} = \ZZ
