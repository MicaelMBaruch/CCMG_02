\section{Resultados}
    Seguindo os procedimentos detalhados na metodologia, obtemos, após análise pelo Tracker, a \cref{canudo}. 
    \begin{figure}[H]
        \centering
        \includegraphics[width=.5\linewidth]{figs/canudo.png}
        \caption{Configuração experimental dos canudos carregados suspensos como pêndulos, mostrando a distância de equilíbrio entre eles.}
        \label{canudo}
    \end{figure}
    Observe que o canudo superior ficou suspenso a uma distância de \qty{3.669}{\centi\meter} do canudo inferior. Esse fenômeno será utilizado para determinar a ordem de grandeza da carga nos canudos.
\\ 
Seguindo a medida da separação entre os canudos (\(r = \qty{3.669}{\centi\meter}\)), podemos determinar o ângulo de deflexão \(\theta\) em função do comprimento do fio \(L\) pela relação geométrica:

\begin{align*}
r &= 2 L \sin\theta \\
\implies \theta &= \arcsin\frac{r}{2L}.
\end{align*}

A força eletrostática \(F_{e}\) entre os canudos, em equilíbrio, é dada por:

\begin{align*}
F_{e} = mg \tan\theta,
\end{align*}

onde \(m\) é a massa de cada canudo. 

Aplicando a Lei de Coulomb, a carga em cada canudo pode ser estimada por:

\begin{align*}
F_{e} &= k \frac{q^2}{r^2} \\
\implies q &= \sqrt{\frac{F_{e} r^2}{k}},
\end{align*}

onde \(k = \dfrac{1}{4\pi\varepsilon_0}\). 

A estimativa da carga obtida a partir das medições experimentais indica uma ordem de grandeza para a carga em torno de \(q \sim 10^{-9}\,\text{C}\), compatível com valores típicos de eletrização por atrito.
