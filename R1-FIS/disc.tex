\section{Discussão}
\section{Modelagem matemática de uma esfera carregada sob o campo do Van der Graff}
Conforme discutido anteriormente, duas cargas iguais exercem uma força de repulsão entre si. Podemos, então extrapolar este resultado para modelar o ângulo formado por uma esfera carregada com carga de mesmo sinal pendurada por um fio isolante há uma distância \(r\) do centro do van de Graff

\begin{align*}
    {|T|} &= \frac{{|P|}}{\cos \theta}\\
    {|T|} &= \frac{{|F_{e}|}}{\sin \theta}\\
    \implies \frac{{|P|}}{\cos \theta} &= \frac{{|F_{e}|}}{\sin \theta}\\
    \implies \theta &= \arctan \left( \frac{{|F_{e}|}}{{|P|}}\right) 
\end{align*}

Podemos utilizar a lei de Coulomb para determinar a força elétrica em função da carga \(Q\) do van der Graff:
\[
    F_{e} = \frac{qQ}{4 \pi \varepsilon_{0} r^{2}}
\]
Assim, substituindo na fórmula:
\begin{align*}
    \theta &= \arctan \left( \frac{qQ}{4 \pi \varepsilon_{0}r^{2}{mg}} \right)
\end{align*}

Com isso, seria possível testar este ângulo como medida indireta da lei de Coulomb. No entanto, não pudemos realizar esta medida pela falta de uma bola e um fio isolante no lab demo.
