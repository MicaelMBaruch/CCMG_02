\section{Introdução}
A observação e estudo de fenômenos que deram origem ao campo da eletrostática, ramo da física que investiga as propriedades e o comportamento de cargas elétricas em repouso, tem suas origens na Grécia Antiga. As primeiras observações documentadas são atribuídas ao filósofo Tales de Mileto\cite{ipolito_2021}, que descreveu como o âmbar (``\textit{elektron}'' em grego) após ser atritado adquiria a propriedade de atrair corpos leves, como fios de cabelo e fios de palha. Porém, foi apenas em 1600, quando o médico e físico inglês William Gilbert, em sua obra seminal \textit{De Magnete}, se aprofundou e sistematizou o estudo desses efeitos e cunhou o termo ``eletricidade'' para descrever a propriedade exibida por diversos materiais após serem atritados\cite{Silva2011Eletrostatica}.

No século XVIII marcou-se uma evolução conceitual de suma importância para o estudo da eletrostática. Em 1733, o físico francês Charles du Fay descobriu a existência de dois tipos distintos de eletricidade, que denominou ``vítrea'' e ``resinosa''\cite{introducao_eletrostatica}, que viriam a ser denominadas como positivas e negativas respectivamente. Ele observou experimentalmente a regra fundamental das interações eletrostáticas: cargas de mesmo tipo se repelem, enquanto cargas de tipos opostos se atraem. Pouco depois, na segunda metade do século XVIII, o físico e ex-presidente americano Benjamin Franklin propôs um modelo de fluido elétrico único, estabelecendo a convenção, utilizada até hoje, de cargas ``positiva'' (um excesso do fluido) e ``negativa'' (uma deficiência do fluido). A partir de suas observações, Franklin concluiu que ``... um ou mais corpos devem ganhar fogo elétrico de corpos que perdem-no (esta afirmação é hoje conhecida como a lei de conservação da carga elétrica)'' (BASSALO, 1996, p. 301 apud SILVA, 2011, p. 101).

A transição da eletrostática de uma ciência qualitativa para uma quantitativa foi consolidada em 1785 pelo físico francês Charles-Augustin de Coulomb. Utilizando uma balança de torção de alta sensibilidade, ele demonstrou empiricamente que a força de interação entre duas cargas puntiformes é diretamente proporcional ao produto de suas magnitudes e inversamente proporcional ao quadrado da distância que as separa\cite{Nussenzveig1997}.
\begin{equation}
    \vec{F}_{2(1)} = \frac{1}{4\pi\epsilon_0} \frac{q_1 q_2}{r_{12}^2} \hat{r}_{12}
\end{equation}

No final do século XIX e início do século XX, a natureza da carga elétrica foi revelada como sendo corpuscular e quantizada. A descoberta do elétron pelo cientista Joseph John Thomson em 1897\cite{lage_2021} e, de forma conclusiva, o experimento da gota de óleo de Robert Millikan em 1909\cite{Silva2011Eletrostatica}, demonstraram que a carga elétrica existe em pacotes, múltiplos de uma carga elementar fundamental, $e \approx 1,602 \times 10^{-19} C$. Ou seja, qualquer carga macroscópica é o resultado de um excesso ou de uma falta de um número inteiro de elétrons.

Os processos de eletrização são os mecanismos pelos quais essa transferência de carga ocorre. Dentre esses processos, temos a eletrização por atrito, ou efeito triboelétrico\cite{introducao_eletrostatica}, que acontece quando o contato intenso entre dois materiais distintos permite que elétrons sejam transferidos de um para o outro\cite{eletrizacao2015}. O material que perde elétrons fica positivamente carregado, enquanto o que os recebe, fica negativamente carregado. Outro processo é a eletrização por contato, onde um condutor carregado transfere parte de sua carga a um condutor ao tocá-lo, sempre transferindo suas cargas de forma a tentar manter um equilíbrio no sistema\cite{eletrizacao2015}. E por fim, tem-se a eletrização por indução, na qual a aproximação de um corpo carregado (indutor) a um condutor neutro provoca uma separação de cargas no condutor. Esta separação permite carregar o condutor com carga de sinal oposto à do indutor sem que haja contato direto entre eles\cite{introducao_eletrostatica}.

Como a carga elementar fundamental é muito pequena, para possibilitar o estudo experimental da eletrostática em escala macroscópica é necessário acumular grandes quantidades de carga, assim gerando diferenças de potencial elevadas que produzirão efeitos observáveis. O dispositivo mais emblemático para este fim é o gerador de Van de Graaff, inventado em 1929\cite{jesus2021}. O funcionamento do gerador de Van de Graaff começa quando um motor gira uma correia sobre um rolete inferior, gerando cargas por atrito. Este processo deixa a correia com carga positiva e o rolete com carga negativa. O campo elétrico intenso do rolete ioniza o ar ao redor, e os íons positivos do ar são depositados sobre a correia em movimento. A correia transporta continuamente essas cargas positivas para o topo, onde uma escova superior as transfere para a superfície externa da esfera metálica. Como a correia desce neutra, o ciclo se repete, fazendo com que a esfera acumule cada vez mais carga elétrica. Esse acúmulo é limitado apenas quando o campo elétrico se torna forte o suficiente para ionizar o ar ao redor da esfera, através do Efeito Corona\cite{jesus2021}.
