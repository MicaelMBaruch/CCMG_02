\section{Resultados e discussões}
\subsection{Pêndulo de Wilberforce}
Neste experimento foi analisado o movimento do Pêndulo de Wilberforce, no qual foi possível observar-se a curiosa dinâmica que define o fenômeno,
em qual períodos de oscilação no eixo longitudinal se alternam com oscilações no eixo rotacional. Aprofundando-se na análise 
do funcionamento do pêndulo, é possível observar que a alternância de oscilações ocorre com uma dinâmica bem estabelecida, 
quando a massa do pêndulo oscila verticalmente, a mola se comprime e alonga, e consequentemente, seu diâmetro aumenta e diminui,
dando uma ligeira torção ao peso. Dessa forma, ocorre a transferência gradativa entre energias potenciais e cinéticas, o que
faz com que os modos de oscilação se alternem gradativamente, resultando em uma relação inversamente proporcional entre as oscilações.

Uma característica muito importante para a dinâmica do pêndulo, é que, para que ocorra a máxima transferência de energia entre os modos
de oscilação é necessário que os mesmos estejam em frequências similares, para entrarem em ressonância, e o ponto de máxima energia potencial
do movimento de um eixo aconteça enquanto a energia cinética do movimento do outro eixo esteja também em seu ponto máximo.

Essa dinâmica de um sistema físico ou eletrônico no qual dois ou mais osciladores que interagem entre si, de forma que a oscilação de 
um afeta a oscilação dos outros, é denominada como oscilador acoplado. Esses sistemas fazem parte da vida cotidiana da maioria da 
população, um grande exemplo é a maquina de lavar vertical. Uma vez que, como um pêndulo de Wilberforce, ela realiza oscilações no 
eixo rotacional e oscilações no eixo longitudinal. Essa dinâmica de um oscilador acoplado, lhe confere a propriedade de transferência 
gradual de energia entre os modos de oscilação, e dependência da ressonância entre as oscilações para melhor funcionamento do equipamento.
Dessa forma, se a frequência da rotação estiver muito descalibrada, devido ao acumulo de massa em um ponto do cesto da máquina por exemplo,
 pode ocasionar em um transferência excessiva de energia para a oscilação longitudinal, o que pode danificar o maquina.

  