\section{Resultados e discussões}
\section{Figuras de Chladni}
%quais fatores influenciam na forma das figuras? Quantas figuras são possíveis?

Ao friccionar o arco de violino contra a placa no ângulo e velocidade corretos, foi possível observar que a areia sobre a placa agitou-se e passou a acumular-se em algumas linhas, formando figuras. Este fenômeno ocorreu em ambas as placas. Cada placa formou padrões diferentes entre si. Além disso, ao excitar a placa em pontos diferentes com o arco, figuras diferentes foram formadas. Também, pressionar o dedo contra a placa em algum ponto ocasionou mudanças na figura. Em particular, o ponto pressionado sempre passa a ser parte de uma das linhas de areia formadas. Dependendo da combinação entre pontos pressionados e posição do arco para excitar a placa, figuras com menos resolução poderiam ser observadas, não formando um padrão claro.

Este fenômeno é explicado pela formação de modos de vibração, no qual as ondas excitadas na placa pelo arrastar do arco de violino têm máxima interferência construtiva. Neste cenário, formam-se ondas estacionárias com pontos da placa tendo deslocamento vertical máximo (pontos de antinós) e pontos na placa com nenhum deslocamento vertical (pontos de nó). Ao vibrar, a placa arremessa a areia, portanto a areia se acumula nos pontos de nós da placa naquele modo de vibração. 

Placas não seguem o padrão de tubos ou cordas de ter modos normais como múltiplos de uma frequência fundamental. Devido à isto, os modos normais não podem ser obtidos através de uma única equação de onda estacionária e a determinação destes modos nem sempre têm solução analítica. Em particular, ao variar os pontos da placa nos quais pressionamos o dedo, variamos às condições de contorno de vibração da placa, forçando a existência de um ponto de nó no ponto em que pressionamos o dedo. Similarmente, ao variar a posição em que o arco é friccionado forçamos a formação de um antinó naquele ponto. A variação dessas condições de contorno altera os modos normais da placa sob estas condições. Portanto, levando em consideração estas condições, é possível variar uma enorme diversidade de figuras em uma mesma placa. 

Indo além, seria possível variar também a placa. Evidentemente, que como a excitação da placa pela fricção com o arco depende da força elástica da placa, isto é, da rigidez do material, esta alteração também fomentaria a formação de outras figuras em comparação à placa inicial em condições de contorno iguais. Por fim, a geometria da placa faz com que as forças atuando sobre cada ponto da placa ao deformar um ponto (pela fricção do arco de violino) também seja diferente visto que a força elástica dependerá da deformação da vizinhança do ponto. E, é em decorrência disto, que foi possível observar a formação de figuras distintas na placa retangular e na placa triangular. 

\section{Tubo de Rijke}
%explique os fenômenos envolvidos neste experimento 
%discuta em termos de similaridades entre o som que sai de tubos fechados e abertos.
%O que muda ao deixarmos o tubo na horizontal ou na vertical?

Ao esquentar a grade de metal e erguer o tubo foi possível ouvir nitidamente um duradouro som em alto volume. Enquanto o som era reproduzido, virar o tubo para posição horizontal resultou na extinção do som. A origem do som emitido tem relação com o sentido de convecção do ar.

Ao aquecer a grade de metal, o ar envolta da grade se expande, ao mesmo tempo, o ar quente se move para cima devido à esta expansão. Ao expandir-se, o ar próximo à grade de metal empurra o ar mais acima, comprimindo-o. Enquanto a vela está aquecendo o tubo, o fluxo de ar que entra é quente, fazendo com que todo ar dentro do tubo se expanda. Porém, ao retirar o tubo de cima da vela, a expansão de ar quente causa uma diminuição na pressão dentro do tubo, por sua vez, esta diferença de pressão entre o interior e exterior do tubo faz com que novas moléculas de ar frio entrem por baixo do tubo. Então, estas novas moléculas entram em contato com a grande quente e se expandem, aquecidas, as moléculas sobem, empurrando e comprimindo o ar um pouco mais frio logo a cima, a menor pressão dentro do tubo faz com que mais moléculas frias entrem e o ciclo se repete periodicamente. Este ciclo de compressão e expansão de ar é o que origina o som. 

Como o tubo tem as extremidades abertas, sabemos que na extremidade superior, a onda de pressão possui um nó, pois naquele ponto a pressão deve se manter constante e igual à pressão atmosférica. 
