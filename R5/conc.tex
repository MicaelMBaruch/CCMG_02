\section{Conclusão}
\section{Conclusão} 

Os experimentos realizados demonstraram os princípios fundamentais do movimento oscilatório e suas aplicações práticas. No estudo do sistema massa-mola, a relação linear da Lei de Hooke foi confirmada, com a frequência angular mostrando dependência direta com a constante elástica e a massa. O experimento com o pêndulo de Wilberforce revelou o fenômeno de acoplamento de modos vibratórios, onde se observou a transferência periódica de energia entre os movimentos longitudinal e torsional, relacionando com o movimento periódico de uma máquina de lavar. Na investigação de ondas estacionárias em cordas, verificou-se a formação de harmônicos discretos, com nós em posições bem definidas. Por fim, o estudo do oscilador amortecido evidenciou o decaimento exponencial da amplitude, com o coeficiente de amortecimento mostrando dependência clara com a viscosidade do meio, confirmando o comportamento previsto teoricamente. Estes resultados não apenas corroboram os modelos teóricos, mas também ilustram sua relevância para compreender fenômenos que vão desde aplicações tecnológicas até sistemas naturais complexos.