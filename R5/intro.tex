\section{Introdução}
O estudo de osciladores harmônicos em muito contribui para a física ondulatória, responsável pela descrição da propagação de perturbações, ou energia, sem haver transporte de matéria. O conhecimento de propriedades ondulatórias é aplicado em diversas áreas, como óptica, eletromagnetismo, quântica e acústica. Para o estudo de sons, que é o enfoque deste relatório, é essencial o conhecimento de que tratam-se de ondas mecânicas longitudinais associadas a variações de pressão, causando regiões de compressão e rarefação no meio \cite{Nussenzveig_2014}. Além disso, vale ressaltar que suas propriedades são perceptíveis à audição humana, como a amplitude sentida na intensidade sonora, a frequência que é notada na afinação (frequências mais altas sendo mais agudas) e interferência de múltiplas ondas percebida pelo timbre. Por fim, constata-se a importância do estudo do som por tratar-se de um fenômeno cotidiano e uma peça essencial da experiência humana do mundo ao seu redor, bem como valores utilitários em áreas como medicina (como no uso de ultrassom), transmissão de informação, detecção de objetos (como por sonares) e inspeção de maquinário industrial.

Tendo isso em vista, este trabalho discute a realização de quatro experimentos sobre propriedades sonoras. O primeiro envolveu o uso de diapasões para gerar ondas de frequências determinadas sobre um tubo com água de altura variável. Esperava-se notar a liberação de som pelos tubos quando a água estivesse próxima a alturas marcadas no tubo, relacionadas às frequências dos diapasões. O objetivo é poder explicar como esse efeito está relacionado a harmônicos em tubos fechados e interferência de ondas estacionárias, além de estimar-se a velocidade do som no ar a partir dos conhecimentos de ondulatória e comparação de diferentes alturas da coluna da água que geram sons de mesma frequência. Por fim, o estudo de harmônicos em tubos fechados é relevante inclusive para o entendimento de instrumentos de sopro, como flautas e clarinetes, que dependem desse fenômeno.

Quanto ao segundo experimento, esse foi da geração de figuras de Chladni em duas placas metálicas cobertas de areia, com o uso de um arco de violino. Ernst Chladni (1756-1827) foi um físico e músico alemão, famoso pela descoberta da formação de padrões em placas vibrantes sobre as quais colocou areia, havendo variação dos desenhos dependendo da frequência de vibração, formato da placa e se fixava/pressionava alguma parte dela \cite{santos2017descobertas}. Tal experimentação permite tanto uma perspectiva visual do som, como uma apreciação do funcionamento de uma superfície vibrante bidimensional.

Já a terceira experimentação envolveu uma recriação do experimento descrito por P. L. Rijike em 1859 \cite{Rijke_1859}, na qual uma rede metálica dentro de um tubo é aquecida e posteriormente retira-se a fonte de calor, resultando na geração de som. Também foi observado o que ocorre ao inclinar-se o tubo e como o som variava ao utilizar-se tubos de dimensões diferentes, sendo esperada uma alteração na frequência em função do tamanho \cite{sarpotdar2003rijke}. Por fim, vale ressaltar a questão particular deste experimento, de envolver geração de som a partir de energia térmica.

Por fim, o quarto experimento envolveu o uso de uma corda esticada sobre uma haste móvel, similar ao monocórdio de Pitágoras \cite{inbook0dd942e5}, tendo uma extremidade presa e outra ligada a pesos variáveis. Com isso, é possível a análise dos sons gerados com variação de comprimento e tensão da corda, uma vez que a frequência é proporcional à raiz quadrada da tensão da corda e inversamente proporcional ao comprimento da corda e à raiz quadrada da sua densidade linear, como descrito pelas leis de Mersenne \cite{Nussenzveig_2014} e sintetizado na seguinte expressão: 
\begin{align*}
	v_1 = \frac{1}{2l} \sqrt{\frac{T}{\mu}}
\end{align*}
Sendo \(v_1\) a frequência fundamental, \(l\) o comprimento da corda, \(T\) a tensão da corda e \( \mu\), a sua densidade linear de massa.
Essas propriedades representam a base do entendimento dos instrumentos de corda e das notas musicais.
