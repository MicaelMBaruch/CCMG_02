\section{Introdução}
O estudo do movimento oscilatório descreve uma série de fenômenos, desde cotidianos como pêndulos e molas, até mais fundamentais como propagação de ondas eletromagnéticas e vibrações moleculares que podem ser tratadas como osciladores harmônicos quânticos \cite{libretexts76Quantum}. Em particular, tais fenômenos podem ser modelados, ao menos de uma forma mais simples (que despreza a ação de forças externas) pelo oscilador harmônico simples (OHS), cuja uma das primeiras descrições foi feita pelo matemático Leonard Euler \cite{barros2023oscilador}, sendo que apresenta a seguinte fórmula:
\begin{align*}
	x(t) = A \cos{(\omega t + \phi)}
\end{align*}
em que \(x(t)\) corresponde à posição da massa oscilante em função do tempo, \(A\) à amplitude do movimento, \(\omega\) à frequência angular e \(\phi\) à fase.
Tendo em vista a importância do estudo do movimento oscilatório, este relatório busca discutir algumas de suas principais características e aplicações, como descrição de sistemas de molas, acoplamento entre diferentes tipos de movimento harmônico em um mesmo objeto, ondas em fios com formação de harmônicos e oscilações amortecidas por atrito.

O primeiro experimento utiliza sistemas massa mola, buscando observar como a variação do tipo da mola e das massas colocadas afetam o movimento, em particular sua frequência. Além disso, pela experimentação é possível obter-se a constante elástica da mola, a partir do uso da Lei de Hooke:
\begin{align*}
	Fel = -kx
\end{align*}
em que \(Fel\) é a força realizada pela mola sobre o objeto, \(k\) é a constante elástica e \(x\) representa a distorção da mola em relação ao seu comprimento relaxado.

Já o segundo experimento envolve o uso de um pêndulo de Wilberforce, inventado por Lionel Robert Wilberforce para o estudo de propriedades de materiais \cite{wilberforceRef}.Tal pêndulo utiliza uma mola capaz de compressão longitudinal e torção transversal, o que possibilita a transferência de energia entre o movimento oscilatório vertical e a rotação, havendo um acoplamento de oscilações. Ademais, além da descrição desse fenômeno, é esperado que possam ser feitas relações com o funcionamento de uma máquina de lavar, por conta do movimento de seu tambor que apresenta características similares às do pêndulo de Wilberforce.

Quanto ao terceiro experimento, nele é analisada a propagação de ondas mecânicas, ou seja, fenômenos de pulsos de propagação de energia em meio material, sem transferência de matéria. São produzidas ondas longitudinais e transversais em molas esticadas, sendo possível gerar ondas senoidais, que visualmente deslocam-se no eixo, com perturbação que varia no tempo, e estacionárias, quando ocorre interferência entre ondas indo e voltando no eixo, causando nós nos pontos de interferência destrutiva e apresentando padrões de ondas que constituem harmônicos, dependendo de quantos nós são formados. No segundo caso, é necessária uma dada combinação de frequência de oscilação e comprimento para haver a formação de ondas estacionárias, sendo notório ao realizar-se o experimento com um brinquedo que vibra a uma frequência fixa, preso à uma corda que pode ter seu comprimento efetivo alterado, dependendo de onde é segurada. Ademais, é relevante ressaltar a importância dos harmônicos, por exemplo na área do estudo da música, uma vez que as notas musicais são formadas por composição de harmônicos.

Por fim, o quarto experimento visa analisar um caso de oscilador harmônico amortecido, sendo utilizado um pêndulo de torção em meio de ar e posteriormente sob óleo. O atrito de diferentes materiais afeta o movimento oscilatório, removendo-se energia do movimento e afetando o deslocamento e frequência, como mostrado na fórmula para osciladores amortecidos:
\begin{align*}
	x(t) = Ae^{-\frac{\gamma}{2}t}\cos{(\omega t + \phi)}
\end{align*}
onde \(e\) é o número de Euler e \(\gamma\) é a densidade do fluído dividida pela massa do objeto que oscila. É esperado que ocorra um amortecimento subcrítico, por tanto o óleo quanto o ar não apresentarem grandes resistências que freiem o movimento rapidamente. Isso deve resultar em uma gradual redução na amplitude do movimento e da frequência \cite{Nussenzveig_2014}, com consequente aumento do período, sendo que deve ocorrer mais intensamente com o óleo por conta do maior atrito.

