\section{Discussões}

\subsection{Determinação da Capacitância Real}
O objetivo desta etapa experimental foi determinar a capacitância real do componente a partir da análise da resposta transiente do circuito RC série. A constante de tempo de relaxação ($\tau$) foi obtida medindo-se, via cursores no osciloscópio, o intervalo de tempo necessário para que a tensão no capacitor decaísse para $1/e$ (aproximadamente $36,8\%$) de seu valor inicial.

O valor experimental obtido foi:
\begin{align*}
	\tau_{exp} = 124 \, \mu s
\end{align*}

Para o cálculo correto da capacitância, é necessário considerar as impedâncias reais do circuito. Conforme discutido na análise experimental, o gerador de funções utilizado não atua como uma fonte ideal, apresentando uma impedância interna de saída ($R_{int}$) de $50 \, \Omega$.

Dessa forma, a resistência efetiva ($R_{ef}$) responsável pela carga e descarga do  capacitor é a soma da resistência nominal do circuito ($R_{ext}$) com a impedância do gerador:

\begin{align*}
	R_{ef} = R_{ext} + R_{int}
\end{align*}
\begin{align*}
	R_{ef} = 100 \, \Omega + 50 \, \Omega = 150 \, \Omega
\end{align*}

Partindo da relação teórica $\tau = R \cdot C$, isola-se a capacitância:

\begin{align*}
	C_{real} = \frac{\tau_{exp}}{R_{ef}}
\end{align*}

Substituindo os valores experimentais:

\begin{align*}
	C_{real} = \frac{124 \times 10^{-6} \, \text{s}}{150 \, \Omega} 
\end{align*}
\begin{align*}  
	C_{real} \approx 8,27 \times 10^{-7} \, \text{F}
\end{align*}


Convertendo para subunidades usuais:

\begin{align*}
	\mathbf{C_{real} \approx 827 \, \text{nF}}
\end{align*}

\section{Discussão dos Resultados}

A análise dos dados revelou uma discrepância importante entre o roteiro experimental sugerido e os resultados obtidos. Embora a montagem proposta sugerisse o uso de um capacitor de $330 \, \text{nF}$, a constante de tempo medida ($\tau = 124 \, \mu\text{s}$) é fisicamente inconsistente com este valor, uma vez que o $\tau$ esperado para $330 \, \text{nF}$ seria de aproximadamente $50 \, \mu\text{s}$ (considerando $R_{tot}=150\,\Omega$).

O valor medido de $124 \, \mu\text{s}$ aproxima-se muito mais do comportamento teórico esperado para o capacitor de \textbf{$1 \, \mu\text{F}$}, cujo $\tau$ teórico seria de $150 \, \mu\text{s}$.

Assumindo, portanto, que a medição refere-se ao componente de $1 \, \mu\text{F}$, o valor experimental calculado de \textbf{$827 \, \text{nF}$} apresenta um desvio de aproximadamente $-17\%$ em relação ao nominal. Este desvio é plenamente justificável por dois fatores, capacitores eletrolíticos ou de poliéster comuns possuem tolerâncias típicas de $\pm 20\%$, e a inclusão da resistência interna do gerador ($50 \, \Omega$) no modelo foi determinante para a precisão do resultado. Caso fosse ignorada, o cálculo resultaria em $1,24 \, \mu\text{F}$, superestimando a capacitância e mascarando o comportamento real do componente.