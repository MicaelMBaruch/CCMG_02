\section{Introdução}

O estudo de circuitos elétricos constitui uma parte essencial da compreensão dos fenômenos eletromagnéticos e de suas aplicações tecnológicas. Entre os modelos mais relevantes para descrever sistemas dinâmicos em regime de corrente alternada estão os circuitos RLC, compostos por resistores (R), indutores (L) e capacitores (C). Esses elementos, quando combinados, permitem analisar processos de armazenamento, dissipação e transferência de energia, bem como fenômenos oscilatórios e de ressonância.

Além dos circuitos RLC completos, é comum estudar configurações reduzidas formadas por pares de elementos, como os circuitos RC e RL. No circuito RC, a interação entre o resistor e o capacitor determina a forma como a carga elétrica se acumula e se descarrega. Já no circuito RL, a combinação entre resistência e indutância governa processos de crescimento e decaimento de corrente, influenciados pelo armazenamento de energia no campo magnético. Em conjunto, esses três tipos — RC, RL e RLC — podem ser organizados em arranjos série, paralelo ou estruturas mistas, onde a distribuição dos elementos modifica de maneira significativa o comportamento dinâmico do sistema.

Nesse contexto, dois conceitos desempenham papel central: \textit{capacitância} e \textit{impedância}. A capacitância caracteriza a habilidade de um capacitor em armazenar energia na forma de campo elétrico, influenciando diretamente a resposta temporal e em frequência de circuitos que o empregam, como RC e RLC. Já a impedância, entendida como a generalização da resistência para circuitos em corrente alternada, incorpora não apenas efeitos dissipativos, mas também componentes reativos associados ao armazenamento de energia elétrica e magnética. Assim, a análise da impedância total de um circuito RLC permite prever seu comportamento frente a diferentes frequências de operação, destacando fenômenos como atenuação, defasagem e ressonância\cite{Nussenzveig1997}.

Neste relatório, será realizada a estimativa da capacitância real do capacitor presente em um \textit{Circuito RC}, bem como a determinação da impedância equivalente do \textit{Circuito RLC} proposto, considerando sua topologia específica. A partir desses resultados, busca-se compreender como a capacitância efetiva se relaciona com o comportamento observado experimentalmente e de que maneira a configuração do circuito influencia suas propriedades elétricas.
