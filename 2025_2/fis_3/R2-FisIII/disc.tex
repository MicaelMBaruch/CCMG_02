\section{Modelo descritivo do fenômeno}

O comportamento dos anéis saltadores pode ser compreendido de forma mais clara quando se adota um modelo descritivo que privilegia as etapas físicas essenciais do processo, conectando a dinâmica eletromagnética transitória ao movimento mecânico subsequente. A sequência de eventos, desde o acionamento da corrente alternada na bobina até o instante em que o anel atinge sua altura máxima, evidencia a interação cooperativa entre indução eletromagnética, defasagens entre corrente e força eletromotriz, e a dinâmica gravitacional. A \cref{fig:esquema} apresenta um esquema típico do arranjo, destacando as linhas de fluxo, as correntes envolvidas e a geometria fundamental do sistema.

\begin{figure}[H]
\centering
\includegraphics[width=0.38\linewidth]{fig/esquema.jpg}
\caption{Esquema do Anel de Thomson. (1) corrente alternada na bobina; (2) linhas de fluxo magnético originadas no núcleo; (3) corrente induzida no anel de alumínio; (4) campo magnético gerado pela corrente induzida. Fonte: MUSEU DAS COMUNICAÇÕES, Anel Saltitante.}
\label{fig:esquema}
\end{figure}

\subsection*{Estabelecimento do fluxo e indução no anel}

Ao ser acionado o interruptor, o solenoide passa a ser percorrido por uma corrente alternada $i_1(t)$. Em virtude de suas variações temporais, essa corrente produz um fluxo magnético igualmente alternado no núcleo ferromagnético. A variação desse fluxo através da área ocupada pelo anel metálico dá origem, pela lei de Faraday, a uma força eletromotriz induzida $\varepsilon(t)$, que por sua vez gera uma corrente $i_2(t)$ no anel condutor.

A polaridade dessa força eletromotriz é regida pela lei de Lenz: sempre que $i_1(t)$ cresce, o fluxo magnético correspondente aumenta, produzindo uma $\varepsilon$ de sinal contrário à variação e, portanto, uma corrente induzida que se opõe ao crescimento de $i_1$. De modo complementar, quando $i_1(t)$ decresce, o fluxo diminui e a $\varepsilon$ surge agora com o mesmo sentido de $i_1$, tentando impedir essa redução. Assim, o anel tenta continuamente restaurar o fluxo original, opondo-se às mudanças impostas pela bobina.

\subsection*{Defasagens entre corrente e força eletromotriz: origem da repulsão líquida}

A interação entre as correntes $i_1$ (na bobina) e $i_2$ (no anel) determina a força magnética que atua sobre o anel. Em um cenário ideal sem perdas, a alternância entre fases de atração e repulsão resultaria em uma força média nula. Entretanto, no caso real, dois efeitos de defasagem rompem essa simetria:

\begin{enumerate}
    \item \textbf{Defasagem entre $\varepsilon$ e $i_2$ no anel.}  
    Como o anel possui resistência e autoindutância próprias, sua corrente não acompanha instantaneamente a força eletromotriz induzida. Há um atraso temporal: mesmo quando $\varepsilon$ muda rapidamente de sinal, $i_2$ demora um intervalo finito para inverter-se completamente. Essa ``inércia elétrica'' faz com que o anel responda de maneira retardada à variação de fluxo.

    \item \textbf{Defasagem entre $\varepsilon$ e $i_1$.}  
    Em um circuito indutivo alimentado por corrente alternada, a força eletromotriz induzida e a corrente da bobina não estão em fase; em regime quase senoidal, a defasagem aproxima-se de $\pi/2$ (um quarto de ciclo). Como a $\varepsilon$ induzida está diretamente relacionada à derivada temporal de $i_1$, essa defasagem é inevitável.
\end{enumerate}

A combinação dessas duas defasagens faz com que o intervalo de repulsão seja mais intenso e mais eficiente do que o intervalo de atração dentro de um ciclo completo. Assim, embora atração e repulsão ocorram alternadamente, a interação média resultante é repulsiva, produzindo a força líquida responsável pelo salto do anel.

\begin{figure}[H]
\centering
\includegraphics[width=0.35\linewidth]{fig/repulsao.png}
\caption{Intervalos de tempo em que os sinais entre as correntes resultam em atração ou repulsão pelos campos magnéticos. Fonte: Silveira, 2013}
\label{fig:repulsao}
\end{figure}

\subsection*{O impulso magnético inicial}

Quando o sistema é ligado, não há ainda regime permanente: a corrente $i_1$ cresce rapidamente a partir do repouso. Esse crescimento abrupto produz uma variação de fluxo particularmente intensa, induzindo no anel uma corrente $i_2$ também intensa. Nesse instante inicial, as correntes $i_1$ e $i_2$ têm sentidos opostos, e a interação entre seus campos magnéticos gera uma força repulsiva vertical muito maior do que qualquer força atuante no regime alternado subsequente.

Esse período de forte aceleração, de curta duração, fornece ao anel quase toda a energia mecânica observada. Trata-se, portanto, de um verdadeiro \textit{impulso magnético}, no qual a transferência de energia elétrica para energia cinética é maximizada.

\subsection*{Perda rápida de acoplamento e movimento por inércia}

À medida que o anel começa a subir, ele se afasta da região de maior densidade de fluxo. Esse deslocamento reduz rapidamente o fluxo magnético que o atravessa, a intensidade da corrente induzida $i_2$ e, consequentemente, a força magnética atuante. Poucos centímetros acima da bobina, a força já é praticamente nula. A partir desse ponto, o anel move-se apenas sob a ação da gravidade, como um projétil lançado verticalmente. A altura máxima atingida depende exclusivamente da velocidade inicial adquirida durante o impulso magnético.
Após a queda, entretanto, o anel não retorna ao contato pleno com a bobina. Devido à corrente alternada que continua circulando no solenoide, pequenas variações residuais de fluxo ainda induzem correntes no anel enquanto ele se aproxima do núcleo. Esses sinais de corrente, embora fracos, encontram a corrente da bobina defasada de modo a produzir uma força repulsiva média suficiente para equilibrar o peso do anel a uma curta distância da bobina. Como resultado, o anel passa a repousar em uma posição de equilíbrio estável ligeiramente acima da superfície do solenoide, sustentado pela repulsão eletromagnética residual.

\subsection*{Dependência da massa e comparação entre os anéis}

A diferença de comportamento entre os anéis leves e o anel pesado segue diretamente dessa dinâmica. Os dois anéis da esquerda possuem massa $m$ e recebem impulsos quase idênticos devido ao forte acoplamento inicial. O anel da direita, com massa $m' = 2m$, não recebe um impulso duas vezes maior, pois a intensidade da corrente induzida é limitada pela geometria e pelo acoplamento magnético. Assim, a mesma energia transferida precisa ser distribuída por uma massa maior, produzindo uma velocidade inicial menor e, portanto, uma altura final mais baixa.

\subsection*{Interação entre anéis empilhados}

Quando dois anéis idênticos estão empilhados sobre a mesma bobina, surge um acoplamento magnético adicional entre eles. O anel inferior é o primeiro a ser fortemente induzido; entretanto, o anel superior sofre tanto a influência direta do fluxo ascendente quanto do campo criado pela corrente no anel inferior, de modo que ambos funcionam como bobinas secundárias mutuamente acopladas. Esse acoplamento redistribui o fluxo entre eles e reduz assimetrias no impulso recebido, resultando no salto praticamente simultâneo observado no experimento.

\subsection*{Interpretação energética}

Sob a perspectiva da conservação de energia, o salto do anel representa a conversão temporária de energia elétrica em energia mecânica. Parte da energia fornecida ao solenoide transforma-se em energia cinética no instante do impulso, parte em energia potencial gravitacional durante a subida, e parte é dissipada como calor na bobina, no anel e no núcleo ferromagnético, devido à resistência elétrica e às correntes de Foucault.