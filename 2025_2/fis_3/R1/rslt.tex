\section{Resultados}
\subsection{Observação da força de repulsão elétrica entre dois canudos carregados}

    Seguindo os procedimentos detalhados na metodologia, obtemos, após análise pelo Tracker, a \cref{canudo}. 
    \begin{figure}[H]
        \centering
        \includegraphics[width=.5\linewidth]{figs/canudo.png}
        \caption{Configuração experimental dos canudos carregados suspensos como pêndulos, mostrando a distância de equilíbrio entre eles.}
        \label{canudo}
    \end{figure}
    Observe que o canudo superior ficou suspenso a uma distância de \qty{3.669}{\centi\meter} do canudo inferior. Esse fenômeno será utilizado para determinar a ordem de grandeza da carga nos canudos.
\\ 
Seguindo a medida da separação entre os canudos (\(r = \qty{3.669}{\centi\meter}\)). Além disso, a massa do canudo foi previamente estabelecida pelo lab demo como sendo de \qty{0.2e-3}{kg}.
