\section{Metodologia}

\subsection{Gerador de Van de Graaff}
O experimento foi realizado com um gerador de Van de Graaff de bancada, composto por esfera metálica superior, coluna isolante e sistema de correias acionado por manivela. Utilizou-se também um eletrodo esférico auxiliar sobre base isolante para observação das descargas elétricas. A \cref{fig:vander} apresenta o equipamento utilizado.

\begin{figure}[H]
\centering
\includegraphics[width=0.35\linewidth]{figs/vander.png}
\caption{Gerador de Van de Graaff e eletrodo auxiliar utilizados no experimento. Fonte: registro fotográfico realizado durante a prática.}
\label{fig:vander}
\end{figure}

\subsection{Canudos carregados}

Dois canudos de plástico foram atritados com papel de mão comum. Um dos canudos foi suspenso na vertical e observou-se que ficou perpendicular a mesa em que o suporte estava apoiado. A suspensão foi de tal forma que o canudo pudesse rotacionar livremente no eixo paralelo ao da mesa, no qual estava suspenso. Então, o segundo canudo, foi aproximado do primeiro até que ambos estivessem paralelos a mesa e em equilíbrio de forças. O aparato experimental pode ser visualizado na \cref{canudos}

\begin{figure}[htb]
    \centering
    \includegraphics[width=.5\linewidth]{figs/canudos}
    \caption{Fotografia representativa do experimento dos canudos. Fonte: registro fotográfico realizado durante a prática}\label{canudos}
\end{figure}

