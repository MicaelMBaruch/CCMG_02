\section{Introdução}

O presente trabalho simula o cenário proposto por Lynch e colaboradores para investigar o papel da deriva genética no aumento da complexidade genômica em eucariontes multicelulares[cite: 3].

\subsection{Interesse Biológico}
O interesse biológico do cenário é compreender os mecanismos evolutivos que levam à grande diferença na arquitetura genômica entre procariontes e eucariontes. A hipótese central é que a ineficiência da seleção natural em populações pequenas (alta deriva) permite a fixação de mutações levemente deletérias que aumentam o tamanho e a complexidade do genoma.

\subsection{Exemplo de Realismo Biológico}
Um exemplo clássico é a comparação entre o tamanho e a proporção de DNA não-codificante nos genomas de bactérias (populações grandes) e organismos multicelulares (populações tipicamente menores).

\subsection{Extensão de Cenários Simples}
Este cenário estende os modelos simples de genética populacional ao incorporar a variação no tamanho do genoma e um espectro de alelos neutros ou com desvantagens evolutivas, tornando o fitness um parâmetro dinâmico e probabilístico.
	

