\section{Estratégias de Simulação}

\subsection{Estratégia Geral}
A simulação utiliza um modelo de gerações sucessivas com cruzamentos, onde o genótipo possui tamanho variável e está sujeito a mutação e duplicação de genes. O teste principal consiste em comparar a evolução da complexidade genômica em populações de diferentes tamanhos.

\subsection{Parâmetros Controlados e Fixos}
\subsubsection*{Parâmetros Fixos}
Os seguintes parâmetros foram mantidos constantes:
\begin{itemize}
	\item Chance de duplicação de gene ($x$)[cite: 8].
	\item Chance de mutação ($y$).
	\item O alfabeto de possíveis alelos (o *genepool*) com suas respectivas desvantagens evolutivas ou neutralidade.
\end{itemize}

\subsubsection*{Parâmetro Controlado}
O principal parâmetro de controle é o tamanho da população, testado em valores progressivamente maiores.