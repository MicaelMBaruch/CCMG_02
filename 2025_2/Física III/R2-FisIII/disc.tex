\section{Discussão do fenômeno observado}

Ao ligar a corrente em cada um dos sistemas, notou-se que, no sistema com duas boninas de massa $m$ os anéis saltaram até a altura máxima permitida pela montagem experiemntal, colidindo com o suporte supeior do equipamento, enquanto o sistema com massa $m'$ saltou até aproximadamente metade da altura permitida pela estrutura do experimento. Ambos os resultados estão representados na cref{fig:resultados}

\begin{figure}[H]
\centering
\includegraphics[width=0.35\linewidth]{fig/resultados.png}
\caption{Comparação das alturas máximas atingidas pelos anéis durante o experimento dos anéis saltitantes. Em A, os dois anéis idênticos (massa 
$m$) atingem cerca de 1,0m, colidindo com o suporte superior do equipamento. Em B, o anel de maior massa ($m'$) alcança 0,46m de altura. As distâncias foram determinadas considerando cuidadosamente o efeito de paralaxe nas medições. Fonte: registro fotográfico realizado durante a prática.}
\label{fig:resultados}
\end{figure}

O comportamento do anel nesse experimento resulta da interação entre fenômenos eletromagnéticos iniciais e a consequente dinâmica mecânica. Quando a corrente alternada percorre a bobina, o fluxo magnético que atravessa o anel varia rapidamente. De acordo com a lei de Faraday-Lenz,

\[
\mathcal{E} = -\frac{d\Phi}{dt},
\]

uma força eletromotriz é induzida no anel, gerando uma corrente cujo sentido se opõe à variação do fluxo. Dessa forma, o anel comporta-se como um circuito secundário acoplado magneticamente à bobina primária, de forma análoga a um transformador.

A corrente induzida produz um campo magnético próprio, que interage com o campo da bobina. A oposição entre esses campos dá origem a uma força vertical repulsiva $F_z$, responsável por impulsionar o anel para cima. A energia transferida ao anel durante esse intervalo inicial é

\[
E = \int F_z \, dz,
\]

representando o trabalho realizado pelo campo magnético. Enquanto o anel permanece próximo ao núcleo, o acoplamento magnético é forte e a corrente induzida é significativa, concentrando nesse curto intervalo praticamente toda a transferência de energia.

À medida que o anel se afasta, o acoplamento magnético $\Phi(z)$ diminui rapidamente, aproximadamente de forma exponencial. Com isso, a corrente induzida se reduz e a força magnética tende a zero, fazendo com que o movimento subsequente seja governado apenas pela gravidade. Toda a energia transmitida no impulso inicial transforma-se em energia potencial gravitacional na altura máxima:

\[
E = m g h_{\max} 
\quad \Rightarrow \quad
h_{\max} = \frac{E}{m g}.
\]

A análise dinâmica também pode ser formulada em termos do impulso magnético:

\[
J = \int F_z \, dt,
\]

que estabelece a velocidade inicial do anel:

\[
J = m v_0 
\quad \Rightarrow \quad
v_0 = \frac{J}{m}.
\]

Considerando então o movimento vertical sob gravidade, a altura máxima atingida é

\[
h = \frac{v_0^2}{2g}.
\]

Substituindo a expressão para $v_0$, obtém-se

\[
h = \frac{1}{2g} \left( \frac{J}{m} \right)^2
= \frac{J^2}{2 m^2 g}.
\]

Essa relação mostra que, para o mesmo impulso, anéis com maior massa atingem alturas muito menores, uma vez que a altura varia inversamente com o quadrado da massa.

Além do efeito da massa, parâmetros elétricos do anel, como resistência e indutância própria, influenciam a corrente induzida e, portanto, a força inicial. Parte da energia fornecida ao sistema é dissipada como calor tanto na bobina quanto no anel. Assim, o fenômeno ilustra claramente a conservação da energia: a energia elétrica fornecida ao primário é parcialmente convertida em energia magnética, parcialmente em energia cinética do anel e parcialmente dissipada como calor.

Em resumo, o experimento evidencia de forma quantitativa a interação entre indução eletromagnética, transferência de energia e dinâmica gravitacional, permitindo relacionar diretamente impulso, energia e massa à altura máxima atingida pelo anel.


\section{Modelo descritivo do fenômeno}

A energia cinética inicial adquirida pelo anel é convertida totalmente em energia potencial gravitacional na altura máxima:

\[
\frac{1}{2} m v_0^2 = m g h,
\]

de onde,

\[
h = \frac{v_0^2}{2g}.
\]

A velocidade inicial obtém-se pelo impulso magnético:

\[
J = \int F_z\,dt = m v_0
\quad \Rightarrow \quad
v_0 = \frac{J}{m},
\]

resultando na relação final:

\[
h = \frac{J^2}{2 m^2 g}.
\]

Assim, a altura máxima é inversamente proporcional ao quadrado da massa (\[h \propto \frac{1}{m^2}\]).

\section{Análise qualitativa e quantitativa}

\subsection{Lado esquerdo — dois anéis leves}

Os anéis da esquerda possuem massa $m$ cada. Como o impulso magnético é aproximadamente igual para ambos, a velocidade inicial $v_0 = J/m$ é alta, e a altura teórica prevista seria grande. Contudo, experimentalmente ambos atingiram o suporte mecânico antes de chegarem à sua altura máxima.

\subsection{Lado direito — anel pesado}

O anel do lado direito possui massa aproximadamente $m' = 2m$. Mantendo o mesmo impulso médio $J$:

\[
h' = \frac{J^2}{2 (2m)^2 g}
= \frac{h}{4}.
\]

O anel pesado, portanto, deve atingir uma altura quatro vezes menor que a dos anéis leves — exatamente como observado: ele se levantou apenas alguns milímetros, enquanto os anéis leves saltaram visivelmente.

\subsection{Considerações adicionais}

Outros fatores contribuem:

\begin{itemize}
    \item maior resistência do anel pesado reduz a corrente induzida;
    \item efeitos de distribuição de corrente (skin effect) alteram o acoplamento;
    \item diferenças geométricas influenciam a indutância e o fluxo.
\end{itemize}

Apesar disso, a dependência dominante é a massa.