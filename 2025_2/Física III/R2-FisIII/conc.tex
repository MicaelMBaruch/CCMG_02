\section{Conclusão}

O estudo do anel de Thomson mostrou que o salto dos anéis é determinado quase integralmente pelo curto intervalo de tempo em que ocorre o forte acoplamento magnético entre a bobina e o anel. Nesse regime transitório, a rápida variação do fluxo gera uma corrente induzida intensa, cuja defasagem em relação à corrente da bobina produz uma força repulsiva média não nula. Esse impulso inicial estabelece a velocidade de lançamento, enquanto o movimento subsequente ocorre praticamente sem influência eletromagnética, sendo governado apenas pela gravidade.

A comparação entre os anéis leves e o anel de massa duplicada evidenciou a relação direta entre energia transferida, massa e altura de subida: embora todos recebam impulsos semelhantes, a maior massa reduz a velocidade inicial e, portanto, a altura atingida. Já o comportamento quase simultâneo dos dois anéis empilhados confirmou a forte distribuição de fluxo no instante inicial e o acoplamento mútuo entre eles.

Os resultados experimentais estão em acordo com o modelo proposto, demonstrando que a análise conjunta do regime transitório de indução, das defasagens típicas de circuitos indutivos e da subsequente dinâmica gravitacional fornece uma descrição completa e consistente do fenômeno observado.
