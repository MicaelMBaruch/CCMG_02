\section{Descrição do experimento}

O experimento consistiu em duas bobinas equivalentes montadas sobre núcleos ferromagnéticos cilíndricos. Sobre a bobina da esquerda foram posicionados dois anéis condutores idênticos, cada um com massa $m$, enquanto sobre a bobina da direita foi colocado um único anel com massa $m' = 2m$. A cref{fig:salto} ilustra a disposição experimental descrita.

\begin{figure}[H]
\centering
\includegraphics[width=0.35\linewidth]{fig/salto.png}
\caption{Arranjo experimental das bobinas com seus respectivos anéis condutores utilizados para demonstrar o fenômeno dos anéis saltitantes. Fonte: registro fotográfico realizado durante a prática.}
\label{fig:salto}
\end{figure}

Ao ligar a corrente alternada, foi observado a elevação vertical ("salto") dos anéis, sendo a altura atingida pelo sistema da esquerda substancialmente maior do que o observado no sistema da direita.
