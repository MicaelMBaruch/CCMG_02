\section{Descrição do experimento}

O experimento consistiu em duas bobinas equivalentes montadas sobre núcleos ferromagnéticos cilíndricos. Sobre a bobina da esquerda foram posicionados dois anéis condutores idênticos, cada um com massa $m$, empilhados verticalmente. Sobre a bobina da direita posicionou-se um único anel com massa $m' = 2m$, geometricamente equivalente aos dois anéis menores ``colados'' um ao outro (mesmo material, mesmo raio, maior espessura).

A \cref{fig:salto} ilustra a disposição experimental utilizada.

\begin{figure}[H]
\centering
\includegraphics[width=0.35\linewidth]{fig/salto.png}
\caption{Arranjo experimental das bobinas com seus respectivos anéis condutores utilizados para demonstrar o fenômeno dos anéis saltitantes. Fonte: registro fotográfico realizado durante a prática.}
\label{fig:salto}
\end{figure}

Ao ligar a corrente alternada, observou-se a elevação vertical dos anéis. Os dois anéis leves da esquerda atingiram a altura máxima permitida pela estrutura, colidindo com o suporte superior, enquanto o anel pesado da direita alcançou cerca de metade dessa altura, como mostrado em \cref{fig:resultados}.

\begin{figure}[H]
\centering
\includegraphics[width=0.35\linewidth]{fig/resultados.png}
\caption{Alturas máximas atingidas pelos anéis durante o experimento. Em A, os dois anéis idênticos (massa $m$) atingem o suporte superior de cerca de 1,0\,m. Em B, o anel de maior massa ($m'=2m$) atinge aproximadamente 0,46\,m. As distâncias foram determinadas considerando cuidadosamente o efeito de parallax nas medições. Fonte: autoral}
\label{fig:resultados}
\end{figure} 
