\section{Introdução}

A interação entre campos magnéticos variáveis e condutores eletricamente fechados produz correntes induzidas, que por sua vez geram campos que se opõem à variação do fluxo. Esse princípio, descrito pela Lei de Lenz, é responsável por uma série de fenômenos como frenagem magnética, aquecimento por indução e o conhecido experimento do \emph{anel saltador}.

O presente relatório descreve em detalhes o modelo físico subjacente, a análise qualitativa e quantitativa das forças envolvidas, e discute o comportamento observado em uma montagem experimental com três anéis: dois anéis idênticos do lado esquerdo, que atingem uma altura maior, e um anel do lado direito, de massa dobrada, que salta menos. Serão abordados também aspectos descritivos, limitações e interpretações energéticas.



