\section{Introdução}

A interação entre campos magnéticos variáveis e condutores eletricamente fechados produz correntes induzidas que, por sua vez, geram campos capazes de se opor à variação do fluxo magnético que as originou. Esse princípio, formalizado pela Lei de Lenz, está na base de diversos fenômenos eletromagnéticos, como frenagem magnética, aquecimento por indução e o clássico experimento dos \textit{aneis saltitantes} de Thomson.

O presente relatório tem como objetivo analisar esse fenômeno sob uma perspectiva física abrangente, articulando os processos transitórios de indução eletromagnética com a dinâmica mecânica subsequente. Para isso, discute-se uma montagem experimental composta por três anéis condutores: dois anéis idênticos posicionados sobre uma bobina e um terceiro anel, equivalente à junção dos dois primeiros (mesmo material, mesmo raio e massa dobrada), colocado sobre uma segunda bobina. A partir do comportamento observado — incluindo a diferença nas alturas atingidas, a repulsão resultante da defasagem entre correntes e forças eletromotrizes, e a interação mútua entre os anéis empilhados — desenvolve-se um modelo descritivo que enfatiza o impulso magnético inicial, a rápida perda de acoplamento magnético durante a ascensão e a interpretação energética do processo. O conjunto dessas análises permite compreender não apenas o salto dos anéis, mas também as sutilezas associadas ao acoplamento eletromagnético e às defasagens típicas de sistemas alimentados por corrente alternada.



