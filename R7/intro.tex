\section{Introdução}


A Revolução Industrial foi um período de intensa transformação social e tecnológica, marcada pela o advento de diversos mecanismos capazes de inovar os meios de produção e transporte, criando um novo dinamismo comercial e contribuindo à lógica mercantil. Tal fenômeno foi em grande parte possibilitado pelo uso de máquinas térmicas, como a máquina a vapor, aprimorada e patenteada por James Watt em 1769, que permitiu aceleração do desenvolvimento industrial, além da colocação dos motores a vapor como dominantes nas indústrias e transportes até o século XX \cite{HistoriaEM}.

Tendo em vista melhor compreensão de fenômenos de expansão térmica e mudança de fase, essenciais para o funcionamento de máquinas a vapor, foi realizado um experimento com um bulbo de vidro que tinha um canudo em sua ponta. Após ser aquecido era colocado em água, esperando-se observar subida do líquido conforme o instrumento esfriasse, devido a mudança de densidade e pressão do ar, além da mudança de fase da água entre líquido e vapor.

Vale ressaltar que muitos fenômenos térmicos relacionados a gases, como os citados acima, podem ser explicados por meio da Lei dos Gases Perfeitos, que trata-se de uma boa aproximação para a maioria dos gases \cite{Nussenzveig_2014}:
\begin{align*}
    PV = nRT
\end{align*}
em que P é a pressão, V o volume, n o número de moles, R a Constante Universal dos Gases e T a temperatura. Além disso, em alguns experimentos discutidos neste relatório envolvem o fenômeno da vaporização, sendo este uma mudança no estado de agregação de líquido para vapor, que pode ocorrer graças a um ganho de energia térmica pelo material, que cause redução nas forças de coesão de suas partículas \cite{FisicaEM}.

Em seguida foi feita uma experimentação em que um pistão conectado a um disco em uma ponta era movido graças ao sopro que recebia na outra extremidade. Ela permitiu análise do movimento e forças necessárias para o funcionamento de pistões, essenciais para a maior parte dos motores de máquinas térmicas.

Quanto a esses maquinários, foi observado o funcionamento de três tipos, sendo eles posteriormente comparados quanto a seus ciclos, vantagens e desvantagens. O primeiro, um Motor de Heron (ou Eolípila) é uma recriação do instrumento inventado por Hero de Alexandria do século I AC, sendo o primeiro motor a vapor criado, embora não tivesse uma aplicação prática \cite{Enciclopedia}. A Eolípila envolve uma esfera com água que é aquecida e dois escapamentos para o vapor, que fazem-na girar pelo princípio de foguete, em que a saída de gás por uma abertura causa uma propulsão do objeto na direção oposta \cite{Nasa}.

A segunda máquina térmica abordada foi o Motor Stirling, patenteado por Robert Stirling em 1816 \cite{lopes2022motor}. Foram observados dois pequenos modelos. Esse sistema depende de uma fonte de calor que aquece um gás, causando sua expansão e impulsionamento de um êmbolo, com o gás sendo então resfriado por uma fonte fria e contraído para começar o próximo ciclo. Sendo que, para uma máquina térmica cíclica, é essencial ter-se ao menos dois reservatórios, por conta do fenômeno descrito pelo enunciado de Kelvin \cite{Nussenzveig_2014}: “ É impossível realizar um processo cujo único efeito seja remover calor de um reservatório térmico e produzir uma quantidade equivalente de trabalho”. 

Por fim, foi feita uma experimentação com uma máquina térmica similar a uma locomotiva. Nela, uma fonte de calor aquece uma caldeira, cujo vapor liberado move pistão de forma a girar uma roda. Vale notar que, entre as três máquinas analisadas, é a que representa o modelo mais utilizado durante a Revolução Industrial, em especial nas locomotivas.
