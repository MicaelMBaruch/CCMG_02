\section{Conclusão} 
Com base nos experimentos, aprofundou-se a compreensão sobre os princípios das máquinas térmicas. A prática com o balão de vidro ilustrou a Lei dos Gases Ideais, onde o resfriamento do ar e a condensação do vapor criaram uma diferença de pressão que succionou a água, evidenciando a importância das transições de fase em processos termodinâmicos.

O Motor de Heron demonstrou a conversão de energia térmica em cinética pela conservação de momento, com a expulsão de vapor gerando torque. A análise apontou limitações práticas de escalonamento e potência. De forma similar, o pistão de sopro serviu como modelo para a realização de trabalho por um gás, transformando a energia da expansão do ar em rotação e ilustrando o funcionamento de um ciclo aberto.

Os motores Stirling evidenciaram o funcionamento de um ciclo termodinâmico fechado, produzindo trabalho ao mover um gás entre fontes de temperaturas distintas. A vantagem da combustão externa e da flexibilidade de combustível contrasta com a dificuldade em alterar sua potência rapidamente.


Por fim, a máquina térmica funcional sintetizou os conceitos ao converter energia térmica em trabalho mecânico para girar uma roda e, subsequentemente, em energia elétrica, comprovada pelo acendimento de um LED.
