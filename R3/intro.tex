\section{Introdução}

Há muito tempo, o estudo e o ensino dos fenômenos físicos relacionados ao
comportamento de objetos na água vêm se tornando cada vez mais importantes,
tanto para o avanço da ciência quanto para aplicações práticas no cotidiano.
Desde os experimentos pioneiros de Arquimedes, que estabeleceram os princípios
da hidrostática, até as modernas investigações em mecânica dos fluidos, a
compreensão desses fenômenos tem sido fundamental para o desenvolvimento de
tecnologias como navios, submarinos e sistemas de medição de densidade.

Os estudos de Arquimedes (287--212 a.C.) estabeleceram os princípios
fundamentais da hidrostática, incluindo o conceito de empuxo\cite{meneses2018}.
Sua famosa história sobre a descoberta de um jeito de verificar a composição de
um objeto, no caso o problema da coroa de ouro, conta que em sua banheira,
observou que objetos de diferentes materiais e portanto diferentes densidades e
de diferentes volumes se comportam de forma específica na água, assim propôs que
um corpo imerso em um fluido sofre uma força vertical para cima igual ao peso do
fluido deslocado\cite{nussenzveig2014}. Então segundo o conto de Arquimedes diz
que ele mediu o peso da coroa no ar e, depois, submersa em água, determinando
seu volume pelo empuxo. Com isso, calculou sua densidade e comparou com a do
ouro puro, descobrindo que a coroa era adulterada\cite{thompson2008}. Esse
experimento mostrou como a física pode identificar materiais sem destruí-los.

Utilizando a mesma ideia da história de Arquimedes, que descobriu a constituição
da coroa do rei, no experimento sobre o princípio de Arquimedes descobriremos
qual é o material do peso de nome ``ouro'', e utilizando as leis de Newton,
poderemos assim analisar qual é a resultante de forças que agem sobre os pesos
na jarra de água. Para determinar se o peso denominado ``ouro'', utilizaremos a
fórmula da densidade
\begin{align*}
	\rho = \frac{m}{V} \label{eq:densidade}
\end{align*}
\cite{nussenzveig2014}, comparando o resultado com a densidade conhecida de
outros materiais. Se os valores forem consistentes, confirmaremos a composição
do material. Quando esse peso é imerso em água, a força resultante que age sobre
ele é dada pela diferença entre o peso real ($P = m \cdot g$) e o empuxo ($E =
\rho_{\text{fluido}} \cdot V_{\text{deslocado}} \cdot g$)
\cite{nussenzveig2014}. Se o objeto está em repouso, a força resultante é nula,
indicando equilíbrio.

E, novamente utilizando os conceitos de empuxo criados por Arquimedes, no
segundo experimento, a balança de empuxo, baseia-se num equipamento que utiliza
do conceito de Arquimedes para comparar o peso de objetos, e tendo calibrado tal
balança corretamente, é possível medir o peso de objetos em gramas. Esse mesmo
conceito, também será utilizado para o terceiro experimento, em que apertando ou
soltando uma garrafa fechada cheia de água haverá a movimentação de um objeto
imerso nessa garrafa, utilizando noções do conceito de densidade, $\rho =
\frac{m}{V}$, e empuxo ($E = \rho_{\text{fluido}} \cdot V_{\text{deslocado}}
\cdot g$)\cite{nussenzveig2014}, é possível explicar o fenômeno do
experimento.

No caso da gangorra com o barquinho, o equilíbrio é mantido porque o empuxo
exercido pela água sobre o barco é igual ao peso do fluido deslocado. Pelo
princípio de ação e reação, a água exerce uma força igual e oposta no
recipiente, equilibrando o sistema\cite{nussenzveig2014}. Assim, o torque total
permanece zero ($\sum \tau = 0$), conforme descrito pela condição de equilíbrio
\begin{align*}
	P_1 \cdot d_1 = P_2 \cdot d_2 
\end{align*}
onde $P_1$ e $P_2$ são os pesos envolvidos e $d_1$ e $d_2$ as distâncias ao
ponto de apoio.

Em síntese, este relatório utilizará as fórmulas apresentadas para analisar
fenômenos de empuxo, pressão e equilíbrio, integrando a contextualização
histórica com a quantificação experimental. As equações permitirão interpretar
observações como a identificação de materiais, a distribuição de forças em
sistemas hidrostáticos e o comportamento de fluidos sob pressão, garantindo uma
compreensão abrangente dos princípios físicos envolvidos.

%TODO: adicionar Princípio de Pascal

