\section{Resultados e discussões}

\subsection{Fenômeno de planagem}
\subsubsection{Asa de um avião}
% Magina, aqui

\subsubsection{Tubo de Pitot}
    A relação entre as medidas relativas de pressão e os pontos medidos foi
    sintetizada na \cref{erapraserumquadro-sequiserarrumevc}.
    %Aqui deveria ser um quadro, porém não tenho como criar essa estrutura agora
    %sem o risco de errar a numeração. Em prol de uma boa numeração, não irei
    %arrumar isso. Aquele que se sentir disposto pode fazê-lo.
    \begin{table}[H]
    \caption{Medidas relativas de pressão observadas}\label{erapraserumquadro-sequiserarrumevc}
    \begin{center}
    \begin{tabular}{c c c}
    \hline
    Tubo & Local & Pressão Relativa à Pressão Atmosférica \\
    \hline
    1    & Parte frontal da asa & Parcialmente positiva\\
    2    & Parte superior da asa & Negativa\\
    3    & Parte inferior da asa & Parcialmente negativa\\
    \hline
    \end{tabular}
    \end{center}
    \end{table}
    Assim, fica evidente a relação dessas medidas com o fenômeno de planagem: a
    diferença de pressões entre a parte inferior e a superior da asa geram uma
    força de pressão direcionada para cima (região com menor pressão). Além
    disso, a pressão parcialmente positiva na parte frontal da asa se relaciona
    com o ponto de estagnação, onde toda a energia cinética do fluido é
    convertida em pressão, que foi estudada no tubo de Pitot.

\subsection{Tubos com esfera}
\subsubsection{Análise qualitativa dos fluidos}
% Luan, aqui
    Após analisar os tubos fechados, observou-se que em um dos tubos a esfera se
    moveu mais lentamente que no outro. Essa diferença pode ser atribuída a
    diferença de viscosidade dos fluidos, tendo em vista que uma força de
    resistência viscosa maior seria responsável por uma velocidade terminal
    menor da esfera. Em relação ao fluido do viscosímetro, veja
    a \cref{sec:viscStokes}, onde foi realizada uma discussão mais
    detalhada da viscosidade e o movimento das esferas nesse meio.

\subsubsection{Análise quantitativa das velocidades}
% Luan, aqui
% Aguardando os vídeos
    Com a análise dos vídeos, obtemos os Gráfico 1 e Gráfico 2:

\subsection{Viscosímetro de Stokes}\label{sec:viscStokes}
% Magina, aqui
