\textbf{(1)} O fluoreto de nitrila (\ce{NO2F}) é um gás incolor e um agente oxidante forte empregado como agente fluoretante. É uma espécie molecular (não iônica) e possui estrutura planar. O estudo de suas propriedades termodinâmicas e de sua formação foi realizado em 1962 por E. Tschuikow-Row\textsuperscript{1}, o qual verificou o mecanismo de formação e sua dependência com a temperatura através de espectroscopias vibracionais (Raman e IV).\\

A reação global observada foi:

\begin{align*}
    2\,\ce{NO2}(g) + \ce{F2}(g) \rightarrow 2\,\ce{NO2F}(g)
\end{align*}

 Essa reação, no entanto, apresentou um mecanismo complexo cuja lei de velocidade é dada por:

\begin{align*}
    v = k[\ce{NO2}][\ce{F2}]
\end{align*}

e o estudo de suas propriedades termodinâmicas apresentou variação da constante de equilíbrio ($K_{\text{eq}}$) em função da temperatura na etapa determinante (etapa lenta), conforme (ln$K_{\text{eq}}$) na tabela abaixo:

\begin{center}
\begin{tabular}{|c|c|c|c|c|c|c|c|c|c|}
\hline
\textbf{T / K} & 200 & 300 & 400 & 500 & 600 & 700 & 800 & 900 & 1000 \\
\hline
\textbf{ln $K_{\text{eq}}$} & 7,55 & 14,43 & 17,84 & 19,85 & 21,16 & 22,08 & 22,72 & 23,28 & 23,67 \\
\hline
\end{tabular}
\end{center}

\bigskip

Com todas essas informações em mãos:\\

\textbf{a)} Proponha o mecanismo de reação para a formação do \ce{NO2F} e explicite a etapa determinante;\\

\textbf{Resposta:} 

Para propor o mecanismo de reação para a formação do \ce{NO2F}, partimos da equação global observada:
\begin{align*}
2\ \ce{NO2}(g) + \ce{F2}(g) \rightarrow 2\ \ce{NO2F}(g)
\end{align*}

Embora a reação pareça simples, estudos experimentais revelaram que ela ocorre por um mecanismo mais complexo, sendo a sua velocidade regida pela expressão:
\begin{align*}
v = k[\ce{NO2}][\ce{F2}]
\end{align*}

Essa lei de velocidade indica que a etapa determinante (ou etapa lenta) do processo envolve uma colisão entre uma molécula de \ce{NO2} e uma molécula de \ce{F2}, ou seja, é uma etapa bimolecular. A partir disso, propõe-se o seguinte mecanismo:

\textit{Etapa lenta (determinante):}
\begin{align*}
\ce{NO2 + F2 <=> [NO2 \cdots F2]^{\ddagger}}
\end{align*}

Nesta etapa, as moléculas colidem e formam um complexo ativado de alta energia, sendo essa a etapa que controla a velocidade global da reação.

\textit{Etapa rápida subsequente:}
\begin{align*}
\ce{[NO2 \cdots F2]^{\ddagger} -> NO2F + F}
\end{align*}

Aqui ocorre a quebra da ligação F–F e formação de uma nova ligação N–F, liberando um radical flúor.

\textit{Etapa final:}
\begin{align*}
\ce{F + NO2 -> NO2F}
\end{align*}

O radical flúor gerado na etapa anterior rapidamente reage com uma nova molécula de \ce{NO2} para formar mais uma molécula de \ce{NO2F}, completando a reação global.\\

% Possivelmente depois eu vou escrever a soma das equações e a formação da equação global aqui, só preciso saber como fazer isso ficar bonitinho (tentei e ficou uma bosta)

Dessa forma, o mecanismo é consistente tanto com a estequiometria global quanto com a lei de velocidade observada experimentalmente, e evidencia que a etapa determinante é a formação do intermediário a partir de \ce{NO2} e \ce{F2}.

\vspace{0.4cm}

\textbf{b)} Proponha a estrutura do complexo ativado para essa reação bimolecular;\\

\textbf{Resposta:} 

Para representar a estrutura do complexo ativado dessa reação bimolecular, é necessário considerar que, durante a colisão entre \ce{NO2} e \ce{F2}, ocorre uma reorganização eletrônica na qual uma das ligações F–F começa a se romper ao mesmo tempo que uma nova ligação N–F começa a se formar. Esse estado de transição pode ser representado por uma estrutura em que os átomos ainda não estão completamente ligados, caracterizando o chamado “complexo ativado” ou “estado de transição”.

A estrutura pode ser descrita como:
\begin{align*}
\ce{[NO2 \cdots F-F]^{\ddagger}}
\end{align*}

Nesse intermediário, o flúor mais próximo do nitrogênio do \ce{NO2} apresenta uma ligação parcialmente formada com o N, enquanto a ligação F–F encontra-se parcialmente rompida. Além disso, a geometria ao redor do nitrogênio provavelmente se torna levemente distorcida em relação ao plano original da molécula, dada a reorganização eletrônica em curso. Essa estrutura representa o ponto de maior energia no caminho da reação, ou seja, o topo da barreira energética que os reagentes devem superar para formar os produtos.

\vspace{0.4cm}

\textbf{c)} Apresente a curva de energia potencial de formação do \ce{NO2F}, apontando a região da curva onde ocorre a formação do complexo ativado. De acordo com suas observações, essa reação é endotérmica ou exotérmica? Explique com o devido formalismo.\\

\textbf{Resposta:} 

Com base nos dados fornecidos sobre a constante de equilíbrio (\(K_\text{eq}\)) da etapa determinante em diferentes temperaturas, é possível analisar a natureza energética do processo. A tabela mostra que o valor de \(\ln(K_\text{eq})\) aumenta com o aumento da temperatura, o que implica que a constante de equilíbrio também cresce. Isso pode ser interpretado por meio da equação de Van’t Hoff:
\begin{align*}
\ln K = -\frac{\Delta H^\circ}{R} \cdot \frac{1}{T} + \frac{\Delta S^\circ}{R}
\end{align*}

Segundo essa equação, se \(\ln(K)\) aumenta com a elevação da temperatura, isso indica que \(\Delta H^\circ\) é positivo, pois somente assim o termo \(-\frac{\Delta H^\circ}{R} \cdot \frac{1}{T}\) se torna menos negativo à medida que \(T\) cresce. Isso significa que a etapa determinante do mecanismo é endotérmica, ou seja, consome energia do meio para ocorrer.

% Ainda preciso gerar esse gráfico, caso alguém queira me dar uma forcinha, seria bem grato

Essa análise pode ser representada por um diagrama de energia potencial em função da coordenada de reação. Nesse gráfico, os reagentes se encontram em um patamar energético inicial. À medida que a reação prossegue, a energia do sistema aumenta até atingir um máximo, correspondente ao complexo ativado — o ponto mais instável e de maior energia do processo. Após ultrapassada essa barreira, a energia diminui com a formação dos produtos.

Como a energia dos produtos é maior do que a dos reagentes, a variação de entalpia total da reação (\(\Delta H\)) é positiva, reforçando que o processo é endotérmico. A energia de ativação (\(E_\text{a}\)) corresponde à diferença entre os reagentes e o complexo ativado, e sua magnitude está relacionada à taxa com que a reação ocorre: quanto maior a \(E_\text{a}\), mais lenta é a reação, justificando a sua presença como etapa determinante.

Portanto, a partir dos dados fornecidos e da análise do gráfico de energia, conclui-se que a formação do \ce{NO2F} apresenta uma etapa lenta que é endotérmica, com formação de um complexo ativado de alta energia, e que a reação global ocorre mediante superação dessa barreira, em consonância com o modelo energético das reações químicas e o formalismo termodinâmico.
