\section{Metodologia}

\subsection{Princípio de Arquimedes}
Para realização deste experimento são utilizados: uma jarra com água, uma balança digital e um sólido de dimensões significativas. Primeiro, coloca-se a jarra com água sobre a balança digital, então o sólido é submerso na água. Por fim, registra-se a força peso indicada na balança.

\subsection{Balança de empuxo}
Para a demonstração é necessário confeccionar uma balança de empuxo. A balança de empuxo consiste em um cilíndro maior, no qual deve-se encaixar um cilindro menor de densidade consideravelmente inferior à da água e diâmetro próximo ao diâmetro do cilindro maior sobre o qual fica preso um prato. Para medir o peso de um objeto é importante que exista uma graduação de altura. O objeto a ser estudado deve ser apoiado sobre o prato e o nível da água deve ser anotado após a balança entrar em equilíbrio estático. Posteriormente é possível obter o peso do objeto através da variação no nível da água.

\subsection{Gangorra}
Neste experimento é utilizada uma bacia paralelepipédica parcialmente preenchida com água. A bácia é apoiada sobre um paralelepípedo de comprimento consideravelmente menor que o comprimento da bacia, de forma que ela possa ser desequilibrada na presença de torque. No meio da bacia coloca-se uma barreira de comprimento delgado, porém profundidade próxima de dois terços da profundidade da própria bacia e altura igual a altura da bacia. A experimentação consiste em apoiar um pote de desnidade menor que a água de um dos lados da bacia, de maneira que fique suspenso. Então, uma pedra (ou outro sólido mais denso que a água) é colocado hora dentro do pote suspenso, hora diretamente na bacia. Deve-se observar os efeitos de colocar a pedra dentro do pote ou diretamente na bacia. 

\subsection{Bonecos flutuantes}
Para esta demonstração, colocam-se dois bonecos em uma garrafa tampada quase completamente preenchida com água. Um dos bonecos é de material pouco denso enquanto o outro de material mais denso que a água. O boneco pouco denso deve ter um gancho que ficará com a concavidade para cima, enquanto o boneco muito denso deve ter um gancho com concavidade para baixo. Deve-se observar de que forma, sem abrir a garrafa, é possível fazer ambos os bonecos subirem juntos. 

%TODO: revisar e colocar imagens

