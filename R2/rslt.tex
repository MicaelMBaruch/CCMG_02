\section{resultados}
\subsection{Princípio de Arquimedes}

% TODO: Magina, coloque apenas os resultados e os erros das medidas (apenas das
% medidas)

\subsection{Balança de empuxo}

Os dados coletados foram sintetizados na \cref{tab1}
\begin{table}
    \caption{Resultados da balança de empuxo}
    \label{tab1}
    \begin{center}
        \begin{tabular}{c c}
            \hline
            Nome & Peso medido \\
            \hline
            Tara & \( (0 \pm 25) \unit{\gram} \)\\
            Peso redondo & \( (510 \pm 25) \unit{\gram} \)\\
            HD & \( (640 \pm 25) \unit{\gram} \)\\
            Celular Q. & \( (270 \pm 25) \unit{\gram} \)\\
            Celular B. & \( (250 \pm 25) \unit{\gram} \)\\
            Celular M & \( (200 \pm 25) \unit{\gram} \)\\
            Celular J. & \( (170 \pm 25) \unit{\gram} \)\\
            Celular L. & \( (220 \pm 25) \unit{\gram} \)\\
            \hline
    \end{tabular}
    \end{center}
\end{table}

\subsection{Gangorra}

% TODO: Coloque as leis e fórmulas relevantes

\subsection{Bonecos flutuantes}

Todos os integrantes do grupo foram capazes de realizar o desafio proposto, veja
a \cref{figdesafio}. Além disso, observou-se que, ao pressionar a garrafa, o
boneco submergia, retornando à posição flutuante quando a pressão era liberada.
O movimento do boneco ocorria de maneira consistente, independentemente do local
onde a garrafa era apertada, sugerindo uma relação direta com a variação de
pressão no interior do recipiente. O fenômeno se repetiu de forma idêntica em
todas as tentativas realizadas.


