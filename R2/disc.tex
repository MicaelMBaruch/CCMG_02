\section{Discussão}
\subsection{Princípio de Arquimedes}

\subsection{Balança de empuxo}
No experimento com a balança de empuxo, foi medida a distribuição de massa dos
celulares dos integrantes da equipe, utilizando um sistema calibrado para
determinar o empuxo correspondente a cada dispositivo. As medidas obtidas
variaram entre 170 g e 270 g, refletindo diferenças nos modelos e materiais dos
aparelhos. A balança, baseada no princípio de Arquimedes, permitiu calcular a
massa indiretamente pela relação entre o empuxo e a densidade do fluido
deslocado.

\subsection{Gangorra}

\subsection{Bonecos flutuantes}
O comportamento do boneco dentro da garrafa pode ser explicado pelos princípios
da hidrostática e da variação de pressão. Já sabemos, que o boneco possui uma
cavidade com ar em sua parte inferior, o que garante sua flutuação inicial
devido ao volume deslocado de água, que gera empuxo maior que o peso do boneco.
Quando a garrafa é pressionada, a pressão hidrostática no interior do líquido
aumenta, comprimindo o ar dentro do boneco e reduzindo seu volume de água
deslocado. Consequentemente, o empuxo diminui, fazendo com que o boneco desça.

Além disso, quanto à posição em que a garrafa é apertada, não há diferença
aparente no movimento do boneco, dado que, de acordo com o Princípio de Pascal,
a pressão aplicada em um fluido incompressível se transmite integralmente em
todas as direções. Portanto, independentemente do local onde se pressiona a
garrafa, a variação de pressão será a mesma em todo o sistema, resultando no
mesmo efeito sobre o boneco.
